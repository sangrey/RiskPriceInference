\documentclass[11pt, letterpaper, twoside, final]{article}
\usepackage{risk_price_inference}
\addbibresource{riskpriceinference.bib}


\title{Characterizing the Structure of the Identification Problem in Continuous-Time}
\author{Paul Sangrey}


\begin{document}

\maketitle

In this section, we have the following two continuous-time processes.
The goal is to show that by using the first-two conditional moments in our identification strategy, the parameters
we are estimating converge to the true equity risk and volatility risk even in the presence of higher-order
time-variation in the price processes.
Our moment conditions we use in the estimation project the risk prices onto the information set generated by
$\E\left[r_{t+1}\mvert \F_{t-1}\right]$  and $\E\left[\sigma^2_{t-1} \mvert \F_{t-1}\right]$.
What we do here is the equivalent projection in population.
This procedure will give us a series of estimates $\hat{\pi}, \hat{\theta}$, etcetera.
These estimates will have some limiting values, as long as the moment conditions are sufficiently non-collinear.
This section characterizes precisely the role these limiting values play in population.
For example, we show that the risk-prices are the actual risk prices under some conditions, which we make
explicit.

The way by which we do this is by representing the processes in continuous-time.
Since the assets we are characterizing are continuously tradable they must satisfy some continuous-time
no-arbitrage conditions. 
In other words, there exists a continuous-time DGP where the DGP we have been using so far is a discretization of. 
This is useful because we can work with the first two moments of the process effectively without loss of
generality.
The linear-affine structure we have been using before will show up as certain terms being constant functions of
time in what follows.
The cost of doing this is that it requires more involved notation and more sophisticated mathematical tools.


It is not entirely obvious that volatility risk premia is even a well-defined concept if neither the volatility
nor the price process itself jumps.\footnote{In general, the predictable projection of the instantaneous
    volatility $\sigma^2(t-)$ is measurable with respect the filtration generated by the data $\F_t$. In addition,
    if both $\sigma^2(t)$ and $p(t)$ are continuous processes, their filtrations are predictable as well: $\F_t =
    \F_{t-}$. Consequently, all of four o the relevant filtrations coincide. Hence, changing filtrations is
    possible. It is not obvious how volatility risk and market risk can be distinguished if conditioning on
    volatility at time $t$ does not change the information set from $\F^p_{t-}$. The discrete-time
characterization of volatility risk is doing precisely this by using the log-Laplace transform.}
Consequently, we will adapt the following continuous-time data-generating process (DGP)
This is arguably the simplest DGP that allows for us to discuss these issues. 
We follow \textcite{sangrey2018jumps} in modeling the jumps as an integral with respect to a variance-gamma
process.
The constraint that $\sigma^2(t) > 0$, will imply some dependence between $\sigma^2(t), \sigma_{\sigma}(t)$, and
$\gamma_{\sigma}(t)$, which we do not rule out.



\begin{defn}{Continuous-Time DGP}
    \label{defn:cont_time_dgp}

    \begin{align}
        \dif p(t) &= \gamma(t) \dif t + \beta(t) \sigma^2(t-) \dif t + \sigma(t) \dif W(t)  \\
        \dif \sigma^2(s)  &= \mu_{\sigma}(t) + \sigma_{\sigma}(t) \dif W_{\sigma}(t) +
        \frac{\gamma_{\sigma}(t)}{\sqrt{2}} \dif \Lap(t) 
    \end{align}

\end{defn}

The other complication with \cref{defn:cont_time_dgp}, is that we split the drift into two parts $\gamma(t)\dif t$
and $\beta(t) \sigma(t-) \dif t$. 
All of the terms here are finite-variation predictable processes.
It is not immediately obvious how to separate out $\gamma(t)$ and $\beta(t)$.
To get the correct discrete-time interpretation of the integrals of these processes, you can take the predictable
finite-variation (drift) of the ordinal process and project it onto $\sigma^2(t)$ without an intercept.
You can then define $\beta(t)$ to be the coefficient from this regression and $\gamma(t)$ to be the residual.
Intuitively, we are separating out the part of price drift driven by expectations of vitality from other sources
of variation.


We now define the relevant filtrations.
We will continue to use $\F_t$ to be the filtration generated by $p(t)$.
We will use $\F_{t-}$ to be the predictable filtration associated with $\F_t$, i.e.\@ the filtration generated by
the predictable processes adapted to $\F_t$.
In addition, we will define $\F_{\sigma^2, t}$ to be $\F_t$ augmented by the right-limits of the volatility. 
You can view it as $\lim_{\Delta to 0} \F_t \cup \F^{\sigma^2}_{t+\Delta}$, where $\F^{\sigma^2}_t$ is the
filtrated generated by $\sigma^2(t)$.
In other words, we will look infinitesimally far into the future path of $\sigma^2(t)$ and add that information to
our filtration.

This is the continuous-time analogue of adding $\int_t^{t+1} \sigma^2(s) \dif s$ to your $\F_t$ information set.
In both cases you are adding the value of the volatility in the \textquote{next period} to the current information
set.
Another way to view this expansion of filtration is that in $\F_t$, $\int^t_{-\infty} \sigma^2(s) \dif s$ is an
optional process, and we are constructing the minimal filtration that makes $\int^t_{-\infty} \sigma^2(s) \dif s$
predictable.\footnote{The progressive enlargement of the filtrations we are considering is an augmentation of
    $\F_t$ with a series of honest times, since $\F^{\sigma^2}_{t}$ is generated by a left-continuous processes.
    Consequently, all of the adapted processes to $\F_t$ are semimartingales with respect to $\F_{\sigma^2,t}$,
    \parencite[Theorem C]{barlow1978study}.  We will further assume that all of the processes maintain the same
stochastic jump-diffusion structure under both filtrations. The relevant technical conditions under which this
structure is preserved is outside this scope of this paper.}

\end{document}
