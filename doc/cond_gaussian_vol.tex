\documentclass[11pt, letterpaper, twoside, final]{article}
\usepackage{risk_price_inference}
\addbibresource{riskpriceinference.bib}

\author{Xu Cheng\thanks{University of Pennsylvania, The Ronald O. Perlman Center for Political Science and
    Economics, 133 South 36th Street, Philadelphia, PA 19104, \href{mailto:xucheng@upenn.edu}{xucheng@upenn.edu}}
    \and 
    Eric Renault\thanks{Brown University, Department of Economics -- Box B, 64 Waterman Street, Providence, RI
    02912, \href{mailto:eric_renault@brown.edu}{eric\_renault@brown.edu}}
    \and 
    Paul Sangrey\thanks{University of Pennsylvania, The Ronald O. Perlman Center for Political Science and
    Economics, 133 South 36th Street, Philadelphia, PA 19104, \href{mailto:paul@sangrey.io}{paul@sangrey.io}}}
    
\title{Inference for Risk Prices using Equity Data: Conditionally Gaussian Volatility Derivation}

\begin{document}

\begin{align}
    \label{eqn:beta_function}
    \beta_{\PP}'(0)  &= a_{\QQ}\left(-\psi_{\QQ} - \frac{1}{2} (1- \phi^2)\right)   \\
%
    \intertext{Some algebra leads to the following.}
%
    &= c^{2} \omega \left(\frac{\phi^{2}}{2} + \pi + \psi_{\QQ} \theta - \psi_{\QQ} - \frac{\theta^{2}}{2}
       \left(1- \phi^{2}\right) - \frac{1}{2}\right)^{2}  \\
    &\quad + c \delta \left(\frac{\phi^{2}}{2} + \pi + \psi_{\QQ} \theta - \psi_{\QQ} - \frac{\theta^{2}}{2}
      \left(1 - \phi^{2} \right) - \frac{1}{2}\right)  \nonumber
\end{align}

We can perform a similar simplification for $\gamma_{\PP}'(0)$.

\begin{align}
    \label{eqn:gamma_function}
    \gamma_{\PP}'(0)  &= b_{\QQ}\left(-\psi_{\QQ} - \frac{1}{2} (1- \phi^2)\right)  \\
%
    \intertext{Again after some algebra.}
%
        &= c^{2} \omega \left(\frac{\phi^{2}}{2} + \pi + \psi_{\QQ} \theta - \psi_{\QQ} - \frac{\theta^{2}}{2}
           \left(1 - \phi^{2} \right) - \frac{1}{2}\right)^{2}   \\
        &\quad + c \delta \left(\frac{\phi^{2}}{2} + \pi + \psi_{\QQ} \theta - \psi_{\QQ} - \frac{\theta^{2}}{2}
          \left(1- \phi^{2}\right) - \frac{1}{2}\right) \nonumber 
\end{align}

The key issue with  \cref{eqn:beta_function} and \cref{eqn:gamma_function} is that they are both truncated Taylor
series expansions. 
The questions is what does this mean?
First, if $\left. \sigma^2_{t+1} \mvert \sigma^2_t \right.$ and $\left. r_{t+1} \mvert  \sigma^2_{t+1}\right.$
are Gaussian distributed this simplification is hold exactly.
The log-cumulant function of a Gaussian variable truncates after the first two terms.

In general, this is should be viewed as an approximation.
Economically, what does this approximation do?
We start by considering the approximation for the volatility.
First, consider the case where  the higher-order moments of the distribution of $\left. \sigma^2_{t+1}
\mvert \sigma^2_t \right.$ are constant. 
In this case, there might be a risk-premia associated with these higher-order moments but this risk-premia is
constant.
We cannot have a time-varying risk-premia that is driven by a constant.
In general, even if these higher-order moments are time-varying, it will not affect our analysis here as long as
that time-variation is independent of $\sigma^2_t$.
A careful reader might notice that the constant case is a special case of this because deterministic variation is
independent  of stochastic variation.
Of course in the general case, this will not hold.
There will be higher-order variation in these moments as a function of $\sigma^2_t$. 
As long as this higher-order variation is not associated with a large amount of time-variation in risk premia our
estimation procedure will still work well in practice.

The situation for $\left. r_{t+1} \mvert \sigma^2, \sigma^2_{t+1} \right.$ is similar.
Again we are replacing an arbitrary function with a 2nd-order Taylor expansion.
In the conditionally case, this is without loss of generality.
In general, however, it is not.
There is one key difference, though, since we are trying to identify the price of equity and volatility risk,
i.e.\@ the risk associated with time-variation in the first two motets, ignoring time-variation in the higher
moments is actually what we want to do in the general case.
We do not want to confuse time-varying skewness risk with time-varying vitality risk.

\end{document}

