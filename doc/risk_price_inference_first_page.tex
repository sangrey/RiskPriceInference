\documentclass[11pt, letterpaper, twoside]{article}
\usepackage{risk_price_inference}
\addbibresource{risk_price_inference.bib}

\author{Xu Cheng\thanks{University of Pennsylvania, The Perelman Center for Political Science and Economics, 133 South 36th Street, Philadelphia, PA 19104, \href{mailto:xucheng@upenn.edu}{xucheng@upenn.edu}} \and Eric Renault\thanks{Brown University, Department of Economics -- Box B, 64 Waterman Street, Providence, RI 02912, \href{mailto:eric_renault@brown.edu}{eric\_renault@brown.edu}} \and Paul Sangrey\thanks{University of Pennsylvania, The Perelman Center for Political Science and Economics, 133 South 36th Street, Philadelphia, PA 19104, \href{mailto:paul@sangrey.io}{paul@sangrey.io}}}

\title{Identification Robust Inference for Risk Prices in Structural Stochastic Volatility Models}

\date{\today}

\begin{document}

\begin{titlepage}


\maketitle
\thispagestyle{empty}
\addtocounter{page}{-1}

\begin{abstract} 

\singlespacing \noindent 
In structural stochastic volatility asset pricing models, changes in volatility affect risk premia through two channels: (1) the investor's willingness to bear high volatility in order to get high expected returns as measured by the market risk price, and (2) the investor’s direct aversion to changes in future volatility as measured by the volatility risk price. Disentangling these channels is difficult and poses a subtle identification problem that invalidates standard inference. We adopt the discrete-time exponentially affine model of \textcite{han2018leverage}, which links the identification of volatility risk price to the leverage effect. In particular, we develop a minimum distance criterion that links the market risk price, the volatility risk price, and the leverage effect to the well-behaved reduced-form parameters governing the return and volatility's joint distribution. The link functions are almost flat if the leverage effect is close to zero, making estimating the volatility risk price difficult. We adapt the conditional quasi-likelihood ratio test \textcite{andrews2016conditional} develop in a nonlinear GMM framework to a minimum distance framework. The resulting conditional quasi-likelihood ratio test is uniformly valid. We invert this test to derive robust confidence sets that provide correct coverage for the prices regardless of the leverage effect's magnitude. 

\end{abstract} 

\vspace{\baselineskip}

\jelcodes{C12, C14, C38, C58, G12}

\vspace{\baselineskip}

\keywords{weak identification, robust inference, stochastic volatility, leverage, market risk premium, volatility risk premium, risk price, confidence set, asymptotic size}

\end{titlepage}

\clearpage

\section*{Rest of paper is coming soon.}


\end{document}


