\documentclass[smaller, aspectratio=169]{beamer}

\usepackage{silence}
\WarningFilter{remreset}{The remreset package is obsolete}
\usetheme{Boadilla}
\usecolortheme{rose}
\usepackage{slides_preamble}
\usepackage[autopunct=true, hyperref=true, doi=false, isbn=false, natbib=true,
url=false, eprint=false, style=chicago-authordate]{biblatex} 
\addbibresource{risk_bibliography.bib}

\let\emph\relax 
\DeclareTextFontCommand{\emph}{\color{DarkGreen} \bfseries }

\title[]{Identification Robust Inference for Risk Prices in Structural Stochastic Volatility Models} 
\author[Cheng, Renault, and Sangrey]{Xu Cheng \inst{1} \and Eric Renault \inst{2} \and Paul Sangrey \inst{1}}
\institute[]{\inst{1} University of Pennsylvania \and \inst{2} University of Warwick}
\date[]{March 25, 2019}

\begin{document}

\begin{frame}[plain, noframenumbering]
	\maketitle
\end{frame}

 
\section{Introduction}

\begin{frame}[c]{Introduction}

    \begin{itemize}
        \item How do investors trade off risk and return?  \textcites{sharpe1964capital}.
            \medskip
%
        \item The capital asst pricing model (CAPM) gives a static trade-off.
            \medskip
%
        \item In practice, volatility varies over time.
            \medskip
%
        \item Modern asset pricing models often have an additional channel, \parencites{chang2013market, dewbecker2017price}.
            \smallskip
%
        \begin{itemize}
            \item Investors care directly about changes in volatility.
        \end{itemize}
    \end{itemize}
\end{frame}


\begin{frame}[c]{Volatility Affects Expected Returns Through Two Channels}

    \begin{enumerate}
        \item Investors' willingness to tolerate high-volatility to get high expected returns.
            \begin{itemize}
                \item This measured by the market return risk price.
            \end{itemize}
%
                \item Investors' direct aversion to changes in future volatility.
            \begin{itemize}
                \item This is measured by the volatility risk price.
            \end{itemize}
    \end{enumerate}
            \pause

%
    \begin{itemize}
        \item \Textcite{han2018leverage} disentangle the two channels when the leverage effect is \emph{large}. 
            \pause
        \item However, the leverage effect is \emph{small} and hard to estimate, \parencites{aitsahalia2013leverage, bandi2012timevarying}.
    \end{itemize}

    \pause
    \bigskip

    \begin{block}{}
        The volatility risk price is weakly identified. We develop robust methods that create uniformly valid confidence sets in the presence of any identification strength.
    \end{block}
    
\end{frame}


\begin{frame}[c]{Plan for the Talk}
    \begin{enumerate}
%
        \item The model.
            \bigskip
%
        \item Asymptotic results for the reduced-form and structural parameters.
            \bigskip
%
        \item Algorithm to construct the confidence set.
            \bigskip
%
        \item Simulation results.  
            \bigskip
%
        \item Empirical results using data on the S\&P 500.
%
    \end{enumerate}
\end{frame}


\begin{frame}[c]{Asset Pricing Equation}

    \begin{itemize}
        \item $\displaystyle P_t = \E\left[M_{t,t+1} \exp(-r_f) P_{t+1} \mvert \F_{t} \right]$.
            \bigskip
%
        \item $\displaystyle M_{t,t+1}(\pi, \kappa) = \exp\left(m_0 + m_1 \sigma^2_t - \pi \sigma^2_{t+1} - \kappa r_{t+1}\right)$.
    \end{itemize}

\end{frame}
\end{document}

