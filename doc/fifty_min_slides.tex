\documentclass[smaller, aspectratio=169]{beamer}

\usepackage{silence}
\WarningFilter{remreset}{The remreset package is obsolete}
\usetheme{Boadilla}
\usecolortheme{rose}
\usepackage{slides_preamble}
\usepackage[autopunct=true, hyperref=true, doi=false, isbn=false, natbib=true, 
url=false, eprint=false, style=chicago-authordate]{biblatex} 
\addbibresource{risk_bibliography.bib}

\let\emph\relax 
\DeclareTextFontCommand{\emph}{\color{DarkGreen} \bfseries}

\title[]{Identification Robust Inference for Risk Prices in Structural Stochastic Volatility Models} 
\author[Cheng, Renault, and Sangrey]{Xu Cheng \inst{1} \and Eric Renault \inst{2} \and Paul Sangrey \inst{1}}
\institute[]{\inst{1} University of Pennsylvania \and \inst{2} University of Warwick}
\date[]{March 25, 2019}

\begin{document}

\begin{frame}[plain, noframenumbering]
	\maketitle
\end{frame}

 
\section{Introduction}

\begin{frame}[c]{Introduction}

  \begin{itemize}
    \item How do investors trade off risk and return? \textcites{sharpe1964capital}.
      \medskip
%
    \item The capital asst pricing model (CAPM) gives a static trade-off.
      \medskip
%
    \item In practice, volatility varies over time.
      \medskip
%
    \item Modern asset pricing models often have an additional channel, \parencites{chang2013market, dewbecker2017price}.
      \smallskip
%
    \begin{itemize}
      \item Investors care directly about changes in volatility.
    \end{itemize}
  \end{itemize}
\end{frame}


\begin{frame}[c]{Volatility Affects Expected Returns Through Two Channels}

  \begin{enumerate}
    \item Investors' willingness to tolerate high-volatility to get high expected returns.
      \begin{itemize}
        \item This measured by the market return risk price.
      \end{itemize}
%
        \item Investors' direct aversion to changes in future volatility.
      \begin{itemize}
        \item This is measured by the volatility risk price.
      \end{itemize}
  \end{enumerate}
      \pause

%
  \begin{itemize}
    \item \Textcite{han2018leverage} disentangle the two channels when the leverage effect is \emph{large}. 
      \pause
    \item However, the leverage effect is \emph{small} and hard to estimate, \parencites{aitsahalia2013leverage, bandi2012timevarying}.
  \end{itemize}

  \pause
  \bigskip

  \begin{block}{}
    The volatility risk price is weakly identified. We develop robust methods that create uniformly valid confidence sets in the presence of any identification strength.
  \end{block}
  
\end{frame}


\begin{frame}[c]{Plan for the Talk}
  \begin{enumerate}
%
    \item The model.
      \bigskip
%
    \item Asymptotic results for the reduced-form and structural parameters.
      \bigskip
%
    \item Algorithm to construct the confidence set.
      \bigskip
%
    \item Simulation results. 
      \bigskip
%
    \item Empirical results using data on the S\&P 500.
%
  \end{enumerate}
\end{frame}


\begin{frame}[c]{Asset Pricing Equation}

  \begin{itemize}
    \item $\displaystyle P_t = \E\left[M_{t, t+1} \exp(-r_f) P_{t+1} \mvert \F_{t} \right]$.
      \bigskip
%
    \item $\displaystyle M_{t, t+1}(\pi, \kappa) = \exp\left(m_0 + m_1 \sigma^2_t - \pi \sigma^2_{t+1} - \kappa r_{t+1}\right)$.
  \end{itemize}

\end{frame}


\begin{frame}[c]{Reduced-Form Model}

  The volatility follows a conditional autoregressive gamma process.
%
  \medskip
  \begin{itemize}
    \item $\displaystyle \E\left[\exp(-x \sigma^2_{t+1}) \mvert \F_t\right] = \exp(-A(x) \sigma^2_{t} - B(x)) $.
    \medskip
%
    \item $\displaystyle A(x) \coloneqq \frac{\rho x}{1 + c x}$.
    \medskip
%
    \item $\displaystyle B(x) \coloneqq \delta \log(1 + c x)$.
  \end{itemize}
 
  \bigskip
  The return is conditionally Gaussian.
  \medskip

  \begin{itemize}
    \item $\displaystyle \E\left[\exp(-x r_{t+1} \mvert \F_t, \sigma^2_{t+1} \right] = \exp(-C(x) \sigma^2_{t+1} - D(x) \sigma^2_t - E(x))$.
    \medskip
%
    \item $\displaystyle C(x) \coloneqq \psi x - \frac{1-\phi^2}{2} x^2 $.
    \medskip
%
    \item $\displaystyle D(x) \coloneqq \beta x$.
    \medskip
%
    \item $\displaystyle E(x) \coloneqq \gamma x$. 
%
  \end{itemize}
\end{frame}



\begin{frame}[c]{Identification}

  \begin{minipage}{.49\textwidth}
    \centering
    \emph{The Link Functions}
%
    \begin{enumerate}
      \item $\displaystyle \gamma = B(\pi + C(\kappa -1)) - B(\pi + C(\kappa))$. 
%
      \item $\displaystyle \beta = A(\pi + C(\kappa -1)) - A(\pi + C(\kappa))$. 
%
      \item $\displaystyle \psi = \frac{\phi}{\sqrt{2 c}} - \frac{1-\phi^2}{2} + (1 -\phi^2) \kappa$.
    \end{enumerate}
  \end{minipage}%
%
  \hfill
%
  \begin{minipage}{.49\textwidth}
    \centering
    \emph{Identification Problem}
  \end{minipage}

\end{frame}


\begin{frame}[c]{Estimating the Reduced-Form Parameters}


  \begin{enumerate}
    \item We estimate $\omega_1$ using two-step GMM
%
      \begin{equation}
        \label{omega 1 est}
        \widehat{\omega}_{1} \coloneqq \underset{\omega_{1}\in O_{1}}{\arg \min}\left(T^{-1}\sum_{t = 1}^{T}h_{t}(\omega_{1})\right)^{\prime}\widehat{V}_{1}\left(T^{-1}\sum_{t=1}^{T}h_{t}(\omega_{1})\right), 
      \end{equation}%
      %
      where $\widehat{V}_{1}$ is a consistent estimator of $V_{1} \coloneqq \sum_{m=-\infty}^{\infty}\Cov[h_{t}(\omega_{10}), h_{t+m}(\omega_{10})].$
%
      \bigskip
%
    \item We estimate $\omega_{2}$ using generalized least squares: 
%
      \begin{equation}
        \widehat{\omega}_{2} \coloneqq \left( \sum_{t=1}^{T}x_{t}x_{t}^{\prime}\right)^{-1}\sum_{t=1}^{T}x_{t}y_{t}, \text{ where} 
      %
        x_{t} \coloneqq \sigma _{t+1}^{-1}(1, \sigma _{t}^{2}, \sigma _{t+1}^{2})^{\prime} \text{ and}y_{t}=\sigma _{t+1}^{-1}r_{t+1}. \label{omega 2 est}
      \end{equation}
%
      \bigskip
%
    \item We estimate $\omega_{3}$ by the sample variance estimator
%
      \begin{equation}
        \label{omega 3 est}
        \widehat{\omega}_{3} \coloneqq T^{-1}\sum_{t=1}^{T}\left(y_{t}-\widehat{y}_{t}\right)^{2}, \text{ where} \widehat{y}_{t} = x_{t}^{\prime}\widehat{\omega}_{2}. 
      \end{equation}
      
  \end{enumerate}
      

\end{frame}

\begin{frame}[c]{Assumptions for any $\theta$, $\omega$ close to its true value, and $ 0 < C < \infty$.}

  \begin{block}{Assumption R}
    \label{assump:R}
%    
    \begin{enumerate}
      \item Let $T^{-1}\sum_{t=1}^{T}(h_{t}(\omega_{1})-\E[ h_{t}(\omega_{1}))\rightarrow_{p}0$ and $T^{-1}\sum_{t=1}^{T}(H_{t}(\omega_{1})-\E[H_{t}(\omega_{1})])\rightarrow_{p}0$, and $\E[H_{t}(\omega_{1})]$ is continuous in $\omega_{1}$; all holding uniformly.
%
      \item $T^{-1}\sum_{t=1}^{T}(x_{t}x_{t}^{\prime}-\E[ x_{t}x_{t}^{\prime}{])\rightarrow}_{p}0.$
%
      \item $V^{-1/2} (T^{-1/2}\left(\sum_{t=1}^{T}f_{t}(\omega_{0})-\E[f_{t}(\omega_{0})]\right) \rightarrow_{d} \N(0, \I)$ and $\widehat{V} -V\rightarrow_{p}0.$
%
      \item $C^{-1}\leq \lambda_{\min}(A)\leq \lambda_{\max}(A)\leq C$ for $A = V$, $\E[H_{t}\left(\omega_{1, 0}\right)^{\prime}H_{t}\left(\omega_{1, 0}\right) ]), \E[x_{t}x_{t}^{\prime}], \E[ z_{t}z_{t}^{\prime}]$, where $z_{t} = (1, \sigma_{t}^{2}, \sigma_{t}^{4})^{\prime}.$
%
    \end{enumerate}
  \end{block}
  
  \vfill

  \begin{block}{Assumption S}
%
    \begin{enumerate}
      \item $g(\theta, \omega)$ is partially differentiable in $\omega$, with derivative --- $G(\theta, \omega)$ --- that is Lipschitz in both arguments. 
    %
      \item $C^{-1} \leq \lambda_{\min}(G(\theta, \omega)^{\prime}G(\theta, \omega)) \leq \lambda_{\max}(G(\theta, \omega)^{\prime}G(\theta, \omega)) \leq C$.
    \end{enumerate}

  \end{block}

\end{frame}


\begin{frame}[c]{Some Definitions}

    Define the parameters:
    \medskip
%
    \begin{enumerate}
        \item $\displaystyle H(\omega_{1}) \coloneqq \E[H_{t}(\omega_{1})]$ 
%
        \item $\displaystyle \overline{H}(\omega_{1}) \coloneqq T^{-1}\sum_{t=1}^{T} H_{t}(\omega_{1})$
            %
        \item $\displaystyle \mathcal{B} \coloneqq \diag\lbrace[H(\omega_{10})V_{1}^{-1}H(\omega_{10})]^{-1}H(\omega_{10})V_{1}^{-1}$, $\displaystyle \E[x_{t}x_{t}^{\prime}]^{-1}, 1 \rbrace$,  with sample analog  $\displaystyle \widehat{\mathcal{B}}$.
%
    \end{enumerate}
%
    \bigskip
%
    Define the criterion:
%
    \begin{equation*}
      QLR(\theta_{0}) \coloneqq T \widehat{g} (\theta_{0})^{\prime}\widehat{\Sigma} (\theta_{0}, \theta_{0})^{-1}\widehat{g}(\theta_{0})-\underset{\theta \in \Theta}{\min}T\widehat{g}(\theta)^{\prime}\widehat{\Sigma} (\theta, \theta)^{-1}\widehat{g}(\theta),  
    \end{equation*}
    %
    \quad where $\widehat{\Sigma}\left(\theta_{1}, \theta_{2}\right) = \widehat{G}(\theta_{1})\widehat{\Omega}\widehat{G}(\theta_{2})^{\prime}$ and $\widehat{\Omega}$ is a consistent estimator.
 
    \bigskip
%    
    Define the residual process:
    %
    \begin{equation*}
      \widehat{r}(\theta) \coloneqq \widehat{g}(\theta)-\widehat{\Sigma}(\theta, \theta_{0})\widehat{\Sigma}(\theta_{0}, \theta_{0})^{-1}\widehat{g} (\theta_{0}). 
    \end{equation*}

\end{frame}

\begin{frame}[c]{The Algorithm for Constructing the Confidence Set}
  \begin{enumerate}
    \item Estimate the reduced-form parameter $\widehat{\omega} \coloneqq \left(\widehat{\omega}_{1}, \widehat{\omega}_{2}, \widehat{\omega}_{3}\right)'$ following the estimators defined above.
      \medskip
%
    \item Estimate $\widehat{\omega}$'s asymptotic covariance using $\widehat{\Omega} = \widehat{\mathcal{B}}\widehat{V}\widehat{\mathcal{B}}^{\prime}, $ where $\widehat{V}$ is a HAC estimator of $V$.
      \medskip
  
    \item For $\theta_{0}\in \Theta$, 
%
    \begin{enumerate}
      \item Construct the QLR statistic $QLR(\theta_{0})$ using $g(\theta, \omega), $ $G(\theta, \omega), $ $\widehat{\omega}, $ and $\widehat{\Omega}.$
%
      \item Compute the residual process $\widehat{r}(\theta)$.
  
      \item Given $\widehat{r}(\theta), $ compute its critical value $c_{1-\alpha}(r, \theta_{0})$.
%
    \end{enumerate}
      \medskip
%
  \item Repeat these steps for different $\theta_{0}$. Collect the not rejected $\theta_{0}$ to form a nominal level $1-\alpha $ confidence set: 
%
    \begin{equation}
      CS_{T} = \left\lbrace \theta_{0} \mvert QLR_{T}(\theta_{0})\leq c_{1-\alpha}(r, \theta_{0}. \right\rbrace
    \end{equation}
  \end{enumerate}
\end{frame}


\begin{frame}[c]{The algorithm above constructs a valid confidence set}

\begin{theorem}
  \label{Lemma CS}
  Suppose Assumptions R and S hold. Then 
%
  \begin{equation*} 
    \underset{T\rightarrow \infty}{\lim \inf}\underset{P\in \mathcal{P}}{\inf}\Pr \left( \theta _{0}\in CS_{T}\right) \geq 1-\alpha .
  \end{equation*}
\end{theorem}

\end{frame}

\end{document}

