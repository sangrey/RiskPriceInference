\documentclass[smaller, aspectratio=169]{beamer}

\usepackage{silence}
\WarningFilter{remreset}{The remreset package is obsolete}
\usetheme{Boadilla}
\usecolortheme{rose}
\usepackage{slides_preamble}
\usepackage[autopunct=true, hyperref=true, doi=false, isbn=false, natbib=true,
url=false, eprint=false, style=chicago-authordate]{biblatex} 
\addbibresource{risk_bibliography.bib}

\let\emph\relax 
\DeclareTextFontCommand{\emph}{\color{DarkGreen} \bfseries }

\title[]{Identification Robust Inference for Risk Prices in Structural Stochastic Volatility Models} 
\author[Cheng, Renault, and Sangrey]{Xu Cheng \inst{1} \and Eric Renault \inst{2} \and Paul Sangrey \inst{1}}
\institute[]{\inst{1} University of Pennsylvania \and \inst{2} University of Warwick}
\date[]{March 25, 2019}

\begin{document}

\begin{frame}[plain, noframenumbering]
	\maketitle
\end{frame}

 
\section{Introduction}

\begin{frame}[c]{Introduction}

    \begin{itemize}
        \item How do investors trade off risk and return?  \textcites{sharpe1964capital}.
            \medskip
%
        \item The capital asst pricing model (CAPM) gives a static trade-off.
            \medskip
%
        \item In practice, volatility varies over time.
            \medskip
%
        \item Modern asset pricing models often have an additional channel, \parencites{chang2013market, dewbecker2017price}.
            \smallskip
%
        \begin{itemize}
            \item Investors care directly about changes in volatility.
        \end{itemize}
    \end{itemize}
\end{frame}


\begin{frame}[c]{Volatility Affects Expected Returns Through Two Channels}

    \begin{enumerate}
        \item Investors' willingness to tolerate high-volatility to get high expected returns.
            \begin{itemize}
                \item This measured by the market return risk price.
            \end{itemize}
%
                \item Investors' direct aversion to changes in future volatility.
            \begin{itemize}
                \item This is measured by the volatility risk price.
            \end{itemize}
    \end{enumerate}
            \pause

%
    \begin{itemize}
        \item \Textcite{han2018leverage} disentangle the two channels when the leverage effect is \emph{large}. 
            \pause
        \item However, the leverage effect is \emph{small} and hard to estimate, \parencites{aitsahalia2013leverage, bandi2012timevarying}.
    \end{itemize}

    \pause
    \bigskip

    \begin{block}{}
        The volatility risk price is weakly identified. We develop robust methods that create uniformly valid confidence sets in the presence of any identification strength.
    \end{block}
    
\end{frame}


\begin{frame}[c]{Plan for the Talk}
    \begin{enumerate}
%
        \item The model.
            \bigskip
%
        \item Asymptotic results for the reduced-form and structural parameters.
            \bigskip
%
        \item Algorithm to construct the confidence set.
            \bigskip
%
        \item Simulation results.  
            \bigskip
%
        \item Empirical results using data on the S\&P 500.
%
    \end{enumerate}
\end{frame}


\begin{frame}[c]{Asset Pricing Equation}

    \begin{itemize}
        \item $\displaystyle P_t = \E\left[M_{t,t+1} \exp(-r_f) P_{t+1} \mvert \F_{t} \right]$.
            \bigskip
%
        \item $\displaystyle M_{t,t+1}(\pi, \kappa) = \exp\left(m_0 + m_1 \sigma^2_t - \pi \sigma^2_{t+1} - \kappa r_{t+1}\right)$.
    \end{itemize}

\end{frame}


\begin{frame}[c]{Reduced-Form Model}

    The volatility follows a conditional autoregressive gamma process.
%
    \medskip
    \begin{itemize}
        \item $\displaystyle \E\left[\exp(-x \sigma^2_{t+1}) \mvert \F_t\right] = \exp(-A(x) \sigma^2_{t} - B(x)) $.
        \medskip
%
        \item $\displaystyle A(x) \coloneqq \frac{\rho x}{1 + c x}$.
        \medskip
%
        \item $\displaystyle B(x) \coloneqq \delta \log(1 + c x)$.
    \end{itemize}
 
    \bigskip
    The return is conditionally Gaussian.
    \medskip

    \begin{itemize}
        \item $\displaystyle \E\left[\exp(-x r_{t+1} \mvert \F_t, \sigma^2_{t+1} \right] = \exp(-C(x) \sigma^2_{t+1} - D(x) \sigma^2_t - E(x))$.
        \medskip
%
        \item $\displaystyle  C(x) \coloneqq \psi x - \frac{1-\phi^2}{2} x^2 $.
        \medskip
%
        \item $\displaystyle D(x) \coloneqq \beta x$.
        \medskip
%
        \item $\displaystyle E(x) \coloneqq \gamma x$.  
%
    \end{itemize}
\end{frame}



\begin{frame}[c]{Identification}

    \begin{minipage}{.49\textwidth}
        \centering
        \emph{The Link Functions}
%
        \begin{enumerate}
            \item $\displaystyle \gamma = B(\pi + C(\kappa -1)) - B(\pi + C(\kappa))$. 
%
            \item $\displaystyle \beta = A(\pi + C(\kappa -1)) - A(\pi + C(\kappa))$. 
%
            \item $\displaystyle \psi = \frac{\phi}{\sqrt{2 c}} - \frac{1-\phi^2}{2} + (1 -\phi^2) \kappa$.
        \end{enumerate}
    \end{minipage}%
%
    \hfill
%
    \begin{minipage}{.49\textwidth}
        \centering
        \emph{Identification Problem}
    \end{minipage}

\end{frame}

\begin{frame}[c]{Assumption R}
    \label{assump:R}
    The following conditions hold uniformly over $P\in \mathcal{P}$, for some fixed $0 < C < \infty$.
    
    \begin{enumerate}
        \item $T^{-1}\sum_{t=1}^{T}(h_{t}(\omega_{1})-\E[ h_{t}(\omega _{1}))\rightarrow _{p}0$ and $T^{-1}\sum_{t=1}^{T}(H_{t}(\omega _{1})-\E[H_{t}(\omega _{1})])\rightarrow _{p}0,$ $\E[H_{t}(\omega _{1})]$ is continuous in $\omega _{1},$ all uniformly over the parameter space of $\omega _{1}$.
    %
        \item $T^{-1}\sum_{t=1}^{T}(x_{t}x_{t}^{\prime }-\E[ x_{t}x_{t}^{\prime }{])\rightarrow }_{p}0.$
    %
        \item $V^{-1/2}\{T^{-1/2}(\sum_{t=1}^{T}f_{t}(\omega _{0})-\E[f_{t}(\omega _{0}){]\} \rightarrow }_{d}N(0,I)$ and $\widehat{V} -V\rightarrow _{p}0.$
    %
        \item $C^{-1}\leq \lambda_{\min }(A)\leq \lambda_{\max }(A)\leq C$ for $A=V,\E[H_{t}\left( \omega _{1,0}\right) ^{\prime }H_{t}\left( \omega _{1,0}\right) ]),\E[x_{t}x_{t}^{\prime }],\E[ z_{t}z_{t}^{\prime }],$ where $z_{t}=(1,\sigma _{t}^{2},\sigma _{t}^{4})^{\prime }.$
    %
    \end{enumerate}
\end{frame}

\begin{frame}[c]{The Algorithm for Constructing the Confidence Set}
    \begin{enumerate}
        \item Estimate the reduced-form parameter $\widehat{\omega }=( \widehat{\omega }_{1},\widehat{\omega }_{2},\widehat{\omega }_{3})^{\prime }$ following the estimators defined in \ref{omega 1 est}, \ref{omega 2 est}, and  \ref{omega 3 est}.
            \midskip
%
        \item Obtain a consistent estimator of $\widehat{\omega}$'s asymptotic covariance $\widehat{ \Omega }=\widehat{\mathcal{B}}\widehat{V}\widehat{\mathcal{B}}^{\prime },$ where $\widehat{\mathcal{B}}$ is defined in \ref{Fhat} and $\widehat{V}$ is a HAC estimator of $V.$
            \midskip
    
        \item For $\theta _{0}\in \Theta$,
%
        \begin{enumerate}
            \item Construct the QLR statistic $QLR(\theta _{0})$ in \ref{QLR stat} using $g(\theta ,\omega ),$ $G(\theta ,\omega ),$ $\widehat{\omega },$ and $\widehat{\Omega }.$
%
            \item Compute the residual process $\widehat{r}(\theta )$ in \ref{red process}.
    
           \item Given $\widehat{r}(\theta ),$ compute the critical value $c_{1-\alpha }(r,\theta _{0})$ described above.
%
        \end{enumerate}
            \midskip
%
        \item Repeat these steps for different values of $\theta _{0}$.  Construct a confidence set by collecting the null values that are not rejected, i.e., the nominal level $1-\alpha $ confidence set for $\theta _{0}$ is
%
        \begin{equation}
            CS_{T}=\{ \theta _{0}:QLR_{T}(\theta _{0})\leq c_{1-\alpha }(r,\theta_{0})\}.
        \end{equation}
    \end{enumerate}
\end{frame}

\end{document}

