\documentclass[11pt, letterpaper, twoside, final]{article}
\usepackage{risk_price_inference}
\addbibresource{riskpriceinference.bib}


\title{Characterizing the Structure of the Identification Problem in Continuous-Time}
\author{Paul Sangrey}


\begin{document}

\maketitle

To fully understand the price of volatility risk we need to understand how the physical change of measure
caused  by switching between $\F_t$ and $\F_{\sigma^2,t}$ interacts with the measure change caused from switching
between $\PP$ and $\QQ$.
Moving forward we will rely on a  specification for risks that depends exclusively on the mean and volatility
(i.e.\@ the first two moments.)
If higher moments, such as skewness and kurtosis are also priced factors, as in \textcites{harvey2000conditional,
conrad2012exante, chang2013market},  and we used higher sample moments as well to determine the price of our
risk-factors our resulting estimates would be biased, likely substantially so. 

The question facing us is how to we prevent our model's potential misspecification regarding the risk prices of
higher-order moments from affecting our estimates of the volatility prices.
To answer this question, we turn to what our model specifies the continuous-time process, that our model is a
discretization of.

Imposing consistency with a continuous-time model is reasonable in our context because even though our model is
specified in discrete-time it is for an asset that is is continuously tradable.
A good discrete-time model for a continuously-tradable asset would is consistent with a continuous-time
generalization that rules our no-arbitrage.

Instead of using the log-Laplace transform as we did in discrete-time, we will tackle the change of measure
induced by the change of filtration for the log-price directly.
We will do by inducing some functions $\psi(t)$ and $\phi(t)$ which control the drift and volatility of $p(t)$
under $\F_{\sigma^2,t}$.
To be concrete, we will show that you can represent the log-price with respect to $\F_{\sigma^2,t}$ as follows,
where $\widetilde{M}_{\sigma}(t)$ has mean zero, and predictable quadratic variation  $t$.

\begin{restatable}[Log-Price Given Future Volatility]{theorem}{priceDGPGivenVol}
    \label{defn:logPriceGivenVol}

    Assume that volatility is an It\^{o} semimartingale without any predictable jumps, and the prices can be
    represented as follows.
    Further assume that all of the prices are locally-square integrable, and have the following representation.

    \begin{equation}
        \dif p(t) = \gamma(t) + \beta(t) \sigma^2(t) + \sigma(t) \sqrt{1 - \phi(t)^2} \dif W^p(t) + \phi(t)
        \sigma(t) \dif W^{\sigma}(t) 
    \end{equation}

    Then we can represent the prices progressively enlarged by $\lim_{\Delta \to 0^{+}} \sigma(t + \Delta)$ for
    each $t$ as follows.

    \begin{equation}
        \label{eqn:eqn:price_process_decomp}
        \dif p(t) = \gamma(t) + \beta(t) \sigma^2(t) + \psi(t) \sigma^2(t) + \sigma(t) \sqrt{1 - \phi(t)^2} \dif
        W^p(t) 
    \end{equation}

\end{restatable}

The structure here closely mimics the discrete-time structure in \textcite{khrapov2016affine}.
The two key differences are that it is on continuous-time, obviously, and that it is an exact nonparametric
representation.
Their representation only holds under some parametric assumptions, and so their discussion of identification is
parametric.


The $\psi(t)$ parameter governs how the drift changes once we condition on the volatility in the immediate
future.
To see this more fully consider the following, where we assume $\psi(t)$ and $\beta(t)$ are constant.
Note, $F_{t+1, \sigma^2}$ conditions on the entire path of volatility from $t$ to $t+1$.


\begin{align}
    &\phantom{=} \E_{\PP}\left[r_{t+1} \mvert \F_{t}\right]  - \E_{\PP}\left[\E_{\PP}\left[r_{t+1} \mvert \F_{t+1, \sigma^2} \right]
      \mvert \F_t \right]  \\
      %
      \intertext{Then by the law of iterated expectations.}
      %
    &=  \E_{\PP}\left[\beta(t) \sigma^2_{t+1} - \beta(t) \sigma^2_{t+1} + \psi \sigma^2_{t+1}  \mvert \F_t \right] \\
      \label{eqn:change_in_expected_rtn}
      &= \psi \E_{\PP}\left[\sigma^2_{t+1} \mvert \F_t \right]
\end{align}

The parameter $\phi(t)$, on the other hand, governs the correlation between $p(t)$ and $\sigma^2(t)$.
Consider their predictable quadratic variation of the part of the precises that is orthogonal to $\sigma^2(t)$.

By properties of Gaussian conditioning, we can split the Wiener process driving $p(t)$ into two parts.
One that drives $\sigma^2(t)$ --- $W^{\sigma}t)$ and one that is independent of $\sigma^2(t)$ --- $W^p(t)$.
Consequently, we can write the price process, with respect to $\F_t$ as follows.

\begin{equation}
    \predQV*{p(t), \sigma^2(t)}  
    = \predQV*{\int^t_0 \phi(t) \sigma(t) \dif W^{\sigma}, \int^t_0 \sigma_{\sigma} \dif W^{\sigma}} 
    = \int^t_0 \phi(s) \sigma(s) \sigma_{\sigma}(s) \dif s
\end{equation}

If all of the functions are  constant then the correlation between the price's and the volatility's diffusion part
is $\phi$, i.e.\@  $\mathrm{corr}\,(p(t), {\sigma^2(t)}^D) = \phi$.

\
In this section, we have the following two continuous-time processes.
The goal is to show that by using the first-two conditional moments in our identification strategy, the parameters
we are estimating converge to the true equity risk and volatility risk even in the presence of higher-order
time-variation in the price processes.
Our moment conditions we use in the estimation project the risk prices onto the information set generated by
$\E\left[r_{t+1}\mvert \F_{t-1}\right]$  and $\E\left[\sigma^2_{t-1} \mvert \F_{t-1}\right]$.
What we do here is the equivalent projection in population.
This procedure will give us a series of estimates $\hat{\pi}, \hat{\theta}$, etcetera.
These estimates will have some limiting values, as long as the moment conditions are sufficiently non-collinear.
This section characterizes precisely the role these limiting values play in population.
For example, we show that the risk-prices are the actual risk prices under some conditions, which we make
explicit.

The way by which we do this is by representing the processes in continuous-time.
Since the assets we are characterizing are continuously tradable they must satisfy some continuous-time
no-arbitrage conditions. 
In other words, there exists a continuous-time DGP where the DGP we have been using so far is a discretization of. 
This is useful because we can work with the first two moments of the process effectively without loss of
generality.
The linear-affine structure we have been using before will show up as certain terms being constant functions of
time in what follows.
The cost of doing this is that it requires more involved notation and more sophisticated mathematical tools.


It is not entirely obvious that volatility risk premia is even a well-defined concept if neither the volatility
nor the price process itself jumps.\footnote{In general, the predictable projection of the instantaneous
    volatility $\sigma^2(t-)$ is measurable with respect the filtration generated by the data $\F_t$. In addition,
    if both $\sigma^2(t)$ and $p(t)$ are continuous processes, their filtrations are predictable as well: $\F_t =
    \F_{t-}$. Consequently, all of four o the relevant filtrations coincide. Hence, changing filtrations is
    possible. It is not obvious how volatility risk and market risk can be distinguished if conditioning on
    volatility at time $t$ does not change the information set from $\F^p_{t-}$. The discrete-time
characterization of volatility risk is doing precisely this by using the log-Laplace transform.}
Consequently, we will adapt the following continuous-time data-generating process (DGP)
This is arguably the simplest DGP that allows for us to discuss these issues. 
We follow \textcite{sangrey2018jumps} in modeling the jumps as an integral with respect to a variance-gamma
process.
The constraint that $\sigma^2(t) > 0$, will imply some dependence between $\sigma^2(t), \sigma_{\sigma}(t)$, and
$\gamma_{\sigma}(t)$, which we do not rule out.



\begin{defn}{Continuous-Time DGP}
    \label{defn:cont_time_dgp}

    \begin{align}
        \dif p(t) &= \gamma(t) \dif t + \beta(t) \sigma^2(t-) \dif t + \sigma(t) \dif W(t)  \\
        \dif \sigma^2(s)  &= \mu_{\sigma}(t) + \sigma_{\sigma}(t) \dif W_{\sigma}(t) +
        \frac{\gamma_{\sigma}(t)}{\sqrt{2}} \dif \Lap(t) 
    \end{align}

\end{defn}

The other complication with \cref{defn:cont_time_dgp}, is that we split the drift into two parts $\gamma(t)\dif t$
and $\beta(t) \sigma(t-) \dif t$. 
All of the terms here are finite-variation predictable processes.
It is not immediately obvious how to separate out $\gamma(t)$ and $\beta(t)$.
To get the correct discrete-time interpretation of the integrals of these processes, you can take the predictable
finite-variation (drift) of the ordinal process and project it onto $\sigma^2(t)$ without an intercept.
You can then define $\beta(t)$ to be the coefficient from this regression and $\gamma(t)$ to be the residual.
Intuitively, we are separating out the part of price drift driven by expectations of vitality from other sources
of variation.


We now define the relevant filtrations.
We will continue to use $\F_t$ to be the filtration generated by $p(t)$.
We will use $\F_{t-}$ to be the predictable filtration associated with $\F_t$, i.e.\@ the filtration generated by
the predictable processes adapted to $\F_t$.
In addition, we will define $\F_{\sigma^2, t}$ to be $\F_t$ augmented by the right-limits of the volatility. 
You can view it as $\lim_{\Delta to 0} \F_t \cup \F^{\sigma^2}_{t+\Delta}$, where $\F^{\sigma^2}_t$ is the
filtrated generated by $\sigma^2(t)$.
In other words, we will look infinitesimally far into the future path of $\sigma^2(t)$ and add that information to
our filtration.

This is the continuous-time analogue of adding $\int_t^{t+1} \sigma^2(s) \dif s$ to your $\F_t$ information set.
In both cases you are adding the value of the volatility in the \textquote{next period} to the current information
set.
Another way to view this expansion of filtration is that in $\F_t$, $\int^t_{-\infty} \sigma^2(s) \dif s$ is an
optional process, and we are constructing the minimal filtration that makes $\int^t_{-\infty} \sigma^2(s) \dif s$
predictable.\footnote{The progressive enlargement of the filtrations we are considering is an augmentation of
    $\F_t$ with a series of honest times, since $\F^{\sigma^2}_{t}$ is generated by a left-continuous processes.
    Consequently, all of the adapted processes to $\F_t$ are semimartingales with respect to $\F_{\sigma^2,t}$,
    \parencite[Theorem C]{barlow1978study}.  We will further assume that all of the processes maintain the same
stochastic jump-diffusion structure under both filtrations. The relevant technical conditions under which this
structure is preserved is outside this scope of this paper.}

\subsubsection{Going from Continuous-Time to Discrete-Time}\label{sec:discrete_time_to_cont_time}

The data that we have are daily, not in continuous-time. 
Consequently, we need to consider what the price process in  \cref{eqn:eqn:price_process_decomp} implies about the
daily data.
Thankfully, all of the infinite-activity components are independent by construction.
Consequently, we can time-aggregate \cref{eqn:eqn:price_process_decomp}, and get the following formula for the
discrete-time returns.  
We can then use those returns to characterize the identification problem.
We can define the returns with respect to the following two filtrations.
First, we define $r_{t+1}$ with respect to the $\F_{t}$ filtration.

\begin{equation}
    \label{eqn:discrete_time_rtn}
    r_{t+1} = \int_t^{t+1} \gamma(s) \dif s + \int_{t}^{t+1} \beta(s) \sigma^2(s) \dif s +
    \int_t^{t+1} \sigma(s) \left[\sqrt{1 - \phi(s)^2} \dif W^{p}(s) + \phi(s) \dif W^{\sigma}(s)\right]
\end{equation}

If we knew the entire path of future volatility, we could use the following equation, where we characterize
$r_{t+1}$ with respect to the $\sigma\left(\F_{t}, \F_{\sigma^2_{t+1}}\right)$ filtration.

\begin{equation}
    \label{eqn:discrete_time_cond_rtn}
    r_{t+1} = \int_t^{t+1} \gamma(s) \dif s + \int_{t}^{t+1} [\beta(s) + \psi(s)] \sigma^2(s) \dif s +
    \int_t^{t+1} \sigma(s) \sqrt{1-\phi(s)^2} \dif W^{p}(s)
\end{equation}

In both \cref{eqn:discrete_time_rtn} and \cref{eqn:discrete_time_cond_rtn}, we have the random functions
$\gamma(t), \beta(t), \psi(t), \phi(t)$.
In addition, in we also condition on the entire path of future volatility in \cref{eqn:discrete_time_cond_rtn}
instead of just $\int_t^{t+1} \sigma^2(s) \dif s$.

However, since we are using long time-span asymptotics, and not infill asymptotics, the model we end up estimating
will have fixed parameters instead of continuous-time processes for each of these parameters.

%TODO Added assumption here.
We start by defining the means of $\gamma_0, \beta(t), \psi(t)$, and $\phi(t)$: $\gamma_0, \beta_0, \psi_0, \phi_0$.
We assume that all of the processes are ergodic and so time and cross-sectional means are the same.

%TODO Added assumption here.
By the intermediate value theorem, assuming the predictable functions are continuous, there exists an
$s_i^{\dagger}$ for each of these functions that function evaluated at $s^{\dagger}$ equals the mean over the
interval.
For example $\beta\left(s_i^{\dagger}\right) = \int_0^{\infty} \beta(s) \dif s$.  
We can take mean value expansions for each of these functions.

\begin{alignat}{1}
    \label{eqn:price_decomp}
    \left. r_{t+1} \mvert \F_t \right. 
    %
    &= \gamma_0 + \beta_0 \E_{\PP}\left[\int_t^{t+1} \sigma^2(s) \dif s \mvert \F_t\right] + \int_t^{t+1} \sigma(s)
       \left[\sqrt{1 - \phi_0^2} \dif W^{p}(s) + \phi_0 \dif W^{\sigma}(s)\right] + \eta_{1,t+1} \\
    \left. r_{t+1} \mvert \F_t, \sigma^2_{t+1} \right. 
    %
    &= \gamma_0 + (\beta_0 + \psi_0) \int_t^{t+1} \sigma^2(s) \dif s + \int_t^{t+1} \sigma(s) \sqrt{1-\phi_0^2}
       \dif W^{p}(s) + \eta_{2, t+1} 
\end{alignat}

To show identification of our results, we need to characterize the error terms in the expression above.
In particular, we need to show that they are mean-zero and conditionally mean-independent.
We can compute what the error terms are by adding and subtracting the relevant terms.

\begin{alignat}{1}
    \eta_{1,t+1} 
    &= \int_t^{t+1}(\gamma(s) - \gamma_0) \dif s + \E_{\PP}\left[\int_t^{t+1} (\beta(s) - \beta_0) \sigma^2(s) \dif s
        \mvert \F_t\right] \\
    &+ \int_t^{t+1} \sigma(s) \left[\left(\sqrt{1 - \phi(s)^2} - \sqrt{1 - \phi_0^2}\right) \dif W^{p}(s) +
        \left(\phi(s) - \phi_0\right) \dif W^{\sigma}(s)\right] \nonumber \\
%
    \eta_{2,t+1} 
    &= \int_t^{t+1}(\gamma(s) - \gamma_0) \dif s + \int_t^{t+1} (\beta(s) + \psi(s) - (\beta_0 + \psi_0))
       \sigma^2(s) \dif s \\
    &+ \int_t^{t+1} \sigma(s) \left(\sqrt{1 - \phi(s)^2} - \sqrt{1 - \phi_0^2}\right) \dif W^{p}(s) \nonumber 
\end{alignat}

Clearly, all of the terms are  unconditionally mean zero. 
The non-martingale terms are deviations from their global means, and the martingale terms are martingales.
To get them to be conditionally zero, we need to work a little harder.

Assume that the prices of volatility and market risk and the dependence of volatility on its past are fixed over
time.
That implies that approximate log-Laplace transform studied in \cref{sec:dynamics} is fixed.
Consequently, the $\gamma_0, \beta_0$, etc.\ functions are valid means for each day, e.g.\@ $\E_{\PP}\left[\int_t^{t+1}
\gamma(t) \mvert \F_t \right] = \gamma_0$. 
Note, the length of the day does not entire in here because it equals $1$.

\begin{equation}
    \label{eqn:cond_zero_err_terms}
    \E_{\PP}\left[\eta_{i,t+1} \mvert \F_{t}\right] = 0
\end{equation}

The only question then is if the $\eta_{i,t+1}$ are correlated with the regressors. 
If we were to use $\sigma^2_{t+1}$ directly as a regressors, they would be.
Consequently, we will solve this using the standard method --- instruments.
We have a collection of $Z_{i,t}$ that are $\F_t$ measurable, but correlated with $\sigma^2_{t+1}$.
Since $Z_{i,t+1} \in \F_t$, \cref{eqn:cond_zero_err_terms} implies they satisfy the necessary orthogonality
condition. 
Any good predictor of $\sigma^2_{t+1}$ will satisfy this criterion.
In the empirical example, we will use lags of $\sigma^2_{t+1}$.


\begin{appendices}

\priceDGPGivenVol*

\begin{proof}

    We work directly with the martingale part of the price, $\sigma(t) \dif W(t)$, the drift part can just be
    added back to the price when we are finished because it is predictable.
    
    To model the additional information in a tractable way, consider a dense discrete grid of points over $\R$ ---
    the magnitudes of the prices.
    Define a set of times $\lbrace \tau_i\rbrace $ such that $\tau_i$ is the time at which the volatility crosses
    the $i$\textsuperscript{th} point, but not the $(i+1)$\textsuperscript{th} one. 
    Note, since equality at a discrete dense set of points is sufficient to imply equality over any open set,
    equality over $\lbrace \tau_i \rbrace$ is sufficient for equality in probability since the opens sets generate
    the Borel measure.
    
    Now, each $\tau_i$ is an honest time with respect to the $\F_{t=\tau_i-}$ filtration, but not a stopping time
    because it is not contained within the $\F_{\tau_i-}$ filtration.
    Define $\F_{\sigma,t} \coloneqq \F_{\sigma, t+} \cap \sigma\left(\lbrace \tau_i \rbrace\right)$. 
    Clearly, there is only one $\tau_i$ in that filtration, which we will refer to as $\tau$ in the sequel.
    
    We are augmenting the predictable filtration $\F_{t-}$ with an honest time --- $\tau$.
    Since, $p(t)$ is adapted to $\F_{t-}$, we can use formulas for augmenting the price with an honest time. 
    Also, since there are at most countably many jumps and the prices and volatilities are locally square
    integrable, there is at most one jump in $\sigma^2(t)$ that is measurable with respect to $\F_{\sigma, t+}$.
    
    Since $\sigma^2(t)$ has no predictable jumps, $\tau$ avoid all stopping times of $\F_{t-}$.  
    Hence, the optional and predictable projections coincide, and we can choose the following quantities to be
    continuous.
    Let $A^{\tau}(t)$ be the predictable projection of the process $1\lbrace \tau \leq t \rbrace$, define the
    c\'{a}dl\'{a}g martingale $\mu^{\tau}(t) \coloneqq \E_{\PP}\left[A^{\tau}(\infty) \mvert \F_{t-}
    \right]$.
    Then by \textcite[eqn 2.3]{nikeghbali2007nonstopping}, we can write the martingale part of the process stopped
    at $\tau$  as follows, where $M(t)$ is a martingale.

    
    \begin{align}
        \label{eqn:filtration_changed_martingale}
        \int_0^{t \wedge \tau} \sigma(s)  \dif W(s)   
         &= M(t) + \int_0^{t \wedge \tau} \frac{1}{\Pr\left(\tau > s
            \mvert \F_{s-}\right)} \dif \predQV*{\int_0^s \sigma(u) \dif W(u), \mu^{\tau}(s) } 
%
            \intertext{Since drifts do not affect quadratic variation and $\mu^{\tau}(t)$ is continuous. }
%
        &= M(t \wedge \tau) + \int_0^{t \wedge \tau} \frac{1}{\Pr\left(\tau > s \mvert \F_{s-}\right)} \dif
          \predQV*{\int_0^t \sigma(s) \dif W(s), \sigma_{\mu} \dif W^{\mu}(s)} (s)  \\
    %
        &= M(t \wedge \tau) + \int_0^{t \wedge \tau} \frac{\sigma(s) \sigma_{\mu}(s)}{\Pr\left(\tau > s \mvert
           \F_{s-}\right)} \dif s
    \end{align}
    
    We now let $\psi(t) \coloneqq \frac{\sigma_{\mu}(s) }{\Pr\left(\tau > s \mvert \F_{s-}\right)}$. 
    Then we can write the drift as $\E_{\PP}\left[\dif p(t) \mvert \F_{t-}\right] = \psi(t) \sigma(s)$.

    In addition, $M(t)$ has some predictable quadratic variation that we assume is absolutely continuous.
    We will define $\phi(t)$ so that its diffusion part --- $M^D(t)$ -- has the appropriate relationship with the
    prices.
    Define $\phi(t) \coloneqq \frac{\dif \predQV*{M}(t)}{\sigma_{\sigma}(t)}$. 
    
    
    $M(t)$ is still a martingale in $\mathcal{G}(t)$, which is the smallest filtration satisfying the standard
    conditions that is generated by the prices augmented with $\F_{t-}$. 
    In addition, since $\predQV{M}(t)$ is predictable, and hence $\F_{t-}$ measurable, we can choose its
    instantaneous variance to be same in the two different filtrations. 

    Martingales in a larger filtration are still martingales in a smaller filtration.
    Consequently, $M(t)$ is still a martingale in $\mathcal{G}(t)$.
    However, since only predictable processes are measurable with respect to $\mathcal{G}(t)$.
    Consequently, $M(t)$ is a predictable martingale, implying it has a continuous version.
    
    Hence $\predQV*{M}(t) <= \predQV*{p(t) \perp \sigma^2(t)}$ because conditioning in Gaussian environments does
    not decrease the variance.
    The reverse inequality holds because $p(t) \perp \sigma^2(t)$ is still a martingale in $\F_{\sigma,t}$.
    Consequently, the two processes are equivalent.
   
    Since $p(t) \perp \sigma^2(t) = \int^t \phi(s) \dif W^{p}(s)$, the result follows.


\end{proof}

\end{appendices}

\end{document}
