\documentclass[11pt, letterpaper, twoside, final]{article}
\usepackage{risk_price_inference}
\addbibresource{riskpriceinference.bib}

\author{Xu Cheng\thanks{University of Pennsylvania, The Perelman Center for Political Science and
    Economics, 133 South 36th Street, Philadelphia, PA 19104, \href{mailto:xucheng@upenn.edu}{xucheng@upenn.edu}}
    \and 
    Eric Renault\thanks{Brown University, Department of Economics -- Box B, 64 Waterman Street, Providence, RI
    02912, \href{mailto:eric_renault@brown.edu}{eric\_renault@brown.edu}}
    \and 
    Paul Sangrey\thanks{University of Pennsylvania, The Perelman Center for Political Science and
    Economics, 133 South 36th Street, Philadelphia, PA 19104, \href{mailto:paul@sangrey.io}{paul@sangrey.io}}}
    
\title{Inference for Risk Prices using Equity Data}

\date{\today}

\begin{document}

\begin{titlepage}


    \maketitle
    \thispagestyle{empty}
    \addtocounter{page}{-1}

    \begin{abstract} \singlespacing \noindent 
        How risks and returns are traded off is arguably the central object of study in modern asset pricing. 
        Even though the theoretical literature cohesively argues there should be a strong positive correlation
        between expected returns and volatility, empirically measuring it has proven difficult. 
        In addition, the option pricing literature shows that volatility enters as its own risk factor, not just
        as a predictor of returns.
        However, because as there is a negative contemporaneous correlation between the volatlity and return
        processes and the return and volatlity relationship is inherently nonlineary measuring these risk prices
        have proven delicate. 
        We develop methods to provide  valid inference for the prices of equity and volatility risk using only
        equity data that directly handles this identification problem. 
        We do this by adapting the results in weak identification literature that uses drifting sequences in order
        to be robust to the various paramters' identification strength. 
    \end{abstract}

    \jelcodes{C12, C14, C38,  C58, G12}

    \keywords{identification, robust inference, stochastic volatility, leverage, equity risk premium, volatility
    risk premium, risk price, confidence set, asymptotic size}

\end{titlepage}

\phantomsection
\addcontentsline{toc}{section}{Introduction}

Modern finance is all about the risk return trade offs that investors face and how to optimally respond to them. 
In particular, the central question of asset pricing is what drives expected returns.
Standard economic theory predicts you must compensate investors with higher expected returns when they face more
risk.
In other words, we would expect a positive relationship between the mean and volatility of returns.
In \gentextcites{sharpe1964capital,lintner1965security} capital asset pricing model (CAPM) the expected
return varies proportionally with the volatility of the market return.
However, to price option prices well, we need not only a price of equity risk that loads upon the market variance,
but also a price of volatlity risk as well \parencite{christoffersen2013capturing}.
Investors care not just about how their returns comove with the market return but also how they comove with the
market volatlity.

This implies that current volatlity will affect expected returns in two ways. 1) Through investors' preferences
over future market returns. 2) Through investors' preferences over future volatlity. 
As one might expect distinguishing these two effects is quite difficult. 
The difference between how the two effect risk premia comes through their different nonlinear relationships in the
preserence of contemporaneous correlation between the volatlity and reutrn processes. 
We adopt the framework of \textcite{khrapov2016affine}, which is a discret-time exponentiallyr-affine stochastic
volatity model.
They mention a potential identification strategy in that paper. 
We develop this strategy,  characterizing how the identification strength of the risk prices varies over the
paramter space, and show how to perform uniformly valid inference.

To take a step back and more fully develop empirical structure at hand, we consider how to measure the
relationship between market volatility and expected returns if we assume that this relationship is constant and
linear. 
Even in this simple case, unlike the consensus in the theoretical literature, the empirical literature has found
pinning down this relationship quite difficult.
Not only has its magnitude proven difficult to determine, but various estimates even differ in sign,
\parencite{lettau2010measuring}.

The empirical literature, which we examine in more detail in the literature review, has focused on point estimates
of this magnitude. 
However, if individual investors are ambiguity averse as in \textcite{hansen2001robust, jiu2012ambiguity}, they
will care not just about how the representative investor prices volatility but also their uncertainty regarding
this estimate. 
Furthermore, when economists calibrate models, they need to know how precisely the data determine the parameters
they are calibrating.
If the need to alter the parameter value slightly in order to make their model perform well, are they bringing in
more restrictions or data to more precisely determine the parameter of interest or are they using a value that the
data tell us is incorrect?

Clearly, as obtaining precise believable point estimates of the price of volatility risk has proven quite
difficult, we should expect doing valid inference to be even more delicate.
To the best of our knowledge, this is the first paper to directly tackle this question.
Various authors report confidence intervals as well as their point estimates.
However, they do not take into account the weak identification that makes getting the point estimate difficult in
computing these estimates.

Why is it that measuring this price is so difficult when the theoretical literature is so cohesive?
Econometrically, it is because the volatility price is weakly identified, as in \textcite{andrews2012estimation},
in that the strength of the identification of the price depends upon the value of other parameters. 
This obviously begs the question --- what are these parameters? 

To estimate the risk prices, there are three different phenomena that must be distinguished.
First, the econometrician must disentangle the volatility feedback effect (leverage) which is a contemporaneous
relationship between the volatility and returns from the risk premium, which is a relationship between volatility
and expected returns. 
It is not a contemporaneous relationship, but rather a predictable one. 

Second, and just as important.
We need to seperate the equity risk price and the volatility risk price. 
In general, and we will show this below, the equity risk price is strongly identified even in the presence of the
volatlity risk price. 
There is a simple linear relationship that we can use.
However, volatlity risk being another priced factor will introduce nonlinear nuisance temrs into this regression. 
They will not matter asymptotically, but in finite samples, they likely do.

Returning to identifying the price of volatlity risk, we know that when the strength of identification varies over
the parameter space and we lack identificaiton entirely for some paramter values, the finite sample distributions
are highly nonstandard. 
Consequently, the usual asymptotic approximations do not perform well. 

In addition, the finite-sample distributions are the relevant ones here. 
Even though we often have thousands of observations, since the variation in the expected return is so much smaller
than variation in the return itself, we have a very low signal-to-noise ratio.
Consequently, our estimators will continue to behave as if they were taken from a \textquote{small} sample even in
\textquote{large} samples.

The obvious next step is considering how we need to do this in practice.
Since the contribution of this paper is in terms of methodology and empirical results, we will take a model from
the literature that has the various components, instead of developing our own asset or option pricing model.
In particular, we take the model from \textcite{khrapov2016affine} and use it to estimate the relevant parameters. 

This model has a few nice features. 
First, it has both equity and volatility prices and a leverage effect. 
As such, it is the natural discrete time analogue of the \textcite{heston1993closedform} option pricing model. 
It has an exponentially affine stochastic discount factor  and shares with \textcite{heston1993closedform} the
advantage of having a structure preserving change of measure between the physical and risk-neutral models.
By doing our analysis in discrete-time we are able to more directly compare our results to risk-premia estimates
outside of the option pricing literature and the jumps in high-frequency innovations will not dramatically affect
our results.  
If we were to use a diffusion process in continuous time, we would be severely counterfactually constraining the
higher-order  moments of the process in way that would likely bias our inference. 

As far as estimation is concierend, we derive a series of conditional means and variances.
We then take these means and variances and plug them into a general method of moments (GMM) criterion.
The data we use are the bivariate series $\begin{pmatrix} r_{t+1}, \sigma^2_{t+1} \end{pmatrix}$.
$r_{t+1}$ is the daily return on some asset, and we use its associated realized volatility for $\sigma^2_{t+1}$.
We go into further detail in \cref{sec:data} regarding how we obtained it, the time-span covered, and so on.

\section{Literature Review}\label{sec:lit_review}


\section{The Model}\label{sec:model}

\addtocounter{subsection}{1}

We estimate the prices of some factors using moment conditions derived from a pricing model. 
As is standard in that literature, we will do  this by specifying the physical and risk-neutral measures and their
relationship, i.e.\@ the stochastic discount factor or pricing kernel.
This is non-trivial because we only observe equity data and so can only use moments with respect to the physical
measure to estimate the parameters. 
However, as will be seen in detail below the risk prices govern the stochastic discount factor (SDF), not the
physical measure directly. 
Consequently, we need to relate the physical and risk-neutral measures through SDF closely in order to get
restrictions on the behavior of the physical measure in terms of the risk prices. 

Let $\F_t$ be the representative investor's information set at time $t$, and $P_t$ be the price on the asset in
question, with associated return $r_{t+1}$ and volatility  $\sigma^2_{t+1}$.
Given $\F_{t}$, the vector $\left( r_{t+1},  \sigma^2_{t+1}\right)$ is drawn from some process --- $\PP$ --- the
physical measure. 
We can further define the risk neutral measure --- $\QQ$ ---  as the process that makes $P_t$ a martingale.
The advantage of defining $\QQ$ is that for some function --- $f$ -- of the future return --- $r{t+1}$   -- 
and volatility $\sigma^2_{t+1}$ and potentially the current information available --
$\F_t$, we can price this payoff as its expectation with respect to $\QQ$.
In other  words, the price of $f(r_{t+1}, \sigma^2_{t+1}, \F_t)$ satisfies \cref{eqn:risk_neutral_measure_defn}
for all $t$ and for all $f$.  
This is useful because we can choose $f$ to make our estimation convenient.

\begin{equation}
    P_t(f) = \E_{\QQ}\left[ f\left(r_{t+1}, \sigma^2_{t+1}, \F_{t}\right)  \mvert \F_{t}\right]
    \label{eqn:risk_neutral_measure_defn}
\end{equation}

Since $P_t(f), r_{t+1}$ and $\sigma_t^2$ are observable, if we specify a model for $\F_t$ in terms of observable
(to the econometrician) variables, this provides a moment condition that we can use. 
However, this condition does not identify everything we wish to estimate, in particular it does not identify the
risk prices because $\QQ$ is not observable.

To resolve this we complete the model by defining the stochastic discount factor --- $M_{t, t+1}$ --- as the
Radon-Nikodym derivative between the $\PP$ and $\QQ$ measures. 
No arbitrage guarantees that this will exist, \parencite{harrison1978martingales}.
Since risk prices arise from investors' demand for compensation to hold risk, the risk price show up here 
in the $\QQ$ measure. 
(We collect the parameter of interest into a vector --- $\omega$.)


\begin{defn}{Asset Pricing Moments}
    \begin{equation}
        P_t(f)  = \E_{\QQ} \left[f\left(r_{t+1}, \sigma^2_{t+1} , \F_t\right) \mvert \F_t \right] =
        \E_{\PP}\left[M_{t,t+1}(\omega)f\left(r_{t+1}, \sigma^2_{t+1}, \F_t\right) \mvert \F_t \right] 
    \end{equation}
\end{defn}

We specify the model by parameterizing the $\PP(\omega)$ and $M_{t, t+1}(\omega)$
We then solve for the $\QQ$ more in terms of these moments.
We start by specifying the $\PP$ measure.

Following \textcite{khrapov2016affine}, we assume that the variables are first-order Markov and there is no
Granger causality from return to the volatility and that returns are serially independent given the volatility
path.
In other words, the volatility drives all of the dynamics of the process.
Note, we do allow $\sigma^2_{t+1}$ and $r_{t+1}$ to be contemporaneously correlated, which they are in the data. 

We construct the following conditional Laplace transforms as follows.
This is well-defined because we can define the true Laplace transform as the expectation of a specific function,
and then integrate out all of the unknown variables using their marginal distributions.
Hence, we can represent out model under the physical measure as follows for some functions $a_{\PP}, b_{\PP},
\alpha_{\PP}, \beta_{\PP}$, and $\gamma_{\PP}$ for all $v$ in its domain.


\begin{restatable}[The Limited Information Model under the Physical Measure]{defn}{physicalMeasureModel}
    \label{defn:physical_model}
    \begin{align}
        \E_{\PP} \left[\exp(-v \sigma^2_{t+1}) \mvert \sigma^2_t \right] &= \exp\left( - a_{\PP}(v) 
        \sigma^2_t - b_{\PP}(v) \right) \\
        \E_{\PP}\left[\exp(-v r_{t+1}) \mvert \sigma^2_t,  \sigma^2_{t+1}\right] &= \exp\left(- \alpha_{\PP}(v)
        \sigma^2_{t+1} - \beta_{\PP}(v)\sigma^2_t - \gamma_{\PP}(v) \right) 
    \end{align}
\end{restatable}


The difficult part moving forward is that we are identifying risk prices, which are  parameters of the stochastic
discount factor (SDF), but since we only observe equity data, we only observe the physical measure.
Consequently, we need to solve for the risk-prices as a function of the parameters governing the physical measure
an then invert this mapping.
In other words, we have some structural parameters --- the risk prices --- and we need to relate them to the
reduced form parameters --- the physical measure parameters.
We will do this by parameterizing the physical measure and the stochastic discount factor in terms of the first
few moments.
If higher moments, such as skewness and kurtosis are also priced factors, as in \textcites{harvey2000conditional,
conrad2012exante, chang2013market},  and we used higher sample moments as well to determine the price of our
risk-factors our resulting estimates would be biased, likely substantially so. 

As discussed in the introduction, this identification strategy is rather fragile.
Consequently, we need to take this weak identification into account when we construct the confidence sets.
We will do this by first deriving the joint distributions of the return and volatility in terms of the physical
measure parameters.
This is not trivial because we only modeled the conditional distribution of the $\left. r_{t+1} \mvert \sigma^2_{t},
\sigma^2_{t+1} \right.$  and $\sigma^2_{t+1}$ is not known at time $t$, and so we need to solve for it.


\subsection{Parameterizing the Physical Measure Dynamics}

We now introduce the data generating process for the volatility.
We use a conditional autoregressive gamma process as in \textcite{gourieroux2006autoregressive, khrapov2016affine}
for the volatility.
This implies that we can parameterize the process using $a_{\PP}$ and $b_{\PP}$ which are defined as follows.

\begin{defn}{Volatility Dynamics Functions}
    \label{defn:physical_vol_dynamics}
    \begin{align}
        \label{defn:a_PP}
        a_{\PP}(v) &= \frac{\rho_{\PP} v}{1 + c_{\PP} v} \\
        \label{defn:b_PP}
        b_{\PP}(v) &= \delta \log(1 + c_{\PP} v)
    \end{align}
    
    with 
    \begin{equation}
        \rho \in [0,1), \quad c > 0, \quad \delta > 0
    \end{equation}

\end{defn}

$\rho_{\PP}$ is a persistence parameter.
$c_{\PP}$ is a scaling parameter.
we can see this this clearly from the following forms of the moment conditions.

\begin{remark}[Volatilty Moment Conditions] 
    \label{remark:vol_moment_conditions}
    \begin{align}
        \E_{\PP}\left[\sigma^2_{t+1} \mvert \sigma^2_t \right]  &= \rho_{\PP} \sigma^2_t  + c_{\PP} \delta\\
%
        \Var_{\PP}\left[\sigma^2_{t+1} \mvert \sigma^2_t \right]  &=  2 c_{\PP} \rho_{\PP} \sigma^2_t  + c_{\PP}^2
        \delta \\
%
    \end{align}
\end{remark}

Since these two moment conditions are sufficient to derive the unconditional moments, all of the parameters are
identified as long as they are in the interior of their appropriately specified domains.
Intuitively, we are using linear regression to estimate the slope and intercept parameters.


\subsubsection{Return Dynamics}

We then turn to computing the moments of the return distribution. 
This is more subtle that computing the moments of the volatility dynamics because we have to relate the dynamics
of the returns to that of the volatility and to the stochastic discount factor, which is not observed. 
We use the conditional autoregressive CAR(1) model here, which we take from
\textcite{darolles2006structural,khrapov2016affine}
This model  specifies the conditional Laplace transform of the return as a function of $\left. r_{t+1} \mvert
\sigma^2_{t+1}, \sigma^2_t \right.$

We start by specifying a parametric form for $\alpha_{\PP}(v)$ and $\beta_{\PP}(v)$, and $\gamma_{\PP}(v)$.
We parameterize $\alpha_{\PP}(v)$ in the way that we do because it gives $\phi$ a nice interpretation.
It is a leverage effect parameter.

\begin{defn}{Return Dynamics Functions}
    \label{defn:physical_return_dynamics}
    \begin{align}
        \alpha_{\PP}(v) &\coloneqq \psi v + \frac{1 - \phi^2}{2} v^2 \\
        \beta_{\PP}(v) &\coloneqq \beta v  \\
        \gamma_{\PP}(v) &\coloneqq \gamma v  
    \end{align}
\end{defn}

Since $\alpha_{\PP}$ is quadratic, the return are conditionally Gaussian.
In addition, these functions that parameterize the Laplace transform, but equal zero at zero, and so we do not
need an intercept.
It might seem as first, that the linear for $\beta_{\PP}(v)$ and $\gamma_{\PP}(v)$ is a strong restriction.
This is not actually the case.
Since we assumed that $\sigma^2_{t+1}$ is an integrated volatility hence greater than $\Var_{\PP}\left(r_{t+1} \mvert
\sigma^2_{t+1}, \sigma^2_{t}\right)$ and variances must be positive, the model implies that they must be linear
(\cref{lemma:linearity_of_physical_functions}).


Again, we start by computing the moments of the conditional distributions of $\left. r_{t+1} \mvert
    \sigma^2_{t+1}, \sigma^2_t \right.$.
As we did above, we do this by evaluating the derivative the log-cumulant function at zero.

\begin{align}
    \E_{\PP}\left[r_{t+1} \mvert \sigma^2_t, \sigma^2_{t+1}\right]  &= \psi \sigma^2_{t+1}  + \beta \sigma^2_t +
    \gamma \\
    \Var_{\PP}\left[r_{t+1} \mvert \sigma^2_t, \sigma^2_{t+1}\right]  &= (1 - \phi^2) \sigma^2_{t+1}  
    \label{eqn:rtn_cond_vol}
\end{align}


\subsection{The Stochastic Discount Factor}\label{sec:deriving_sdf_functions}

We now introduce the stochastic discount factor in order to relate the physical moments above to moments in terms
of the risk-prices.
Since the Laplace transform for $r_{t+1}$  is exponentially affine under both $\PP$ and $\QQ$, the measure change
between them is also exponentially affine.
If we let $\pi$ be the price of volatility risk and $\theta$ be the price of equity risk, we can write down the
change of measure as follows.
Let $r_{f,t}$ refer to the risk-free rate.
It deterministically discounts the prices but plays no further role in the analysis.

\begin{defn}{The Stochastic Discount Factor}
    \label{defn:SDF}
    \begin{equation}
        M_{t,t+1}(\pi, \theta) = \exp\left(-r_{f,t}\right) \exp\left(m_{0}(\pi, \theta) + m_1(\pi, \theta)
        \sigma_t^2 - \pi \sigma^2_{t+1} - \theta r_{t+1}\right) 
    \end{equation}
\end{defn}


Before we show how the risk prices $\pi, \theta$ show up in the equations above, we will introduce some parameters
that govern the dynamics of the volatility process.
$M_{t,t+1}$ must integrate to $1$ for all $\sigma^2_t$.
We can view $m_{0}$ and $m_1$ as integration constants.
By plugging in the Laplace transforms for $\sigma^2_{t+1}$ and $r_{t+1}$ we can derive the following.

\begin{restatable}[Uniform Convergence under Strong Identification]{lemma}{sdfConstants}
    Let the SDF be given as in \cref{defn:SDF}, and the model be paramterized as in
    \cref{defn:physical_vol_dynamics} and \cref{defn:physical_return_dynamics}.

    Then the SDF constants follow the following equations.\footnote{This is Equation $3.4$ in
    \textcite[3.4]{khrapov2016affine}.}


    \begin{align}
        \label{eqn:sdf_functions_vs_physical_functions}
        m_0(\theta, \pi)  &= \gamma_{\PP}(\theta) + b_{\PP}(\alpha_{\PP}(\theta) + \pi) \\
        m_1(\theta, \pi)  &= \beta_{\PP}(\theta) + a_{\PP}(\alpha_{\PP}(\theta) + \pi) \nonumber
    \end{align}

\end{restatable}


Since the right-hand side of \cref{eqn:sdf_functions_vs_physical_functions} are entirely in terms of the physical
measure functions, which are in principle observable, we can estimate the SDF functions entirely using equity
data.
In other words, we do not need, at least in principle, option data to estimate the SDF.
In particular, we can determine the risk prices from equity prices if the identified parts of the $\alpha_{\PP},
\beta_{\PP}$ etc.\@ functions are \textquote{sufficiently} invertible.
(I.e.\@ the matrix of their derivatives satisfies the appropriate non-singularity conditions.)

\subsection{The Risk-Neutral Distribution}

The conditions in \cref{eqn:sdf_functions_vs_physical_functions} are not yet usable in estimation because they
relate a series of unknown functions.
To simplify these equations we impose some restrictions that must hold in the risk-neutral model. 
We know that risk-neutral expected returns must equal zero.
Consequently, any expected returns in the physical measure must be caused by the change of measure.

As we did above, we will use conditional Laplace transforms to parameterize the distributions.
In this case, the Laplace functions maintain the same structure for some functions $a_{\QQ}, b_{\QQ},
\alpha_{\QQ}, \beta_{\QQ}$, and $\gamma_{\QQ}$.

\begin{restatable}[The Limited Information Model under the Risk-Neutral Measure]{defn}{riskNeutralMeasureModel}
    \label{defn:risk_neutral_model}
    \begin{align}
        \E_{\QQ} \left[\exp(-u \sigma^2_{t+1}) \mvert \sigma^2_t \right] &= \exp\left( - a_{\QQ}(u) \sigma^2_t -
        b_{\QQ}(u) \right) \\
        \E_{\QQ}\left[\exp(-u r_{t+1}) \mvert \sigma^2_t,  \sigma^2_{t+1}\right] &= \exp\left(- \alpha_{\QQ}(u)
        \sigma^2_{t+1} - \beta_{\QQ}(u)\sigma^2_t - \gamma_{\QQ}(u) \right) 
    \end{align}
\end{restatable}


Since the change of measure preserves the structure, the parametric forms and moments from above also hold in the
risk-neutral measure.
We will use the same notation, but use subscript $\QQ$ to differentiate the two measure. 
For example, $\psi_{\QQ} \coloneqq \alpha_{\QQ}'(0)$ and $\phi_{\QQ} = \sqrt{1 - \abs*{\alpha''(0)}}$.
We start by characterizing $\beta_{\QQ}$ and $\gamma_{\QQ}$.
Before we do that, it is worth noting that as is standard in these models, the measures' absolute continuity
implies $\phi = \phi_{\QQ}$, \parencite[17]{khrapov2016affine}.
Since we have a risk-neutral distribution the only change between $\E_{\QQ}\left[\exp(r_{t+1}) \mvert
\sigma^2_t\right]$ and $\E{\QQ}\left[\exp(r_{t+1}) \mvert \sigma^2_{t+1}, \sigma^2_t\right]$ comes from the
increased in information.

\subsection{Relating the Physical Measure Functions and the Risk Prices}

We characterize $a_{\QQ}$ and $b_{\QQ}$ by deriving the restrictions that risk-neutrality implies upon the
risk-neutral measure functions.
Risk-neutrality means that the expectation of $\exp(r_{t+1}) = 1$.

\begin{align}
    1 &= \E_{\QQ}\left[\E_{\QQ}\left[\exp(r_{t+1}) \mvert \sigma_t^2, \sigma^2_{t+1}\right]  \mvert \sigma^2_t
        \right] \\
%
      \intertext{This is the conditional Laplace transform evaluated at $-1$.}
%
      \label{eqn:risk_neutral_at_neg_1}
      &= \E_{\QQ}\left[ \exp\left(-\alpha_{\QQ}(-1) \sigma^2_{t+1}\right)  \mvert \sigma^2_t \right] 
         \exp\left(-\beta_{\QQ}(-1) \sigma^2_t - \gamma_{\QQ}(-1)\right)   \\
%
      \intertext{The first term is the Laplace transform evaluated at $\alpha(-1)$.}
%
      &= \exp(-a_{\QQ}(\alpha_{\QQ}(-1)) \sigma^2_t - b_{\QQ}(\alpha_{\QQ}(-1))) \exp\left(-\beta_{\QQ}(-1)
         \sigma^2_t - \gamma_{\QQ}(-1)\right)  
\end{align}

Since this equation must hold for all $\sigma_t^2$, we have the following two equations.

\begin{align}
    a_{\QQ}(\alpha_{\QQ}(-1)) = - \beta_{\QQ}(-1) \\
    b_{\QQ}(\alpha_{\QQ}(-1)) = - \gamma_{\QQ}(-1) 
\end{align}


The linearity of $\beta$ and $\gamma$ as shown by \cref{lemma:linearity_of_physical_functions} implies that the
physical and risk-neutral forms of these functions will coincide. 
To see this, you can differentiate \cref{eqn:sdf_functions_vs_physical_functions} with respect to $\theta$, and
then note that the physical and risk-neutral functions must equal at zero.
We can replace the last two functions with their physical counterparts because linearity implies they
coincide. 
We can then substitute in the risk-neutral distributions.

\begin{gather}
    \label{eqn:beta_qq_at_minus_one}
    \phantom{\implies}   - \beta_{\PP}(-1) = a_{\QQ}(\alpha_{\QQ}(-1))\\
%
    \intertext{Filling in Taylor series expansions of $\beta_{\PP}$ and $\alpha_{\QQ}$ and recalling that
    $\phi_{\QQ} = \phi_{\PP}$ implies.}
%
    \implies \beta_{\PP}'(0) = a_{\QQ}\left(-\psi_{\QQ} + \frac{1}{2} (1- \phi^2)\right)  \\
%
    \intertext{Then we can fill in the formula for $a_{\QQ}$ from \cref{defn:a_PP} because the change of measure
    preserves the structure.}
%
    \implies \beta_{\PP}'(0) = \frac{\rho_{\QQ} \left(-\psi_{\QQ} + \frac{1}{2} (1- \phi^2)\right)}{1 + c_{\QQ}
    \left(-\psi_{\QQ} + \frac{1}{2} (1- \phi^2)\right)} 
    \label{eqn:beta_function_QQ}
\end{gather}

Before we derive $\rho_{\QQ}$, $c_{\QQ}$, and $\psi_{\QQ}$ in terms of the physical measure functions, we perform
a similar calculation for $\gamma_{\QQ}(-1)$.

\begin{gather}
    \label{eqn:gamma_qq_at_minus_one}
    \phantom{\implies}   - \gamma_{\PP}(-1) = b_{\QQ}(\alpha_{\QQ}(-1))\\
%
    \intertext{Again taking Taylor series expansions of $\gamma_{\PP}$ and $\alpha_{\QQ}$.}
%
    \implies \gamma_{\PP}'(0) = b_{\QQ}\left(-\psi_{\QQ} + \frac{1}{2} (1- \phi^2)\right)  \\
%
    \intertext{Then we can fill in the formula for $b_{\QQ}$ from \cref{defn:b_PP} because that the measure change
    is structure preserving.}
%
    \implies \gamma_{\PP}'(0) = \delta \log\left(1 + c_{\QQ} \left(-\psi_{\QQ} + \frac{1}{2} (1-
    \phi^2)\right)\right)
    \label{eqn:gamma_function_QQ}
\end{gather}

To simplify \cref{eqn:beta_function} and \cref{eqn:gamma_function} further, we need to solve for $c_{\QQ}$,
$b_{c_\QQ}$, and $\psi_{\QQ}$  in terms of their physical counterparts.
We can take the equations in \textcite[Proposition 5]{khrapov2016affine} and differentiate and derive the
following relationship between $a_{\QQ}, b_{\QQ}$ and $a_{\PP}, b_{\PP}$.

\begin{align}
    a_{\QQ}'(v) &= a_{\PP}'(v + \pi + \alpha_{\PP}(\theta))  \nonumber \\
%
    b_{\QQ}'(v) &= b_{\PP}'(v + \pi + \alpha_{\PP}(\theta))  \\
%
    b_{\QQ}''(v) &= b_{\PP}''(v + \pi + \alpha_{\PP}(\theta))  \nonumber \\
%
    \intertext{Filling in parameterization of the right-hand sides, and evaluating the equations at $0$ gives the
    following.} 
%
    a_{\QQ}'(0) &= \frac{\rho_{\PP}}{1 + c_{\PP}(\pi + \alpha_{\PP}(\theta_2))} \nonumber \\
%
    b_{\QQ}'(0) &= \frac{c_{\PP} \delta_{\PP}}{1 + c_{\PP}(\pi + \alpha_{\PP}(\theta_2))} \\
%
    b_{\QQ}''(0) &= -\frac{c_{\PP}^2 \delta_{\PP}}{(1 + c_{\PP}(\pi + \alpha_{\PP}(\theta_2)))^2} \nonumber \\
%
    \intertext{Then since the left hand sides of the equation above are $\rho_{\QQ}$ and $c_{\QQ} \delta$, we
    have the following expressions.}
%
    \rho_{\QQ} &= \frac{\rho_{\PP}}{1 + c_{\PP}(\pi + \alpha_{\PP}(\theta_2))} \nonumber \\
%
    \label{eqn:c_rho_delta_QQ}
    c_{\QQ} \delta_{\QQ} &= \frac{c_{\PP} \delta_{\PP}}{1 + c_{\PP}(\pi + \alpha_{\PP}(\theta_2))}  \\
%
    -c_{\QQ}^2 \delta_{\QQ} &= -\frac{c_{\PP}^2 \delta_{\PP}}{(1 + c_{\PP}(\pi + \alpha_{\PP}(\theta_2)))^2}
    \nonumber
\end{align}

To simplify notation, define the following rescaling function.

\begin{equation}
    \chi(\pi, \theta) \coloneqq 1 + c_{\PP}(\pi + \alpha_{\PP}(\theta))
\end{equation}

We can use \cref{eqn:c_rho_delta_QQ} and substitute in $\chi(\pi, \theta)$ to get the following.

\begin{align}
    \delta_{\QQ} &= \delta_{\PP} \\
    \rho_{\QQ} &= \frac{\rho_{\PP}}{\chi^2(\pi, \theta)} \\
    c_{\QQ} &= \frac{c_{\PP}}{\chi(\pi, \theta)} 
\end{align}

The one value we have yet to derive is $\psi_{\QQ}$.
Again we use \textcite[Proposition 5]{khrapov2016affine}, to relate $\psi_{\PP}$ and $\psi_{\QQ}$. 
Since the $\alpha$ functions are quadratic, we have the following.

\begin{align}
    \alpha_{\QQ}(v) = \psi_{\QQ} v + \frac{1- \phi^2_{\QQ}}{2} v^2  \\
%
    \alpha_{\PP}(v) = \psi_{\PP} v + \frac{1- \phi^2_{\PP}}{2} v^2  
\end{align}

As in \cref{eqn:risk_neutral_at_neg_1}, $\alpha_{\QQ}(-1) = \alpha_{\PP}(\theta)$, recalling that $\phi_{\PP} =
\phi_{\QQ}$, we can differentiate and get the following.

\begin{align}
    \implies \alpha_{\QQ}'(0) = \alpha_{\PP}'(0) + (1 - \phi^2) \theta
%
    \implies \psi_{\QQ} = \psi_{\PP} + (1 - \phi^2) \theta
\end{align}


\subsubsection{Deriving \texorpdfstring{$\psi$}{psi}}

The goal moving forward is to solve for $\gamma$, $\beta$, and $\psi$ in terms of the risk parameters. 
Once we do this we can use the difference in expected returns between the distribution given $\sigma^2_t$ and the
distribution given $\sigma^2_{t+1}$ and $\sigma^2_t$ to separately identify the two risk prices.
In this section, we focus on $\psi$ because it controls this difference in means.

In general, $\psi$ will have three parts.
First, we have the Jensen's inequality term --- the mean will shift by a value proportional to the variance.
Second, the reduction in variance will shift the price as consumers are risk averse.
Third, since $\sigma^2_{t+1}$ and $r_{t+1}$ are correlated, (the leverage effect) the drift will change directly. 

Intuitively, if volatility and returns are uncorrelated, we cannot disentangle the shift in the mean because the
price of volatility risk from the shift in the mean induced by price of equity risk.
They both show up in the same way in $\psi$.
However, if we have a leverage effect, when we condition on $\sigma^2_{t+1}$ two different components of the
return are moving -- the mean and variance.
Since we now have two moments, we can identify both parameters.

The discussion in the previous paragraph was very high level, and so we now turn to the details.
The goal is to separate out the three components.
We know that $\E_{\PP}\left[M_{t,t+1} \exp(r_{t+1}) \mvert \sigma^2_t \right] = 0$.
We want to relate $\E_{\PP}\left[\exp(r_{t+1}) \mvert \sigma^2_{t+1} \sigma^2_t \right]$  to this expression.
Clearly, there are two different things, we have the SDF in the first expression and the conditioning information
is different.

We can view this as two measure changes.
Since prices here are conditionally log-Gaussian, the measure change  can be paramterized in terms of the
covariance between the log SDF ($m_{t,t+1} \coloneqq \log M_{t,t+1}$) and the return.

If we apply the logarithm to \cref{defn:SDF}, we get he following expression for $m_{t,t+1}$.

\begin{equation}
        m_{t,t+1}(\pi, \theta) = \left(m_{0}(\pi, \theta) + m_1(\pi, \theta) \sigma_t^2 - \pi \sigma^2_{t+1} -
        \theta r_{t+1}\right) 
\end{equation}

All of the terms on the right are constant given $\sigma^2_t, \sigma^2_{t+1}$ except for $\theta r_{t+1}$, and so
they do not affect covariances.

\begin{equation}
    \Cov\left(m_{t,t+1}, r_{t+1} \mvert \sigma^2_{t+1}, \sigma^2_t \right)   
%
    = \Cov\left(-\theta r_{t+1}, r_{t+1} \mvert \sigma^2_{t+1}\right)   
%
    = -\theta \Var\left(r_{t+1} \mvert \sigma^2_{t+1}\right)   
\end{equation}

Once we have plugged in the covariance, we can simply slightly, getting the following expression.

\begin{equation}
    \label{eqn:return_covarinace}
    \Cov\left(m_{t,t+1}, r_{t+1} \mvert \sigma^2_{t+1}, \sigma^2_t \right)   
%
    = -\theta (1 - \phi^2) \sigma^2_{t+1} 
\end{equation}

The expression in \cref{eqn:return_covarinace} is the change in the mean driven by investors' risk aversion, and
is controlled by their equity risk price.
It clearly equals zero if $\theta = 0$.

The second term is the Jensen's effect term.
The mean of a log-Gaussian price variable depends on both the mean and variance of the conditionally Gaussian
return.
It has the standard form, minus one-half the conditional variance $\left(-\frac{1 - \phi^2}{2}
\sigma^2_{t+1}\right)$. 


% This is not necessarily useful because we have not derived $\psi_{\PP}$, and so we have no way to estimate it.
% We resolve this issue by relating $\Var\left(r_{t+1} \mvert \sigma^2_t\right)$ and $\E\left[\sigma^2_{t+1}\right]$.
% In continuous-time this is easy because $r_{t+1}$ is a semimartingale and so we can use the It\^{o} Isometry to
% show that they equal. 
% However, in discrete-time this does not quite hold.
% You cannot ignore drifts.

We resolve this issue by deriving a pseudo-return that is mean-zero with respect to the physical distribution. 
Let $\tilde{r}_{t+1} = r_{t+1} - \frac{1 - \phi^2}{2} \sigma^2_{t+1} - (1 - \phi^2) \theta \sigma^2_{t+1}$.
The second two terms here are the 
One from the physical measure to the risk-neutral measure and the second from the filtration generated by
$\sigma^2_{t+1}$ to the one generated by $\sigma^2_{t+1}$.


\begin{align}
    &\phantom{=} \log \E_{\QQ}\left[\exp\left(r_{t+1}\right) \mvert \sigma^2_{t+1}\right] 
%
    \intertext{We can add and subtract the covariance between the return and the SDF.}
%
    &= \log \E_{\QQ}\left[\exp\left(r_{t+1}\right) \exp\left(-(1-\phi^2)\theta \sigma^2_{t+1}\right) \mvert
       \sigma^2_{t+1}\right] + (1-\phi^2) \theta \sigma^2_{t+1} 
%
    \intertext{Since this term is a covariance, we can use it to form a measure change. Intuitively, we are
    applying the chain rule to the multivariate exponential function and noting that only the cross term
    survives.}
%
    &= \log \E_{\PP}\left[\exp\left(r_{t+1}\right) \mvert \sigma^2_{t+1}\right] + (1-\phi^2) \theta \sigma^2_{t+1} 
%
    \intertext{We can now use the formula for the mean of a conditionally log-Gaussian random variable.}
%
    &= \E_{\PP}\left[r_{t+1} \mvert \sigma^2_{t+1}\right] + \frac{1 - \phi^2}{2} \sigma^2_{t+1} + (1 - \phi^2)
        \theta \sigma^2_{t+1}   \\
\end{align}

We know that  


We can absorb the formula for the log-expectation of a conditional Gaussian variable to convert this expression to
an expression in terms of $\exp(r_{t+1})$. 
The second convexity adjustment is not affected by this because it is measurable with respect to the conditioning
set.

\begin{align}
    &= \log \E_{\PP}\left[\exp\left(r_{t+1}\right) \exp\left((1 - \phi^2) \theta \sigma^2_{t+1}\right) \mvert
       \sigma^2_{t+1} \right]\\
%
    \intertext{Since the second term is the conditional covariance between the SDF and the return, we can use it
    as a change of measure.}
%
    &= \log \E_{\QQ}\left[\exp\left(r_{t+1}\right) \mvert \sigma^2_{t+1} \right]
\end{align}

The last term must equal zero because it is the log-expectation of a change in the price with respect to the
risk-neutral measure.\footnote{I am implicitly assuming that the risk-neutral distribution prices assets in a
    valid manner with respect to this filtration as well. In continuous-time this is true because volatility is
predictable as a continuous-time process.} 

$\E_{\PP}\left[\tilde{r}_{t+1} \mvert \sigma^2_{t+1} \right] = 0$, implies $\E_{\PP}\left[\tilde{r}_{t+1}
\mvert \sigma^2_{t} \right] = 0$.
We need to use this smaller filtration because we want to only condition on discrete-time predictable objects.
Because both conditional mean of the log-return $\tilde{r}_{t+1}$ equals zero, it is the discretization of some
continuous-time martingale. 
Consequently, we can use the It\^{o} Isometry.\footnote{We need to use the argument above to avoid any
    discrete-time drifts. Even though the It\^{0} Isometry holds for general semimartingales in continuous-time,
the aggregation we are doing here does not.} 
In other words, the following holds. 

\begin{equation}
    \label{eqn:return_var_versus_expected_vol_pp}
    \E_{\PP}\left[\Var_{\PP}\left[\widetilde{r}_{t+1} \mvert \sigma^2_t \right]\right]  =
    \E_{\PP}\left[\sigma^2_{t+1} \right]
\end{equation}

There is another way that we can compute $\Var_{\PP}\left[r_{t+1} \mvert \sigma^2_t \right]$.
We can use the law of total variance.

\begin{equation}
    \Var_{\PP}\left[\widetilde{r}_{t+1} \mvert \sigma^2_t\right]  =
    \E_{\PP}\left[\Var_{\PP}\left[\widetilde{r}_{t+1} \mvert \sigma^2_{t+1}\right] \mvert \sigma_t^2 \right] +
    \Var_{\PP}\left[\E_{\PP}\left[\widetilde{r}_{t+1}\mvert \sigma_{t+1}^2\right] \mvert \sigma^2_t\right]
\end{equation}

Since $\sigma^2_{t+1}$ is constant given $\sigma^2_{t+1}, \sigma^2_t$, the conditional variance of
$\widetilde{r}_{t+1}$ is the same as $r_{t+1}$.

Now, we can fill in the values for both of the inside variables on the right-hand side and
\cref{eqn:return_var_versus_expected_vol_pp} on the left-hand side and taking unconditional expectations of all of
the variables.

\begin{equation}
    \E_{\PP}\left[\sigma^2_{t+1} \right]  = \E_{\PP}[ \E_{\PP}\left[(1 - \phi^2) \sigma^2_{t+1} \right]] +
    \E_{\PP}\left[\Var_{\PP}\left[\widetilde{\psi} \sigma^2_{t+1} \mvert \sigma^2_t\right]\right] 
\end{equation}

Solving for $\widetilde{\psi}$ we have the following.

\begin{equation}
    \widetilde{\psi} = \phi \sqrt{\frac{\E_{\PP}\left[\sigma^2_{t+1} \right]}{\E_{\PP}
    [\Var_{\PP}\left(\sigma^2_{t+1} \mvert \sigma^2_t\right)]}}  
\end{equation}

Now we can use the relationship between $r_{t+1}$ and $\tilde{r}_{t+1}$ to relate $\tilde{\psi}$ and $\psi$.

\begin{equation}
    \label{eqn:psi_pp_as_func_of_params}
    \psi_{\PP} = \phi \sqrt{\frac{c_{\PP} \delta / (1 - \rho_{\PP})}{2 c_{\PP} \rho_{\PP} \left(c_{\PP} \delta /
    (1 - \rho_{\PP})\right) + c_{\PP}^2 \delta}} - \frac{1- \phi^2}{2} - (1 - \phi^2) \theta = \frac{\phi}{\sqrt{c
    (1 + \rho_{\PP})}} - \frac{1 - \phi^2}{2}  + (1 - \phi^2) \theta
\end{equation}

Now that we have a formula for $\psi_{\PP}$, $\rho_{\QQ}$, and $c_{\QQ}$ we can substitute them into
\cref{eqn:beta_function_QQ}  and \cref{eqn:gamma_function_QQ} in order to reparameterize $\beta$ and $\gamma$ as
functions of the parameters.
(Recall, $\omega$ is the vector of parameters.)

\begin{align}
    \label{eqn:beta_function}
    \beta(\omega) &\coloneqq \frac{\rho_{\PP} \left(-\psi_{\PP} - (1- \phi^2)(\theta -
    \frac{1}{2})\right)}{\chi^2\left(\pi, \theta\right) + \chi(\pi, \theta)c_{\PP} \left(-\psi_{\PP} - (1-
    \phi^2)(\theta-\frac{1}{2} )\right)}  \\
%
    \label{eqn:gamma_function}
    \gamma(\omega) &\coloneqq \delta \log\left(1 + \frac{c_{\PP}}{\chi(\pi,\theta)} \left(-\psi_{\PP} - (1-
    \phi^2)\left(\theta - \frac{1}{2}\right) \right)\right) \\
%
    \label{eqn:psi_function}
    \psi(\omega) &= \frac{\phi}{\sqrt{c_{\PP} (1 + \rho_{\PP})}} + \frac{1 - \phi^2}{2} - (1 - \phi^2) \theta 
\end{align}


\subsection{Identification of the Risk Prices}

The final goal is to identify the risk-prices $\theta$ and $\pi$.
In the previous sections, we have derived a series of moment conditions in terms of the parameters.
We now need to analyze when these moment conditions allow to identify the risk prices. 
The information that return data contain about equity pricing data is entirely encapsulated by the asset pricing
equation for excess returns.  
In what follows, we will use $rx_{t+1}$ as the excess log-return.
The definition of $M_{t,t+1}$ as a change of measure means that the following holds.

\begin{equation}
    \E_{\PP}\left[ M_{t,t+1}(\theta, \pi) \exp(rx_{t+1}) \mvert \F_{t} \right] = 1
\end{equation}

We now characterize, the information in this set of moment conditions regarding the risk prices.
The difficult part is identifying the volatility risk price --- $\pi$, and so we will focus first on the
information regarding that parameter.
We start by substituting in the SDF formula from above, and we also replace all of the dependence on $\F_t$ with
$\sigma^2_t$.

%Where exactly should I mention that I focusing on pi?

\begin{gather}
    \E \left[ \exps*{ - \pi \sigma^2_{t+1} - (\theta - 1) rx_{t+1} } \mvert \sigma^2_t \right]
        = \exps*{- m_0(\theta, \pi) - m_1(\theta, \pi) \sigma^2_t}
%
    \intertext{Similar to above, we used the law of iterated expectations to substitute in the conditional Laplace
        transforms of $rx_{t+1}$ and $\sigma^2_{t+1}$.}
%
    \E_{\PP}\left[\exps*{- a_{\PP}\left(\pi + \alpha_{\PP}(\theta -1)\right) \sigma^2_t - b_{\PP}(\pi) -
    \beta_{\PP}(\theta-1) \sigma^2_t - \gamma_{\PP}(\theta-1)} \mvert \sigma^2_t \right] = \exps*{- m_0(\theta,
    \pi) - m_1(\theta, \pi) \sigma^2_t} 
\end{gather}

Similar to above, we can now match the coefficients of the $\F_t$-measurable variables in the above moment
condition. 
We also substitute in the formulas for $m_0(\theta, \pi)$ and $m_1(\theta, \pi)$ that we derived in
\cref{sec:deriving_sdf_functions}.

\begin{align}
   \label{eqn:identification_eqn_1}
   \gamma_{\PP}(\theta-1) + b_{\PP}(\pi + \alpha_{\PP}(\theta - 1)  &= \gamma_{\PP}(\theta) + b_{\PP}(\pi +
    \alpha_{\PP}(\theta))  \\
    \label{eqn:identification_eqn_2}
    \beta_{\PP}(\theta-1) + a_{\PP}(\pi + \alpha_{\PP}(\theta -1)) &= \beta_{\PP}(\theta) +
        a_{\PP}(\pi + \alpha_{\PP}(\theta)) 
\end{align}

We can characterize the identification restrictions in \cref{eqn:identification_eqn_1} and
\cref{eqn:identification_eqn_2} in two different cases.\footnote{These equations are Equation 3.7 in
\textcite{khrapov2016affine}.}

\begin{enumerate}
    \item[Case 1:] The price of equity risk $\theta$ satisfies the following equations. 
        \begin{equation}
            \alpha_{\PP}(\theta - 1) = \alpha_{\PP}(\theta)
            \label{eqn:lack_of_id_condition}
        \end{equation}

        If \cref{eqn:lack_of_id_condition} holds, then some simple algebra shows that $\gamma(\theta) =
        \gamma(\theta-1)$ and $\beta(\theta) = \beta(\theta-1)$.
        In this situation, clearly, regardless of the value of $\pi$, we can satisfy these
        \cref{eqn:identification_eqn_1} and \cref{eqn:identification_eqn_2}.
        In other words, the asset pricing equation does not identify $\pi$. 
        As noted by \textcite{khrapov2016affine}, this is in line with the common belief that the econometrician
        needs options data to be able to identify the price of volatility risk. 

    \item[Case 2:] 
        In general, there is no reason to expect the \cref{eqn:lack_of_id_condition} to hold.
        If it does not, it might seem reasonable to expect that we should be able to identify both $\theta$ and
        $\pi$.
        In other words, we should, in principle, at least be able to identify $\pi$ from the difference between the
        functions in the previous case when evaluated at $\theta-1$ and $\theta$.
\end{enumerate}

We now show that if $\phi = 0$, then \cref{eqn:lack_of_id_condition} will hold, but it will fail to hold if $\phi
\neq 0$.
In other words, as mentioned in \textcite[13]{khrapov2016affine}, a leverage effect will allow us to separately
identify $\theta$ and $\pi$.

To see why this is the case note the following.
To see when this will happen consider a 2nd-order Taylor series expansion of $\alpha_{\PP}$, which since returns
are conditionally Gaussian is an exact expansion.

\begin{align}
    \alpha(\theta) - \alpha(\theta - 1) &= \alpha_{\PP}'(0)  + \alpha_{\PP}''(0) \left(\theta - \frac{1}{2}\right)
    \\
%
    \label{eqn:lack_of_id}
    &=  \psi_{\PP} + (1 - \phi^2) \left(\theta - \frac{1}{2}\right)  \\
%
    \intertext{Then filling in the definition of $\psi_{\PP}$ from \cref{eqn:psi_pp_as_func_of_params}.}
%
    &= \frac{\phi}{\sqrt{c_{\PP} (1 + \rho_{\PP} )}} + \frac{1 - \phi^2}{2} - (1 - \phi^2) \theta +  (1 - \phi^2)
       \left(\theta - \frac{1}{2}\right) 
%
    \intertext{Simplifying we have the following.}
%
    \label{eqn:alpha_difference}
    &= \frac{\phi}{\sqrt{c_{\PP} (1 + \rho_{\PP} )}} 
\end{align}

Clearly, if \cref{eqn:alpha_difference} equals zero if and only if $\phi = 0$.
If $\phi \neq 0$, , the derivatives of \cref{eqn:identification_eqn_1} and \cref{eqn:identification_eqn_2}  do not
equal at $\theta$ and $\theta-1$ since $a_{\PP}$ and $b_{\PP}$ are not linear functions.
Consequently, we can separately identify $\pi$ and $\theta$.

\section{GMM}\label{sec:GMM}

To estimate the model we use the various moment conditions that we have derived.
The one that we have yet to derive is the ones implied by the continuous-time model that allow us to estimate
$\phi$ and $\psi_{\PP}$.
Deriving them is straightforward because we constructed what the discrete-time equations, and they are linear.

First, we define two functions that we use to reparameterize the moment conditions.
We do this because the model creates some cross-equation equations to eliminate two of the redundant parameters.
By doing this we are able to avoid the econometric complications that we would have to handle in the general
case.
Note, we are only specifying values for the return conditional on both $\sigma^2_{t+1}$ and $\sigma^2_t$ because
the other moments can be derived from them and the volatility moments.
Also, it is worth noting that we consider \cref{eqn:beta_function}, \cref{eqn:gamma_function}, and
\cref{eqn:psi_function} as implicitly defining those parameters as functions of the other parameters.
\cref{eqn:cond_expected_rtn_moment} can be derived from those equation and the first derivative of the conditional
log-cumulant function equaling the conditional mean.


\begin{defn}{Equilibrium Moment Conditions}
    \label{defn:equilibrium_moment_conditions}
    \begin{align}
        \label{eqn:cond_vol_mean}
        \E_{\PP}\left[\sigma^2_{t+1} \mvert \sigma^2_t \right]  &= \rho \sigma^2_t  + c \delta\\
%
        \label{eqn:global_vol_mean}
        \E_{\PP}\left[\sigma^2_{t+1}\right]  &= \frac{c \delta}{1 - \rho} \\
%
        \label{eqn:cond_vol_var}
        \Var_{\PP}\left[\sigma^2_{t+1} \mvert \sigma^2_t \right]  &=  2 c \rho \sigma^2_t  + c^2 \delta \\
%
        \label{eqn:global_vol_var}
        \Var_{\PP}\left[\sigma^2_{t+1} \right]  &=  \frac{c^2}{1 - \rho^2}  \left(\frac{2 \rho \delta}{1 - \rho}  +
        \delta \right)  \\
%
        \label{eqn:cond_expected_rtn_moment}
        \E_{\PP}\left[r_{t+1} \mvert \sigma^2_t, \sigma^2_{t+1}\right] &= \gamma(\omega) + \beta(\omega) \sigma^2_t +
        \psi_{\PP}(\omega) \sigma^2_{t+1} \\
%
        \label{eqn:cond_rtn_var}
        \Var_{\PP}\left[r_{t+1} \mvert \sigma^2_t, \sigma^2_{t+1}\right] &= (1 - \phi^2) \sigma^2_{t+1} 
\end{align}
\end{defn}

To fully specify the GMM conditions, we need to also specify the instruments we are using.
Doing this correctly is somewhat subtle because we need to only use $\F_{t-1}$-measurable variables in order to
satisfy the necessary exogeneity conditions, while providing instruments for both $\sigma^2_t$ and
$\sigma^2_{t+1}$. 
The natural instrument to use in both cases is $\sigma^2_t$.
This comes from the first-order Markov assumption that we made.
However, we need more than one instrument in order to be identified.
The way around this is to notice that although we assumed first-order Markov, we did not assume that the
transition was linear. 

To see this more fully, note that the volatility risk price is being identified through the part of $\psi_{\PP}$
above and beyond a Jensen inequality effect.
Consequently, we use lags of the volatility and powers of those lags as instruments.
In principle, we would only need to use powers of $\sigma^2_t$ as long as we used enough of them.
However, since in practice we are using an estimate of $\sigma^2_t$, using additional lags instead of very high
powers of $\sigma^2_t$ will likely perform better in practice.


\begin{defn}{Instruments}
    \label{defn:instruments}
    \begin{equation}
        Z_t \coloneqq 1, \sigma^2_{t}, \sigma^2_{t-1}, \sigma^2_{t-2}, \ldots, (\sigma^2_{t})^2,
        (\sigma^2_{t-1})^2, (\sigma^2_{t-2})^2, \ldots
    \end{equation}
\end{defn}

Having constructed the moments and the instruments, we can use GMM to estimate the parameters.
We do need to run a constricted optimization though, because only certain values of the parameters are valid. 

\begin{lemma}{Identified Set}
    
    Assume that the moment conditions specified in \cref{defn:equilibrium_moment_conditions} have the correct form
    and that the instruments we are using satisfy the standard exogeneity and relevant conditions. 

    Let the true parameter vector $\omega \coloneqq (\rho, c, \delta, \phi, \theta, \pi) \in [-1+\epsilon_1,
    1 - \epsilon_2] \times [M_1, M_2] \times [\epsilon_4, M_4]\times [M_5, M_6]\times \times [-1 +
    \epsilon_4, 1 - \epsilon_5] \times [M_7, M_8] \times [M_12, M_13]$, where the $M_{\ast}$ are some large (in
    magnitude) known constants and the $\epsilon_{\ast}$ are some small positive constants.  

    Let $Q_T(\omega, X)$  be the GMM objective function with moment conditions given in
    \cref{defn:equilibrium_moment_conditions} and instruments given in  \cref{defn:instruments}.

    If there exists a $\epsilon$ such that $\abs{\phi} > \epsilon > 0$, then all of the parameters are identified. 
    If $\phi = 0$, then the objective function is independent of $\pi$. 
    Hence, $\pi$ is not identified but all of the other parameters are still identified.

\end{lemma}

\begin{proof}
    Since $Q_{T}$ is a quadratic in terms of deviations between sample and population moment conditions as long as
    the population moment can be inverted to solve for the parameters, the $Q_T$ process identifies them as well. 
    In addition, since we have a sufficient number of exogenous valid instruments the conditioning implied by
    projecting on the instruments does not affect the arguments above. 

    Identifying the four parameters that govern the volatility dynamics is not particularly complicated. 
    We have four parameters and four non-redundant moment conditions.
    The first two equations in    \cref{defn:equilibrium_moment_conditions} identify $\rho$ and $c \delta$.
    \cref{eqn:cond_vol_var} identifies $\rho c$ and $c^2 \delta$. 
    This allows us to separately identify $c$ and $\delta$.
    
    Identifying $\phi$ is also relatively straightforward. Since $r_{t+1}$ and $\sigma^2_{t+1}$ are known, as long
    as we know the conditional mean of $r_{t+1}$, then identifying $\phi$ is identified by \cref{eqn:cond_rtn_var}.
    Identifying the conditional mean of $r_{t+1}$ is straightforward because we observe volatility and the
    conditional mean is a linear equation in these variables. 
    

    Identifying the risk prices $\pi$ and $\theta$ is more complicated.
    We have to identify both parameters off of \cref{eqn:cond_expected_rtn_moment}. 
    This is in principle possible because we now have two non-redundant sources of variation in the data ---
    $\sigma^2_t$ and $\sigma^2_{t+1}$.

    The only place that \cref{defn:equilibrium_moment_conditions} that the risk-prices occurs in
    \cref{eqn:cond_expected_rtn_moment}. 
    We showed that $\gamma, \beta$ and $\psi$ functions are independent of $\pi$ if $\phi = 0$ in the discussion
    leading up \cref{eqn:alpha_difference}.
    We further showed that they are not independent if $\phi \neq 0$.
    In addition the dependence of the identification in terms of the non-singularity of the derivatives
    equilibrium conditions in terms of $\pi$ and $\theta$ depends smoothly on $\phi$. 
    This will be important later.
    
\end{proof}


\section{Weak Identification Setup}

In this section, take the model described in the previous sections and place it in the setup of
\textcite{andrews2014Gmm} so that we ca analyze the effects of possible lack of identification in the model in a
nice clean way.
The goal here is to perform valid inference for $\pi, \theta$ even when $\phi$ might be zero. 


From the discussion above, we can collect the parameters discussed above into a parameter vector of the following
form,i.e.\@ recall the following: $\omega = \lbrace \rho, c, \delta, \phi, \pi, \theta \rbrace$
To write it in the notation of \textcite{andrews2014Gmm}, we partition $\omega$ into three subsets.

\begin{align}
    \phi &\coloneqq \phi  \in (-1, 1) \\ 
    \zeta &\coloneqq \lbrace \rho, c, \delta, \theta \rbrace \in [0,1) \times \R_{++} \times \R_{++} \times
    \R  \\
    \pi &\coloneqq \pi \in \R
\end{align}

Let $\omega$ be the set of possible $\omega$, that as defined above.
It is worth noting that the parameter space has a product form, i.e.\@ the values do not affect the valid values
of the other parameters.

In this environment, $\pi$ is not identified when $\phi = 0$.
Both $\phi$ and $\zeta$ are always identified, and $\zeta$ does not affect the identification of $\pi$.

Let $Q_T(\omega)$ be the GMM criterion function, then the GMM estimator $\hat{\omega}_T$ satisfies the following.


\begin{equation}
    \widehat{\omega}_T \in \omega\ \text{and}\ Q_T(\hat{\omega}_T) = \inf_{\omega \in \Omega} Q_T(\omega) +
    o\left(T^{-1}\right) 
\end{equation}


Now that we have defined the parameters, we can characterize the set of assumptions necessary for valid inference.
We will work through the assumptions described in \textcite{andrews2014Gmm}.
The set of necessary assumptions is relatively complicated because we have to characterize the asymptotic
distribution under several different estimation strengths simultaneously, and the assumptions required to do that
  differ in the various cases. 
In what follows, we will use 

The first assumption specifies the basic identification
problem. It also provides conditions that are used to determine the
probability limit of the GMM estimator, when it exists, under all categories
of drifting sequences of distributions.
Let $\xi$ index the part of the distribution of the data $r_{t+1}, \sigma^2_{t+1}$ that is not determined by the
moment equations.
In general, it is a (likely infinite-dimensional) nuisance parameter that affects the distribution of the data. 


We collect the parameters that we are estimating $\omega$ and the nuisance parameter $\xi$ into one parameter,
$\gamma$ and associated parameter space $\Gamma$.
In the previous discussion we characterized the parameter spaces in a non-compact fashion, let $\omega^{*}$ be a
compact subset of $\omega$, where the true parameter values live.

\begin{defn}{Complete Parameter Space}
    \begin{equation}
        \Gamma \coloneqq \left\lbrace \gamma = (\omega, \xi) \mvert \omega \in \Omega, \xi \in \Xi \right\rbrace 
    \end{equation}
\end{defn}

We characterize these drifting sequences of distributions by sequences of true parameters $\gamma_T \coloneqq
(\omega_T, \phi_T)$.

\purple{TODO Add discussion of the limiting process.}
\purple{Verify that the assumptions on the parameter space hold.}
\purple{Discuss what happens if we lack identification and hence cannot consistently estimate the parameter.}


\begin{restatable}[Inference for $\omega$ under Weak Identification]{theorem}{InferenceWeakID}
    Let that $\phi_0  \in \left(\underline{\phi}_0,1\right)$, for some $\underline{\phi}_0 > -1$. 
    $\rho_0 \in \left[0,1\right)$, and $c_0 > 0$. 

    \purple{TODO  Add Conclusion}
\end{restatable}


\section{Simulations}

\section{Data}\label{sec:data}

We also assume that $\left[ \frac{\psi_{\PP}}{\phi} \right]^2 \approx \frac{\E \left[\sigma^2_{t+1} \mvert
\F_t\right]}{\Var\left[r_{t+1} \mvert \F_t\right]}$, which enables our approximation of $\sigma^2_{t+1}$ by the
realized volatility.

\section{Empirical Results}

\section{Conclusion}

\clearpage

\phantomsection
\addcontentsline{toc}{section}{References}
\printbibliography
\clearpage

\begin{appendices}


\section{Model Characterization}

\begin{lemma}[Linearity of $\beta$ and $\gamma$]
    \label{lemma:linearity_of_physical_functions}
    Letting $\sigma^2_{t+1}$ be the integrated volatility of a process with return $r_{t+1}$.
    Assume that $\sigma^2_{t+1}$ and $r_{t+1}$ follow a bivariate CAR(1) process parametrized as in
    \cref{defn:physical_vol_dynamics} and \cref{defn:physical_return_dynamics}. 
    Then $\beta_{\PP}''(0)$ and $\gamma_{\PP}''(0)$ both equal zero.
\end{lemma}

\begin{proof}
    By the \Ito\ Isometry, and the definition of $r_{t+1}$ as an integrated variance, the following holds for the
    returns' predictable information set $\F^r_{t-}$.  

    \begin{equation}
        \Var_{\PP}\left(r_{tau+1}\mvert \F_{tau-}\right) \leq \E_{\PP}\left(r^2_{tau+1} \mvert \F_{tau-}\right) 
        = \E_{\PP}\left(\sigma^2_{t+1}\mvert \F_{\tau-}\right)
    \end{equation}

    The integrated volatlity is predictable, and so $\sigma^2_{t+1}$ is contained in the return's predictable
    $\sigma$-algebra. 

    Consequently, $\Var_{\PP}\left(r_{t+1} \mvert \sigma^2_t, \sigma^2_{t+1}\right) \leq \sigma^2_{t+1}$
    In addition, variance must always be positive, and so $\Var_{\PP}\left(r_{t+1} \mvert \sigma^2_t,
    \sigma^2_{t+1}\right) \geq 0$.

    Since the second derivative of the log-cumulant funciton evaluted at zero equals the variance, we have the
    following set of inequalities.

    \begin{gather}
        0 \leq (1 - \phi^2) \sigma^2_{t+1} - \beta_{\PP}''(0) \sigma^2_t - \gamma_{\PP}''(0) \leq
        \sigma^2_{t+1} 
%
        \intertext{Dividing through by $\sigma^2_{t+1}$ and pulling the first term outside}
%
        \label{eqn:second_derivative_inequalities}
        \implies \phi^2 - 1 \leq -\frac{1}{\sigma^2_{t+1}} \left(\beta_{\PP}''(0)  \sigma^2_t +
        \gamma_{\PP}''(0)\right) \leq \phi^2 
%
    \end{gather}

    On the outside of the two inequalities we have constants, and the distribution of $\sigma^2_{t+1}$ given
    $\sigma^2_t$ is not bounded away from zero.
    Consequently, the only way for \cref{eqn:second_derivative_inequalities} to hold for all $\sigma^2_{t+1}$ is
    if the term inside the parantheses equals  zero.

    \begin{equation}
        0 = \beta_{\PP}''(0) \sigma^2_t + \gamma_{\PP}''(0)
    \end{equation}

    However, the only way for this to hold is for both $\gamma''(0)$ and $\beta''(0)$ to equal zero.
    This plus the condtional gaussianity of the returns implied by the paramtric model implies that $\gamma_{\PP}$
    and $\beta_{\PP}$  are both linear.

\end{proof}


\sdfConstants*

\begin{proof}

\begin{equation}
    \label{eqn:reweigted_sdf}
    \E_{\PP}\left[\exp\left(m_0(\theta, \pi) + m_1(\theta, \pi) \sigma^2_t - \pi \sigma^2_{t+1} - \theta r_{t+1}
    \right) \mvert \F_t \right] = 1 
\end{equation}

We can use \cref{eqn:reweigted_sdf} to relate $m_0(\theta, \pi)$ and $m_1(\theta, \pi)$ to the physical measure
functions. 

\begin{gather}
    \E_{\PP}\left[\exp\left(m_0(\theta, \pi) + m_1(\theta, \pi) \sigma^2_t - \pi \sigma^2_{t+1} - \theta r_{t+1}
    \right) \mvert \F_t \right] = 1 \\
    \intertext{By the law of iterated expectations.}
    \E_{\PP}\left[ \E\left[\exp\left(m_0(\theta, \pi) + m_1(\theta, \pi) \sigma^2_t - \pi \sigma^2_{t+1}\right)
        \exp\left( - \theta r_{t+1}\right) \mvert \F_t, \sigma^2_{t+1} \right]\right] = 1 \\
    \intertext{The second term is the Laplace transform of $r_{t+1}$.}
    \E_{\PP}\left[\exp\left(m_0(\theta, \pi) + m_1(\theta, \pi) \sigma^2_t - \pi \sigma^2_{t+1} \right)
        \exp(-\alpha_{\PP}(\theta) \sigma^2_{t+1} - \beta_{\PP}(\theta) \sigma^2_{t} - \gamma_{\PP}(\theta_2)
        \mvert \F_t \right] = 1 \\
    \intertext{Reorganizing terms.}
    \E_{\PP}\left[\exp\left(m_0(\theta, \pi) + m_1(\theta, \pi) \sigma^2_t - \beta_{\PP}(\theta) \sigma^2_{t} -
        \gamma_{\PP}(\theta_2) \right) \exp(-\left(\pi + \alpha_{\PP}(\theta)\right) \sigma^2_{t+1}) \mvert \F_t
        \right] = 1 \\ 
    \intertext{Substituting in the Laplace transform for $\sigma^2_{t+1}$.} 
    \label{eqn:expected_sdf_wrt_PP}
    \E_{\PP}\left[\exp(m_0(\theta, \pi) + m_1(\theta, \pi) \sigma^2_t - \beta_{\PP}(\theta) \sigma^2_{t} -
        \gamma_{\PP}(\theta_2)  - a_{\PP}(\pi + \alpha_{\PP}(\theta)) - b_{\PP}(\pi + \alpha_{\PP}(\theta))
        \mvert \F_t \right] = 1 
\end{gather}

\end{proof}


\section{Identification Proofs}

\section{Inference Assumptions}

    In what follows, three sets of drifting sequences $\lbrace \gamma_T \rbrace$ are key. 
    
    \begin{defn}{Drifting Sequence Parameter Spaces}
        \begin{align}
            \Gamma\left(\gamma_0\right) &\coloneqq \left\lbrace \left\lbrace \gamma_T \in \Gamma \right\rbrace
            \mvert \gamma_T \to \gamma_0 \in \Gamma \right\rbrace\\ 
            \Gamma(\gamma_0, 0, b) &\coloneqq \left\lbrace \lbrace \gamma_T \rbrace \in \Gamma(\gamma_0) \mvert
            \phi_0 = 0\ \text{and}\ \sqrt{T} \phi_T \to b \in (\R \cup \lbrace \pm \infty) \right\rbrace \\
            \Gamma(\gamma_0, \infty, b_0) &\coloneqq \left\lbrace \lbrace \gamma_T \rbrace \in \Gamma (\gamma_0)
            \mvert \sqrt{T} \norm{\phi_T} \to \infty\ \text{and}\ \frac{\phi_T}{\norm{\phi_T}} \to b_0
            \right\rbrace 
        \end{align}
    \end{defn}
    
    These are the standard GMM regularity conditions appropriately adjusted for the lack of identification when
    $\phi =0$.
    
    \begin{assump}[GMM 1]\label{ass:GMM1}
    \begin{assumplist}
        \item If $\phi_0=0$, $\sampmom(\omega)$ and $\W_{T}(\omega)$ do not depend on $\pi$ for all $\omega \in \Omega$,
            for all $T \geq 1$, and for all $\gamma^{*}\in \Gamma.$ 
            \label{ass:GMM1a}
        \item If $\lbrace \gamma_{T} \rbrace \in \Gamma\left(\gamma_0\right)$, $\sup_{\omega \in \Omega}
            \norm*{\sampmom(\omega) - \E\left[g\left(\omega \mvert \gamma_0\right)\right]} \pto 0$ and $\sup_{\omega
            \in \omega} \norm{\W_{T}(\omega)-\E\left[\W\left(\omega \mvert \gamma_0\right)\right]} \pto 0$.
            \label{ass:GMM1b}
        \item When $\phi_0 = 0$,  $g_0\left(\phi, \zeta ,\pi \mvert \gamma_0\right) = 0$ if and only if $\phi
            =\phi_0$ and $\zeta = \zeta_0$ for all $\pi \in \Pi$ and for all $\gamma_0 \in \Gamma.$
            \label{ass:GMM1c}
        \item When $\phi_0 \neq 0$, $g_0\left(\omega \mvert \gamma_0\right)=0$ if and only if $\omega =\omega_0$ for all
            $\gamma_0 \in \Gamma.$
            \label{ass:GMM1d}
        \item  $g_0\left(\omega \mvert \gamma_0\right)$ is continuously differentiable in $\omega $ on $\omega$ with
            partial derivatives with respect to $\omega$ and $\xi$ denoted by $g_{\omega}\left(\theta \mvert
            \gamma_0\right) \in R^{k\times d_{\omega }}$ and $g_{\xi }\left(\omega \mvert \gamma_0\right)\in R^{k\times
            d_{\xi }}$, respectively.
            \label{ass:GMM1e}
        \item $\W\left(\omega \mvert \gamma_0\right)$ is continuous in $\omega$ on $\omega$ for all $\gamma_0\in
            \Gamma$.  \label{ass:GMM1f}
        \item $0 < \lambda_{\min}(\W\left(\xi_0, \pi \mvert \gamma_0\right))\leq \lambda_{\max }(\W\left(\xi_0,\pi
            \mvert \gamma_0\right)) < \infty$, $\forall \pi \in \Pi$, for all $\gamma_0 \in \Gamma$.
            \label{ass:GMM1g}
        \item $\lambda_{\min} (g_{\xi}\left(\xi_0,\pi \mvert \gamma_0\right)^{\prime} \W\left(\xi_0,\pi \mvert
            \gamma_0\right)g_{\xi }\left(\xi_0,\pi \mvert \gamma_0\right))>0$, for all $\pi \in \Pi$,  and for all 
            $\gamma_0 \in \Gamma$ with $\phi_0=0.$
            \label{ass:GMM1h}
        \item$\Xi(\pi)$ is compact for all $\pi \in \Pi$, and both $\Pi$ and $\omega$ are compact.
            \label{ass:GMM1i}
        \item For all $\epsilon > 0$, there exits a $\delta > 0$ such that $d_{H}\left(\Xi \left(\pi_{1}\right),
            \Xi \left( \pi_{2}\right) \right) < \epsilon$ for $\pi_{1}, \pi_{2} \in \Pi$ with
            $\norm*{\pi_{1}-\pi_{2}} < \delta$, where $d_{H}\left( \cdot \right)$ is the Hausdorff metric.
            \label{ass:GMM1j}
    \end{assumplist}
    \end{assump}
    
    
    
    \begin{assump}[GMM 2*]\label{ass:GMM2}
    \begin{assumplist}
        \item $\sampmom(\omega)$ is continuously differentiable in $\omega$ for all $T \geq 1$. 
            \label{ass:GMM2a}
        \item If $\{\gamma_T\} \in \Gamma\left(\gamma_0, 0, b\right)$, $\sup_{\left\lbrace \omega \in \Omega \mvert
            \norm*{(\phi, \zeta')' - (\phi_T, \zeta_0')} \leq \delta_T \right\rbrace}
            \norm*{\frac{\partial}{\partial (\phi, \zeta')'} \sampmom(\omega) - \E\left[\popmom_{(\phi,
            \zeta')'}(\omega) \mvert \gamma_0\right]} = o_p(1)$ for all deterministic sequences  $\delta_T \to 0$.
            \label{ass:GMM2b}
        \item  Let $\omega_T \coloneqq \left\lbrace \omega \in \Omega \mvert \norm*{(\phi, \zeta_) - (\phi_T, \zeta_T)}
            \leq \delta_T \norm*{\beta_T}\, \text{and}\, \norm*{\pi - \pi_T} \leq \delta_T \right\rbrace$.  Let
            $\delta_T$ be a deterministic sequence that converges to zero.  If $\{\gamma_T \} \in
            \Gamma\left(\gamma_0, \infty, b_0\right)$, then we have the following asymptotic behavior.
            $\sup_{\omega \in \Omega_T} \norm*{\left(\frac{\partial}{\partial \omega'} \overline{g}_T -
            \E\left[g_{\omega}(\omega) \mvert \gamma_0\right]\right) \diag\left(1_{1+d_\zeta}',
            (1/\phi_T)_{d_{\pi}}'\right)}  = o_p(1)$. 
            \label{ass:GMM2c}
    \end{assumplist}
    \end{assump}
    
    Once we have \nameref{ass:GMM1} and \nameref{ass:GMM2}, we use \nameref{ass:GMM3} to derive the asymptotic
    distribution under weak and semi-strong identification.
    These conditions will be characterized using the expected derivative of the population moment conditions. 
    
    \begin{defn}
        \label{defn:moment_derivative_func}
        \begin{equation}
            K_{T,g}\left(\omega \mvert \gamma^{*}\right) \coloneqq  \frac{1}{T} \sum_{i=1}^T \frac{\partial}{\partial
            \phi^{*}} \E \left[ \popmom(W_T, \omega) \mvert \gamma^{*} \right]
        \end{equation}
    \end{defn}
    
    
    \begin{assump}[GMM 3]\label{ass:GMM3}
    \begin{assumplist}
        \item $\sampmom(\omega) = \frac{1}{T} \sum_{i=1}^T \popmom(W_T, \omega)$  for some function $\popmom(W_T,
            \omega) : \R^{k \times k} \times \omega \to \R^k$.
            \label{ass:GMM3a}
        \item $\E\left[\popmom(W_T, \beta_0, \zeta^{*}, \pi) \mvert \gamma^{*} \right] = 0$ for all $\pi \in \Pi$ and
            for all $i \geq 1$ if $\gamma^{*} = \left(0,\zeta^{*}, \pi^{*}, \xi^{*} \right) \in \Gamma$.
            \label{ass:GMM3b}
        \item If $\{ \gamma_T \} \in \Gamma(\gamma_0, 0, b)$, $\frac{1}{\sqrt{T}} \sum_{i=1}^T \left(g(W_T,
            \zeta_{0,T}, \pi_T) - \E \left[g(W_T, \zeta_{0,T}, \pi_T)\mvert \gamma_T \right]\right)  \dto \N\left(0,
            \aleph(\gamma_0)\right)$, where $\aleph(\gamma_0)$ is a $k \times k$ matrix.
            \label{ass:GMM3c}
        \item 
            \label{ass:GMM3d}
            \begin{enumerate}
                \item  $K_{T,g}\left(\omega \mvert \gamma^{*}\right)$ exists for all $\{\omega, \gamma^{*} \} \in
                    \left(\omega_{\delta} \times \Gamma_{0}\right)$ and for all $T \geq 1$.
                \item $K_{T,g}\left(\phi_T, \zeta_T, \pi \mvert \widetilde{\gamma}_T\right)$ uniformly converges
                    to some non-stochastic matrix-valued function  $K_{g}\left(0, \zeta_0, \pi \mvert
                    \gamma_0\right)$ over $\pi \in \Pi$ for all deterministic sequences $\{\phi_T, \zeta_T,
                    \widetilde{\gamma}_T \}$ satisfying $\widetilde{\gamma}_T \in \Gamma$, $\widetilde{\gamma}_T
                    \to \gamma_0 \coloneqq (0, \zeta_0, \pi_0, \xi)$, $\{\phi_T, \zeta_T, \pi \} \in \omega$ and
                    $\{\phi_T, \zeta_T \} \to (0, \zeta_0)$.
                \item $K_g\left(\phi_0, \zeta_0, \pi \mvert \gamma_0\right)$ is continuous on $\Pi$ for all
                    $\gamma_0 \in \Gamma$ with $\phi_0 = 0$.
            \end{enumerate}
            \item $K\left(\phi_0, \zeta_0, \pi \mvert \gamma_0\right) = \popmom_{\phi, \zeta}\left(\phi_0,
                \pi\mvert \gamma_0\right) x$ for some $x \in \R^{1+d_{\zeta}}$ if and only $\pi =
                \pi_0$.\footnote{Since $\dim(\phi) = 1$, we can assume without loss of generality that the
                $\omega_0$ from \textcite{andrews2014Gmm} equals $1$.}
                \label{ass:GMM3e}
            \item If $\{ \gamma_T \} \in \Gamma(\gamma_0, 0, b)$, $\frac{1}{T} \sum_{i=1}^T
                \frac{\partial}{\partial \omega}  \E\left[ \popmom\left(W_T, \omega_T \right ) \mvert \gamma_T \right]
                \to \popmom_{\omega}\left(\omega_0 \mvert \gamma_0\right)$.
            \label{ass:GMM3f}
    \end{assumplist}
    \end{assump}

    \begin{defn}{\popmom*}
        \begin{equation}
            g_{\phi, \zeta}^{*}\left(\phi_0, \zeta_0, \pi_1, \pi_2 \mvert \gamma_0\right)  =
            \left[g_{\phi}\left(\phi_0, \zeta_0, \pi_1 \mvert \gamma_0\right)  , g_{\phi}\left(\phi_0, \zeta_0,
            \pi_2 \mvert \gamma_0\right) , g_{\zeta} \left(\phi_0, \zeta_0 \mvert \gamma_0\right)  \right]  \in
            \R^{k \times (d_{\zeta} + 2)}
        \end{equation}
    \end{defn}


    \begin{assump}[GMM 4]\label{ass:GMM4}
    \begin{assumplist}
        \item $\phi$ is a scalar.
            \label{ass:GMM4a}
        \item $g_{\phi, \zeta}^{*}\left(\phi_0, \zeta_0, \pi_1, \pi_2 \mvert \gamma_0\right)$ has full column
            rank. 
            \label{ass:GMM4b}
        \item $\aleph(\gamma_0)$ is positive definite for all $\gamma_0 \in \gamma $ with $\phi_0 = 0$. 
            \label{ass:GMM4c}
    \end{assumplist}
    \end{assump}

\section{Inference Proofs}


\begin{restatable}[Uniform Convergence under Strong Identification]{lemma}{UllnStrongID}
    \label{lemma:UniformConvergenceStrongID}
    Let $\omega$ be the identified set.
    Further assume that $\phi_0 \neq 0$. 
    Let $\sampmom$ be the sample moment condition defined above, and $\W_T$ be the associated optimal weight matrix
    estimator.
    Then we have the following convergence.

    \begin{align}
        &\sup_{\omega \in \Omega} \norm*{\sampmom(\omega) - \E\left[g\left(\omega \mvert
          \gamma_0\right)\right]}_{Fro} \pto 0 \\ 
        &\sup_{\omega \in \Omega} \norm{\W_{T}(\omega)-\E\left[\W\left(\omega \mvert \gamma_0\right)\right]} \pto 
    \end{align}

\end{restatable}

\begin{proof}

    In this proof we rely heavily on the continuity of the moment conditions over their domain. 
    This can be seen from simple inspection since we assumed that $\phi_0 \geq \underline{\phi} -1$.
    Furthermore since $\omega$ is compact, this continuity implies uniform continuity.
    
    For any positive definite weight-matrix by \textcite[Lemma 2.3]{newey1994large} our criterion function has
    a unique optimum.
    The data, $\sigma^2_{t+1}, r_{t+1}$, are ergodic and stationary.
    Since the moment conditions are not redundant the optimal (GMM) weight matrix $\W$ is positive definite. 
    In addition, $\popmom$ is continuous at each $\omega$, given the restrictions above and properties of
    characteristic functions imply that $\popmom$ is uniformly bounded. 
    For convenience, we assume that the space of $\omega$ is compact.
    This should not be an issue here because the parameters  are either a priori bounded, such as $\phi$ or we
    have substantial a priori knowledge on their plausible magnitudes.
    Hence, \textcite[Theroem 2.6]{newey1994large} implies our estimator is consistent.
    
    However, when we allow for weak identification late on, we need this convergence to be uniform. 
    One straightforward way to show this is to show that our criterion function is globally Lipschitz in a set of
    high probability. 
    
    The other issue is that we need the weight matrix to converge uniformly to its expectation.
    Since the moments are continuous functions over their domain as is the square function.
    This convergence is uniform if and only if the matrix inverse is continuous.
    
    Since we have a finite number of non-redundant moments, the minimum eigenvalue, \\
    $\lambda_{min}\left(\W\left(\phi_0, \zeta_0, \pi \mvert \gamma_0\right)\right) > 0$, and so the matrix inverse
    is uniformly continuous in $\gamma_0$ with respect to the Frobenius norm, which is the sum of the eigenvalues.
    (Recall, that the eigenvalues of the inverse are the inverse of the eigenvalues.)


\end{proof}


\begin{assump}[Weak Dependence]
    \label{assumption:weak_dependence}
    $z_t \coloneqq \begin{pmatrix} r_{t+1} \\ \sigma^2_{t+1} \end{pmatrix}$ are $\alpha$-mixing with $\alpha_t =
       O\left(T^{-5}\right)$
\end{assump}


\begin{theorem}[Inference for $\omega$ under Strong Identification]
    Assume that $\phi_0  \in (-1,1) \setminus 0$, $\rho_0 \in [0,1)$, and $c_0 > 0$. 
    Further assume that the data are ergodic, stationary, and satisfy \cref{assumption:weak_dependence}.
    Then the following convergence in distribution holds.

    \begin{equation}
    \sqrt{T} (\widehat{\omega}_T - \omega_0) \dto \N\left(0, \left(G' \E[\W] G\right)^{-1}\right)
    \end{equation}
\end{theorem}

\begin{proof}

    By the above arguments, we have a consistent estimator for $\omega$ and the optimal weight matrix $\W \coloneqq
    (\E\left[g g'\right])^{-1}$, and we will assume that the true value $\omega_{0}$ is in the interior of its
    sample space $\omega$.\footnote{Throughout we will use subscript \num{0}  to denote true values for parameters.}
    Let $G \coloneqq \E\left[\frac{\partial}{\partial \omega} \popmom \right]$ Clearly, $g$ is continuously
    differentiable, and its derivative $G$ is continuous.
    In addition, by the identification discussion $G' W \nabla G$ is nonsingular.
    
    %TODO Replace this with strong (alpha) mixing of order (1+\epsilon). 
    %Don't you need a lot more than this...?

    Since, $\norm*{g_t}$ is almost surely bounded by $1$ it has all of its moments and $z_t$ being $\alpha$-mixing
    implies $g_t$ is as well by the central limit theorem for strongly mixing process $\sqrt{T} \sampmom(\omega^{*})
    \dto \N\left(0, \E\left[\W\right]^{-1}\right)$ as required. 
    Consequently, by \textcite[theorem 3.2]{newey1994large} we have convergence in distribution as well as
    convergence in probability.
    

\end{proof}


\InferenceWeakID*

\begin{proof}
We prove this result by showing that Assumptions GMM 1-4 are satisfied.

\begin{proofpart}
    \label{part:main_theorem_proof_part1}
    In this part, we show that \nameref{ass:GMM1} is satisfied. 
    To do this, we break \nameref{ass:GMM1}  down into three subsections.
    Assumptions \namedref{ass:GMM1a}, \namedref{ass:GMM1b}, \namedref{ass:GMM1c}, and \namedref{ass:GMM1d} state
    that when $\phi = 0$, the moment conditions contain no information regarding $\pi$, but when $\phi \neq 0$,
    the model is identified.
    
    \purple{We need to show the above statement}

    We further showed the relevant uniform convergence to verity \namedref{ass:GMM1b} in
    \cref{lemma:UniformConvergenceStrongID}.
    
    The next two assumptions (\namedref{ass:GMM1e} and \namedref{ass:GMM1f}) are  technical conditions regarding
    the behavior of the moment conditions and weight matrix. 
    Since our moment conditions are derived from an infinitely-differentiable  characteristic function and the
    weight matrix is the optimal one, they both hold trivially.
    
    The third subsection of Assumption \nameref{ass:GMM1} concerns the weight matrix.
    Since we are using the inverse covariance matrix of valid non-redundant model, assumptions
    \namedref{ass:GMM1g} and \namedref{ass:GMM1h} automatically hold.
    
    The last two assumptions, \namedref{ass:GMM1i} and \namedref{ass:GMM1j} require that the parameter spaces do
    not vary too much with the parameters and are compact.
    Since $\omega$ is compact, \namedref{ass:GMM1i} holds trivially, and since it has  has a product form,
    \namedref{ass:GMM1j}  holds trivially as well.
    
\end{proofpart}


\begin{proofpart}
    \label{part:mainTheoremProofPart2}

    In this section, we show that the derivatives of the moment conditions have the correct behavior locally to
    the true parameters.
    We have to do this for the different classes of drifting sequences.
    We will do this by verifitying \cref{ass:GMM2}.
    This is valid since \textcite{andrews2014Gmm} show that this is a sufficient condition for their Assumption
    GMM2, which is what we actually need. 

    Our moment conditions are sample averages of the characteristic function, they satisfy \namedref{ass:GMM2a}
    automatically. 
    Since characteristic functions are uniformly bounded, by the dominated convergence theorem we can interchange the 
    expectation and derivative operators. 
    Hence \namedref{ass:GMM2b} and \namedref{ass:GMM2c} are equivalent to the statements in terms of the moment
    conditions themselves mutatis mutandis.  
    In addition, sice the derivate is a linear operator, we can pull it outside of the norm.
    The reason that the uniform law of large numbers in \cref{part:main_theorem_proof_part1} does not trivially
    imply this result is because we are not considering sequences $\phi_T \to \phi_0$. 


    We create a mean value expansions around around $(\phi_0, \zeta_T, \pi_T)$ of the sample moment condition and
    around $(\phi_0, \zeta_0, \pi_0)$ for the population moment condition.
    (This is not the same in both cases, not is it the true parameter for the drifting sequence in the case of the
    sample moment condition.)
    In addition, also since we are considering continuous functions of compact spaces --- the $\delta_T$ ball in
    $\R^{\dim(\omega)}$ --- pointwise convergence implies uniform convergence, and so we only need to show pointwise
    convergence below.

    \begin{alignat}{2}
        & &&\norm*{\sampmom(\phi_T, \zeta_T, \pi_T) -  \E\left[\popmom(\phi_T, \zeta_T, \pi_T) \mvert
          \gamma_0\right] } \\ 
        \intertext{We take a mean value expansion of both functions around $\omega_0$. The point at which the
        derivative in the two locations is taken may not be the same.}
        &= &&\left\lVert \sampmom(\phi_0, \zeta_0, \pi_0) + \frac{\partial}{\partial (\phi, \zeta,
           \pi)}\sampmom(\widetilde{\phi}^s, \widetilde{\zeta}^s, \widetilde{\pi}^s)\left((\phi_0, \zeta_0, \pi_0)
            - (\phi, \zeta, \pi)\right)\right. \\
        &  &&\quad \left. - \E\left[\popmom(\phi_0, \zeta_0, \pi_0) \mvert \gamma_0\right] + 
           \frac{\partial}{\partial (\phi, \zeta, \pi)} \E\left[\popmom(\widetilde{\phi}^p, \widetilde{\zeta}^p,
           \widetilde{\pi}^p)\mvert \gamma_0\right] \left((\phi_0, \zeta_0, \pi_0) - (\phi, \zeta,
           \pi)\right) \right\rVert \\ 
        \intertext{By the triangle inequality.}
        &\leq && \norm*{\sampmom(\phi_0, \zeta_0, \pi_0) - \E\left[\popmom(\phi_0, \zeta_0, \pi_0) \mvert
           \gamma_0\right]}  \\
        &+  && \norm*{\frac{\partial}{\partial (\phi, \zeta)}\sampmom(\widetilde{\phi}^s, \widetilde{\zeta}^s,
          \widetilde{\pi}^s)\left((\phi_0, \zeta_0)  - (\phi, \zeta)\right) -  \frac{\partial}{\partial (\phi,
          \zeta)} \E\left[\popmom(\widetilde{\phi}^s, \widetilde{\zeta}^p, \widetilde{\pi}^p)\mvert \gamma_0\right]
          \left((\phi_0, \zeta_0) - (\phi, \zeta)\right)} \\
        &+  && \norm*{\frac{\partial}{\partial \pi}\sampmom(\widetilde{\phi}^s, \widetilde{\zeta}^s,
          \widetilde{\pi}^s)\left(\pi_0 - \pi\right) -  \frac{\partial}{\partial \pi}
          \E\left[\popmom(\widetilde{\phi}^p, \widetilde{\zeta}^p, \widetilde{\pi}^p)\mvert \gamma_0\right]
          \left(\pi_0 - \pi\right)} 
          \label{eqn:pi_derivative_term}
    \end{alignat}

    By the uniform law of law numbers in \cref{part:main_theorem_proof_part1}, the first equation is $o_p(1)$.
    For $(\phi_0, \zeta_0) - (\phi, \zeta)$ small, the middle term is bounded by the quantity below. 


    \begin{equation}
        \norm*{\frac{\partial}{\partial (\phi, \zeta)}\sampmom(\widetilde{\phi}^s, \widetilde{\zeta}^s,
        \widetilde{\pi}^s) -  \frac{\partial}{\partial (\phi, \zeta)} \E\left[\popmom(\widetilde{\phi}^s,
        \widetilde{\zeta}^p, \widetilde{\pi}^p)\mvert \gamma_0\right]} \norm*{(\phi_0, \zeta_0) - (\phi, \zeta)}
    \end{equation}

    The first term is almost surely bounded, and the second term is less than $\delta_T$ by assumption, and so the
    product is $o_p(1)$.

    The hard part is the third expression.
    Like before we can bound the pull the $\pi_0 - \pi$ term out of the equation.
    However, this term is no longer converges to zero.
    We will consider the two cases, separately.
    Throughout, we will refer to the behavior of the following equation, which bounds
    \cref{eqn:pi_derivative_term}.


    \begin{equation}
        \norm*{\frac{\partial}{\partial \pi}\sampmom(\widetilde{\phi}^s, \widetilde{\zeta}^s, \widetilde{\pi}^s) -
        \frac{\partial}{\partial \pi} \E\left[\popmom(\widetilde{\phi}^p, \widetilde{\zeta}^p,
        \widetilde{\pi}^p)\mvert \gamma_0\right] } \norm*{\pi_0 - \pi} 
        \label{eqn:pi_derivative_norm_bound}
    \end{equation}

    In general, $\tilde{\pi}^s$ and $\tilde{\pi}^p$ can be arbitrarily far apart.
    However, since $\sampmom \pto \E\left[\popmom \mvert \gamma_0\right]$, and the limiting value is independent of
    $\pi$, the derivative does not depend upon $\pi$ asymptotically by the dominated convergence theorem.
    This applies that the difference between the two derivatives evaluated at different $\pi$ converges to zero.
    Consequently, \cref{eqn:pi_derivative_norm_bound} is $o_p(1)$ and we have shown \namedref{ass:GMM2b}.

    \begin{equation}
        \norm*{\abs*{\frac{\partial}{\partial \pi}\sampmom(\widetilde{\phi}^s, \widetilde{\zeta}^s,
        \widetilde{\pi}^s) - \frac{\partial}{\partial \pi} \E\left[\popmom(\widetilde{\phi}^p,
        \widetilde{\zeta}^p, \widetilde{\pi}^p)\mvert \gamma_0\right]} (1, 1, T)} \norm*{\pi_0 - \pi} 
        \label{eqn:pi_derivative_rescaled_bound}
    \end{equation}

    If we consider the setup in \namedref{ass:GMM2c}, $\tilde{\pi}^s$ and $\tilde{\pi}^p$ are now close together.
    However, we need to show that \cref{eqn:pi_derivative_rescaled_bound} is $o_p(1)$.
    Since we are considering the limiting behavior of a function with a continuous derivative, we can assume that
    the derivative is uniformly bounded without loss of generality. 
    (The constant might depend upon the true value, but not $T$.)
    By a Taylor series expansion of $\sampmom$ around the true value, $\frac{\widetilde{\pi}^p}{\sqrt{T}} \to 0$,
    and $\norm{\pi_0 - \pi} \propto \delta_T$ the result follows.

\end{proofpart}

\begin{proofpart}
    \label{part:mainTheoremProofPart3}

    Assumption \namedref{ass:GMM3a} is trivially satisfied,  and we showed that \namedref{ass:GMM3b} is satisfied
    in \cref{sec:GMM}.  
        
    The conceptual idea driving the reulsts this section is that moment conditions for each $T$ minus their
    conditional exceptions converge to a normal random variable.
    In other words, we are almost in a standard triangular C.L.T.\@ setup with weak time-dependence.

    In particular, both $\sigma^2_t$ and $r_t$ are infinitely differentiable functions of the innovations to
    the volatility and return processes and $\popmom$ are infinitely differentiable functions of $r_{t+1}$ and
    $r_t$ and the innovations are i.i.d.\@ across time by assumption.
    (Note, i.i.d.\@ implies strong mixing of any size.)
    Consequently, $\popmom$ is near epoch dependent (NED) of any size as defined in
    \textcite{andrews1991empirical}.  
    (Take $s>2.$) 
    By \textcite[Theorem 3]{andrews1991empirical}, we have the necessary finite-dimensional convergence in
    distribution to a Gaussian random variable. 

    We now show that \namedref{ass:GMM3d} holds.
    Clearly, $K_{t,g}\left(\omega \mvert \gamma^{*}\right)$ always exists.
    It uniformly converges because the derivatives of the moments are continuous functions of the data and the
    parameters, the process is ergodic, and the characteristic function lives on a compact set.
    It is also clearly continuous.
    By the dominated convergence theorem, we can exchange the derivative and expectations.
    In addition, since \popmom\ does not depend upon $\phi_T$, the limiting behavior is independent of the value
    of $\phi_T$.
    Hence \namedref{ass:GMM1d} holds.
    
    \namedref{ass:GMM3e} says the derivative of the moment function does not depend the true parameter $\phi$ if
    and only if  $\pi =  \pi_0$. 
    We showed this in the proof of \cref{part:mainTheoremProofPart2}.
    \namedref{ass:GMM3f} follows directly from the compactness of the parameter space and the continuity of the
    cross derivatives of \popmom\ by the dominated convergence theorem.

\end{proofpart}

\begin{proofpart}
    \label{mainTheoremProofPart4}
   
    \namedref{ass:GMM4a} holds trivially.

    \purple{This is no longer true.}
    We verified \namedref{ass:GMM4b} when we showed that $\pi$ and $\zeta$ are strongly identified

    \namedref{ass:GMM4c} holds because the identification conditions do not create any singularity in the
    asymptotic covariance matrix. 

\end{proofpart}

\end{proof}





\end{appendices}


\end{document}


