\documentclass[11pt, letterpaper, twoside]{article}
\usepackage{risk_price_inference}
\addbibresource{risk_price_inference}

\author{Xu Cheng\thanks{University of Pennsylvania, The Perelman Center for Political Science and Economics, 133 South 36th Street, Philadelphia, PA 19104, \href{mailto:xucheng@upenn.edu}{xucheng@upenn.edu}} \and Eric Renault\thanks{Brown University, Department of Economics -- Box B, 64 Waterman Street, Providence, RI 02912, \href{mailto:eric_renault@brown.edu}{eric\_renault@brown.edu}} \and Paul Sangrey\thanks{University of Pennsylvania, The Perelman Center for Political Science and Economics, 133 South 36th Street, Philadelphia, PA 19104, \href{mailto:paul@sangrey.io}{paul@sangrey.io}}}

\title{Identification Robust Inference for Risk Prices in Structural Stochastic Volatility Models}

\date{\today}

\begin{document}

\begin{titlepage}


\maketitle
\thispagestyle{empty}
\addtocounter{page}{-1}

\begin{abstract} 

\singlespacing \noindent 
In structural stochastic volatility asset pricing models, changes in volatility affect risk premia through two channels: (1) the investor's willingness to bear high volatility in order to get high expected returns as measured by the market risk price, and (2) the investor’s direct aversion to changes in future volatility as measured by the volatility risk price. Disentangling these channels is difficult and poses a subtle identification problem that invalidates standard inference. We adopt the discrete-time exponentially affine model of \textcite{han2018leverage}, which links the identification of volatility risk price to the leverage effect. In particular, we develop a minimum distance criterion that links the market risk price, the volatility risk price, and the leverage effect to the well-behaved reduced-form parameters governing the return and volatility's joint distribution. The link functions are almost flat if the leverage effect is close to zero, making estimating the volatility risk price difficult. We apply the conditional quasi-likelihood ratio test \textcite{andrews2016conditional} develop in a nonlinear GMM framework to a minimum distance framework. The resulting conditional quasi-likelihood ratio test is uniformly valid. We invert this test to derive robust confidence sets that provide correct coverage for the risk prices regardless of the leverage effect's magnitude. 

\end{abstract} 

\vspace{\baselineskip}

\jelcodes{C12, C14, C38, C58, G12}

\vspace{\baselineskip}

\keywords{weak identification, robust inference, stochastic volatility, leverage, market risk premium, volatility risk premium, risk price, confidence set, asymptotic size}

\end{titlepage}

% \tableofcontents
\clearpage

\section{Introduction}

An important question in finance is how investors optimally trade off risk and return. Economic theories predict investors demand a higher return as compensation for bearing more risk. Hence, we should expect a positive relationship between the mean and volatility of returns, see \gentextcites{sharpe1964capital,lintner1965security} for the capital asset pricing model (CAPM).  %[XC. Even without stochastic volatility, the relationship does not have to be linear because the pricing kernel can take different functional form.]% 
In models with stochastic volatility, the relationship is often nonlinear \parencites{bansal2014volatility, dewbecker2017price}: %[XC. I deleted option pricing given that we do not consider that here.] 
The investors care not just about how an asset's returns co-move with the volatility but also care how they co-move with the change in volatility. %[XC. I think we should add why stochastic volatility is important for asset pricing.]

XC. As we discussed, this first paragraph needs rewriting.

In structural stochastic volatility models, changes in volatility affect risk premia through two channels: (1) the investor's willingness to tolerate high volatility in order to get high expected returns as measured by the market risk price, and (2) the investor’s direct aversion to changes in future volatility as measured by the volatility risk price. We adopt the discrete-time exponentially affine model of \textcite{han2018leverage}, where the market risk price and the volatility risk price are represented by two structural parameters. In this model,  \textcite{han2018leverage} establish the important result that the identification of the volatility risk price depends on a substantial leveage effect, which is the negative correlation between return and change in volatility. 

Although the leverage effect is theoretically less than zero, it is difficult to quantify empirically and its estimate usually is small \parencites{aitsahalia2013leverage}. With a mild leverage effect, the data only provide limited amount of information about the volatility risk, compared to the finite-sample noise in the data. This low signal-to-noise ratio,  modelled by weak identification, invalidates standard inference based on the generalized method of moments (GMM) estimator, see Stock and Wright (2000) and \textcite{andrews2012estimation}.

We provide identification-robust confidence set (CS) for the structural parameters that measure the market risk price, the volatility risk price, and the leverage effect. 
The robust CS provides correct asymptotic coverage, uniformly over a large set of models that allow for any amount of leverage effect. This uniform validity is crucial for the CS to have good finite-sample coverage (Mikusheva, 2007; Andrews and Guggenberger, 2010). In constrast, standard CS based on the GMM estimator and its asymptotic normality does not have uniform validity in the presence of small leverge effect. This applies to all structural parameters because they are estimated simultaneously.


The robust inference is achieved in two-steps. First, we establish a minimum distance criterion using link functions between the structural parameter and a set of reduced-form parameters that determine the joint distribution of the return and volatility. The structural model implies that the link functions are zero when evaluated at the true values of the structral parameters and the reduced-form paramters. Identification and estimation of these reduced fom parameters are standard and not affected by small leverage effect. However, the link function is almost flat in the structural parameter when the leverage effect is small, resulting in weak identfication. Second, given this minimum distance crtiterion, we invert the conditional quasi-likelihood ratio (QLR) test by Andrews and Mikusheva (2016) to constrcut a robust CS. The key feature of this test is that the flat link function is treated as an infinite dimensional nuisance parameter. The critical value is contructed by conditioning on a sufficient statistic for this nuisance parameter and it is shown to yield a valid test regardless of the nuisance parameter. Andrews and Mikusheva (2016) develop test in a GMM framework and we apply it to the minimum distance context here.



Our paper relates to the empirical analysis of the effect of volatility on risk premia. As \textcite{lettau2010measuring} mention,  the evidence here is inconclusive. \textcites{bollerslev1988capital, harvey1989timevarying, ghysels2005there, bali2006there, ludvigson2007empirical} find a positive relationship, while \textcites{campbell1987stock, breen1989economic, pagan1991nonparametric, whitelaw1994time, brandt2004relationship} find a negative relationship. In addition, some papers use both a market risk factor and a variance risk factor to explain the risk premia dynamics, including \textcites {christoffersen2013capturing, feunou2014risk, dewbecker2017price}. A substantial positive variance risk premium is documented by \textcites {bollerslev2008risk, drechsler2011whats}. We contribute to this literature by providing a new way to do inference for the market risk price and the volatility risk price. This new confidence set not only allows for both effects but also takes into account the potential identification issue.


The weak identification issue studied in this paper is relevant in many economic applications, ranging from linear instrumental variable models (Staiger and Stock, 1997) to nonlinear structural models (Mavroeidis, Plagborg-Møller, Stock, 2014; Andrews and Mikusheva, 2015, QE). This is the first paper to study this issue in asset pricing models with structural volatility. The conditional inference approach is introduced to the linear instrumental variable model by Moreia (2003) and applied to the nonlinear GMM problem by Kleibergen (2005). Andrews and Mikusheva (2016) propose conditional inference for nonlinear GMM problems with infinitely dimensional nuisance parameter. This paper extends the scope of its application to a minimum distance criterion and to a new type of asset pricing model.

The rest of the paper is organized as follows. \cref{sec:model} provides the model and its parameterization. Section ** provide model-implied restrictions and use them to derive the link function. Section ** provides the asymptotic distribution of the reduced-form parameter and robust inference for the structural parameter. A detailed algorithm to construct the robust confidence set is given in Section **. Section ** provides simulation results and an empirical application. Section ** concludes. Proofs are given in the appendix.

\section{The Model}\label{sec:model}

%\addtocounter{subsection}{1}

This section provides a parametric structural model with stochastic volatility, following \textcite{han2018leverage}. We specify this model using a stochastic discount factor (SDF), also called the pricing kernel, and the physical measure, which gives the joint distribution of the return and volatility dynamics. Since we do not have options data, the SDF is unobserved. We first define the SDF and parameterize it as an exponential affine function with unknown parameters. Then we provide parametric distribution for the physical measure. 

  
Let $P_t$ be the price of the asset. Let $r_{t+1}=log(P_{t+1}/P_t)-r_f$ denote the log excess return minus the risk-free rate and $\sigma^2_{t+1}$ denote its volatility. The observed data is $W_t=(r_t,\sigma^2_{t})$ for $t=1,...,T$. 
Let $\F_t$ be the representative investor's information set at time $t$ . 


\subsection{Stochastic Discount Factor and Its Parameterization}\label{sec:deriving_sdf_functions}

The stochastic discount factor (SDF), denoted by $M_{t,t+1}$, prices all of the assets following the asset pricing equation 
  \begin{equation}
    P_t = \E\left[M_{t,t+1} P_{t+1} exp\left(-r_f\right) \mvert \F_t \right]. 
  \end{equation}
Following the definition of $r_t$, the pricing equation implies that for all assets
\begin{equation}
1 = \E\left[M_{t,t+1} exp\left(r_{t+1}\right) \mvert \F_t \right].
\end{equation}


We start by parameterizing the SDF by the exponential affine model. Let $\pi$ be the price of volatility risk and $\theta$ be the price of market risk.

\begin{defn}{Parameterize The Stochastic Discount Factor}
 \label{defn:SDF}
 \begin{equation}
 M_{t,t+1}(\pi, \theta) = \exp\left(m_{0} + m_1 \sigma_t^2 - \pi \sigma^2_{t+1} -
 \theta r_{t+1}\right). 
 \end{equation}
\end{defn}

Throughout we assume that the two risks that command nonzero prices are the market risk price and the volatility risk price. Consequently, we only use variation in the first two moments of the data to estimate these parameters. If higher moments, such as skewness and kurtosis are also priced factors, as in \textcites{harvey2000conditional, conrad2012exante, chang2013market}, our model is misspecified.





\subsection{Parameterizing the Volatility and Return Dynamics}

Next, we parameterize the joint distribution of $\left\lbrace W_t:t=1,...,T\right\rbrace $. 
Following \textcite{han2018leverage}, we make the following assumptions. (i) The return $r_t$ and volatility $\sigma^2_t$ are first-order Markov. (ii) There is no Granger-causality from the return to the volatility. (iii) Returns are independent given the volatility. We do allow $\sigma^2_{t}$ and $r_{t}$ to be contemporaneously correlated, as they are in the data. 
Under these assumptions, the volatility drives all of the dynamics of the process. The only relevant information in the information set $\F_{t}$ for time $t+1$-measurable variables is contained in $\sigma^2_t$. In general, $\sigma^2_t$, $\sigma^2_{t+1}$, and $r_{t+1}$ form a sufficient statistic for $\F_{t+1}$. 


We adopt the conditional autoregressive gamma process as in \textcite{gourieroux2006autoregressive, han2018leverage} for the volatility process. The model is parameterized in terms of the Laplace transform: 
%
\begin{equation}
    \E\left[\exp(-x \sigma^2_{t+1}) \mvert \F_{t}\right] = \exp\left(- A(x) \sigma^2_{t} - B(x)\right)
\end{equation}
%
for all $x \in \R$. The function $A(x)$ and $B(x)$ are parameterized as follows.


\begin{defn}{Parameterize the Volatility Dynamics}
     \label{defn:physical_vol_dynamics}
     \begin{align}
        \label{defn:a_PP}
        A(x) &= \frac{\rho x}{1 + c x}, \\
        \label{defn:b_PP}
        B(x) &= \delta \log(1 + c x),
     \end{align}
with $\rho \in [0,1-\epsilon],$ $c > \epsilon$, $\delta > \epsilon$ for some $\epsilon > 0$.
\end{defn}

In this specification, $\rho$ is a persistence parameter, and $c$ and $\delta$ are scaling parameters. Under this specification, we have the following conditional mean and variance for $\sigma^2_{t+1}$.


\begin{remark}[Volatilty Moment Conditions] 
 \label{remark:vol_moment_conditions}
    \begin{align}
        \E\left[\sigma^2_{t+1} \mvert \sigma^2_t \right] &= \rho \sigma^2_t + c \delta\\
%   
        \Var\left[\sigma^2_{t+1} \mvert \sigma^2_t \right] &= 2 c \rho \sigma^2_t + c^2 \delta 
%   
    \end{align}
\end{remark}

Since these two moment conditions are sufficient to derive the unconditional moments, all of the parameters are identified as long as they are in the interior of their appropriately specified domains. We can use linear regressions to estimate the slope and intercept parameters.

%\subsubsection{Return Dynamics}


Next, we model the return dynamics. Similar to the volatility dynamcs, the distribution of $r_t$ given $\sigma^2_{t+1}$ and $\sigma^2_{t}$ is specified in terms of the Laplace transform:
%
\begin{equation}
    \E\left[\exp(- x r_{t+1}) \mvert \F_{t}, \sigma^2_{t+1} \right] = \exp\left(- C(x) \sigma^2_{t+1} - D(x) \sigma^2_t - E(x)\right).
\end{equation}
%
for all $x \in \R$. The function $C(x)$, $D(x)$, and $E(x)$ are parameterized as follows such that the return has a conditional Gaussian distribution.

\begin{defn}{Parameterize the Return Dynamics}
    \label{defn:physical_return_dynamics}
    \begin{align}
        C(x) &\coloneqq \psi x + \frac{1 - \phi^2}{2} x^2,\\
        D(x) &\coloneqq \beta x, \\
        E(x) &\coloneqq \gamma x.
    \end{align}
with $\phi \in [-1+\epsilon, 0]$ for some $\epsilon>0$.
\end{defn}



Under this specification, we have the following representation of the conditional mean and variance for $r_{t+1}$.

\begin{remark}[Return Moment Conditions] 
	\label{remark:return_moment_conditions}
	\begin{align}
		\label{eqn:rtn_cond_mean}
		\E\left[r_{t+1} \mvert \sigma^2_t, \sigma^2_{t+1}\right] = \psi \sigma^2_{t+1} + \beta \sigma^2_t + \gamma \\
		%   
		\label{eqn:rtn_cond_vol}
		\Var\left[r_{t+1} \mvert \sigma^2_t, \sigma^2_{t+1}\right] = (1 - \phi^2) \sigma^2_{t+1}.
		%   
	\end{align}
\end{remark}


Both the parameter $\psi$ and $\phi$ are linked to the leverage effect as shown below.

********PUT THE SCIENTIFIC WORKPLACE NOTES HERE*************


\section{Conclusion}

we have reduce the overlap with the abstract and intro. I will think about it.

%In structural stochastic volatility models such as the one developed here, changes in the volatility affect returns through two channels. First, the investor's willingness to tolerate high volatility in order to get high expected returns as measured by the price of market risk. Standard economic models imply there is a static trade-off between risk and expected return. Consequently, when the risk changes the expected return must change to restore equilibrium. However, investors may also be directly averse to changes in future volatility, and the empirical evidence shows that they are. \Textcite{han2018leverage} shows that we can disentangle these two channels through their differing relationships to the leverage effect. However, estimating the leverage effect is quite tricky, \parencite{aitsahalia2013leverage}. This difficulty poses a subtle identification problem that invalidates standard inference. When the data's signal-to-noise ratio is small, as it is here, standard tests and confidence intervals provide misleading results. We adopt the discrete-time exponentially-affine of \textcite{han2018leverage} and adapt weak identification methods to this framework to ensure that the resulting confidence intervals are uniformly valid.
%
%In particular, we develop a minimum distance criterion that links the market risk price, the volatility risk price, and the leverage effect to some well-behaved reduced-form parameters that govern the return and volatility's joint distribution. We do this by adapting the conditional quasi-likelihood ratio test (CLR) \textcite{andrews2016conditional} develop in a GMM framework to a minimum distance framework. The resulting CLR test is uniformly valid. We invert this test to derive a robust confidence set. We then apply this methodology to data on the S\&P 500 and show that the market risk price lies between \num{0.20} and \num{0.60} yearly percentage points, and the volatility risk price lies between \num{-0.20} and \num{0.00} yearly percentage points. These estimates are both substantially smaller in magnitude than the values chosen by \textcite{han2018leverage}.
%

\clearpage

\end{document}


