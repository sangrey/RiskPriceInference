\documentclass[11pt, letterpaper, twoside]{article}
\usepackage{risk_price_inference}
\usepackage[backend=biber, autopunct=true, authordate, hyperref=true, doi=false,
isbn=false, url=false, eprint=false]{biblatex-chicago} 
\addbibresource{risk_price_inference.bib}

\author{Xu Cheng\thanks{University of Pennsylvania, The Perelman Center for Political Science and Economics, 133 South 36th Street, Philadelphia, PA 19104, \href{mailto:xucheng@upenn.edu}{xucheng@upenn.edu}} \and Eric Renault\thanks{Brown University, Department of Economics -- Box B, 64 Waterman Street, Providence, RI 02912, \href{mailto:eric_renault@brown.edu}{eric\_renault@brown.edu}} \and Paul Sangrey\thanks{University of Pennsylvania, The Perelman Center for Political Science and Economics, 133 South 36th Street, Philadelphia, PA 19104, \href{mailto:paul@sangrey.io}{paul@sangrey.io}}}

\title{Identification Robust Inference for Risk Prices in Structural Stochastic Volatility Models}

\date{\today}

\begin{document}

\begin{titlepage}


\maketitle
\thispagestyle{empty}
\addtocounter{page}{-1}

\begin{abstract} 

\singlespacing \noindent 
In structural stochastic volatility asset pricing models, changes in volatility affect risk premia through two channels: (1) the investor's willingness to bear high volatility in order to get high expected returns as measured by the market risk price, and (2) the investor’s direct aversion to changes in future volatility as measured by the volatility risk price. Disentangling these channels is difficult and poses a subtle identification problem that invalidates standard inference. We adopt the discrete-time exponentially affine model of \textcite{han2018leverage}, which links the identification of volatility risk price to the leverage effect. In particular, we develop a minimum distance criterion that links the market risk price, the volatility risk price, and the leverage effect to the well-behaved reduced-form parameters governing the return and volatility's joint distribution. The link functions are almost flat if the leverage effect is close to zero, making estimating the volatility risk price difficult. We apply the conditional quasi-likelihood ratio test \textcite{andrews2016conditional} develop in a nonlinear GMM framework to a minimum distance framework. The resulting conditional quasi-likelihood ratio test is uniformly valid. We invert this test to derive robust confidence sets that provide correct coverage for the risk prices regardless of the leverage effect's magnitude. 

\end{abstract} 

\vspace{\baselineskip}

\jelcodes{C12, C14, C38, C58, G12}

\vspace{\baselineskip}

\keywords{weak identification, robust inference, stochastic volatility, leverage, market risk premium, volatility risk premium, risk price, confidence set, asymptotic size}

\end{titlepage}

% \tableofcontents
\clearpage

\section{Introduction}

An important question in finance is how investors optimally trade off risk and return. Economic theories predict investors demand a higher return as compensation for bearing more risk. Hence, we should expect a positive relationship between the mean and volatility of returns, see \gentextcites{sharpe1964capital,lintner1965security} for the capital asset pricing model (CAPM).  %[XC. Even without stochastic volatility, the relationship does not have to be linear because the pricing kernel can take different functional form.]% 
In models with stochastic volatility, the relationship is often nonlinear \parencites{bansal2014volatility, dewbecker2017price}: %[XC. I deleted option pricing given that we do not consider that here.] 
The investors care not just about how an asset's returns co-move with the volatility but also care how they co-move with the change in volatility. %[XC. I think we should add why stochastic volatility is important for asset pricing.]

%XC. As we discussed, this first paragraph needs rewriting.

In structural stochastic volatility models, changes in volatility affect risk premia through two channels: (1) the investor's willingness to tolerate high volatility in order to get high expected returns as measured by the market risk price, and (2) the investor’s direct aversion to changes in future volatility as measured by the volatility risk price. We adopt the discrete-time exponentially affine model of \textcite{han2018leverage}, where the market risk price and the volatility risk price are represented by two structural parameters. In this model,  \textcite{han2018leverage} establish the important result that the identification of the volatility risk price depends on a substantial leverage effect, which is the negative correlation between return and change in volatility. 

Although the leverage effect is theoretically less than zero, it is difficult to quantify empirically and its estimate usually is small \parencites{aitsahalia2013leverage}. With a mild leverage effect, the data only provide limited amount of information about the volatility risk, compared to the finite-sample noise in the data. This low signal-to-noise ratio,  modeled by weak identification, invalidates standard inference based on the generalized method of moments (GMM) estimator, see Stock and Wright (2000) and \textcite{andrews2012estimation}.

We provide identification-robust confidence set (CS) for the structural parameters that measure the market risk price, the volatility risk price, and the leverage effect. 
The robust CS provides correct asymptotic coverage, uniformly over a large set of models that allow for any amount of leverage effect. This uniform validity is crucial for the CS to have good finite-sample coverage \parencites{mikusheva2007uniform, andrews2010applications}. In contrast, standard confidence sets based on the GMM estimator and its asymptotic normality does not have uniform validity in the presence of small leverage effect. This applies to all structural parameters because they are estimated simultaneously.


The robust inference is achieved in two-steps. First, we establish a minimum distance criterion using link functions between the structural parameter and a set of reduced-form parameters that determine the joint distribution of the return and volatility. The structural model implies that the link functions are zero when evaluated at the true values of the structural parameters and the reduced-form parameters. Identification and estimation of these reduced form parameters are standard and not affected by small leverage effect. However, the link function is almost flat in the structural parameter when the leverage effect is small, resulting in weak identification. Second, given this minimum distance criterion, we invert the conditional quasi-likelihood ratio (QLR) test by \textcite{andrews2016conditional} to construct a robust confidence set. The key feature of this test is that the flat link function is treated as an infinite dimensional nuisance parameter. The critical value is constructed by conditioning on a sufficient statistic for this nuisance parameter and it is shown to yield a valid test regardless of the nuisance parameter. \Textcite{andrews2016conditional} develop a test in a GMM framework, and we apply it to the minimum distance context here.


Our paper relates to the empirical analysis of the effect of volatility on risk premia. As \textcite{lettau2010measuring} mention,  the evidence here is inconclusive. \textcites{bollerslev1988capital, harvey1989timevarying, ghysels2005there, bali2006there, ludvigson2007empirical} find a positive relationship, while \textcites{campbell1987stock, breen1989economic, pagan1991nonparametric, whitelaw1994time, brandt2004relationship} find a negative relationship. In addition, some papers use both a market risk factor and a variance risk factor to explain the risk premia dynamics, including \textcites{christoffersen2013capturing, feunou2014risk, dewbecker2017price}. A substantial positive variance risk premium is documented by \textcites{bollerslev2008risk, drechsler2011whats}. We contribute to this literature by providing a new way to do inference for the market risk price and the volatility risk price. This new confidence set not only allows for both effects but also takes into account the potential identification issue.


The weak identification issue studied in this paper is relevant in many economic applications, ranging from linear instrumental variable models (Staiger and Stock, 1997) to nonlinear structural models \parencites{mavroeidis2014empirical, andrews2015maximum}  (Stock, 2014). This is the first paper to study this issue in structural asset pricing models with stochastic volatility. The conditional inference approach is introduced to the linear instrumental variable model by \textcite{moreira2003conditional} and applied to the nonlinear GMM problem by \textcite{kleibergen2005testing}. \Textcite{andrews2016conditional} propose conditional inference for nonlinear GMM problems with infinitely dimensional nuisance parameter. This paper extends the scope of its application to a minimum distance criterion and to a new type of asset pricing model.

The rest of the paper is organized as follows. \cref{sec:model} provides the model and its parameterization. Section ** provide model-implied restrictions and use them to derive the link function. Section ** provides the asymptotic distribution of the reduced-form parameter and robust inference for the structural parameter. A detailed algorithm to construct the robust confidence set is given in Section **. Section ** provides simulation results and an empirical application. Section ** concludes. Proofs are given in the appendix.

\section{The Model}\label{sec:model}

%\addtocounter{subsection}{1}

This section provides a parametric structural model with stochastic volatility, following \textcite{han2018leverage}. We specify this model using a stochastic discount factor (SDF), also called the pricing kernel, and the physical measure, which gives the joint distribution of the return and volatility dynamics. Since we do not have options data, the SDF is unobserved. We first define the SDF and parameterize it as an exponential affine function with unknown parameters. Then we provide parametric distribution for the physical measure. 

  
Let $P_t$ be the price of the asset. Let $r_{t+1}=log(P_{t+1}/P_t)-r_f$ denote the log excess return minus the risk-free rate and $\sigma^2_{t+1}$ denote its volatility. The observed data is $W_t=(r_t,\sigma^2_{t})$ for $t=1,...,T$. 
Let $\F_t$ be the representative investor's information set at time $t$ . 


\subsection{Stochastic Discount Factor and Its Parameterization}\label{sec:deriving_sdf_functions}

The stochastic discount factor (SDF), denoted by $M_{t,t+1}$, prices all of the assets following the asset pricing equation 
  \begin{equation}
    P_t = \E\left[M_{t,t+1} P_{t+1} exp\left(-r_f\right) \mvert \F_t \right]. 
  \end{equation}
Following the definition of $r_t$, the pricing equation implies that for all assets
\begin{equation}
1 = \E\left[M_{t,t+1} exp\left(r_{t+1}\right) \mvert \F_t \right].
\end{equation}


We start by parameterizing the SDF by the exponential affine model. Let $\pi$ be the price of volatility risk and $\theta$ be the price of market risk.

\begin{defn}{Parameterize The Stochastic Discount Factor}
 \label{defn:SDF}
 \begin{equation}
 M_{t,t+1}(\pi, \theta) = \exp\left(m_{0} + m_1 \sigma_t^2 - \pi \sigma^2_{t+1} -
 \theta r_{t+1}\right). 
 \end{equation}
\end{defn}

Throughout we assume that the two risks that command nonzero prices are the market risk price and the volatility risk price. Consequently, we only use variation in the first two moments of the data to estimate these parameters. If higher moments, such as skewness and kurtosis are also priced factors, as in \textcites{harvey2000conditional, conrad2012exante, chang2013market}, our model is misspecified.





\subsection{Parameterizing the Volatility and Return Dynamics}

Next, we parameterize the joint distribution of $\left\lbrace W_t:t=1,...,T\right\rbrace $. 
Following \textcite{han2018leverage}, we make the following assumptions. (i) The return $r_t$ and volatility $\sigma^2_t$ are first-order Markov. (ii) There is no Granger-causality from the return to the volatility. (iii) Returns are independent given the volatility. We do allow $\sigma^2_{t}$ and $r_{t}$ to be contemporaneously correlated, as they are in the data. 
Under these assumptions, the volatility drives all of the dynamics of the process. The only relevant information in the information set $\F_{t}$ for time $t+1$-measurable variables is contained in $\sigma^2_t$. In general, $\sigma^2_t$, $\sigma^2_{t+1}$, and $r_{t+1}$ form a sufficient statistic for $\F_{t+1}$. 


We adopt the conditional autoregressive gamma process as in \textcite{gourieroux2006autoregressive, han2018leverage} for the volatility process. The model is parameterized in terms of the Laplace transform: 
%
\begin{equation}
    \E\left[\exp(-x \sigma^2_{t+1}) \mvert \F_{t}\right] = \exp\left(- A(x) \sigma^2_{t} - B(x)\right)
    \label{eqn:vol_laplace_transform}
\end{equation}
%
for all $x \in \R$. The function $A(x)$ and $B(x)$ are parameterized as follows.


\begin{defn}{Parameterize the Volatility Dynamics}
     \label{defn:physical_vol_dynamics}
     \begin{align}
        \label{defn:a_PP}
        A(x) &= \frac{\rho x}{1 + c x}, \\
        \label{defn:b_PP}
        B(x) &= \delta \log(1 + c x),
     \end{align}
with $\rho \in [0,1-\epsilon],$ $c > \epsilon$, $\delta > \epsilon$ for some $\epsilon > 0$.
\end{defn}

In this specification, $\rho$ is a persistence parameter, and $c$ and $\delta$ are scaling parameters. Under this specification, we have the following conditional mean and variance for $\sigma^2_{t+1}$.


\begin{remark}[Volatilty Moment Conditions] 
 \label{remark:vol_moment_conditions}
    \begin{align}
        \E\left[\sigma^2_{t+1} \mvert \sigma^2_t \right] &= \rho \sigma^2_t + c \delta\\
%   
        \Var\left[\sigma^2_{t+1} \mvert \sigma^2_t \right] &= 2 c \rho \sigma^2_t + c^2 \delta 
%   
    \end{align}
\end{remark}

Since these two moment conditions are sufficient to derive the unconditional moments, all of the parameters are identified as long as they are in the interior of their appropriately specified domains. We can use linear regressions to estimate the slope and intercept parameters.

%\subsubsection{Return Dynamics}


Next, we model the return dynamics. Similar to the volatility dynamcs, the distribution of $r_t$ given $\sigma^2_{t+1}$ and $\sigma^2_{t}$ is specified in terms of the Laplace transform:
%
\begin{equation}
    \label{eqn:return_laplace_transform}
    \E\left[\exp(- x r_{t+1}) \mvert \F_{t}, \sigma^2_{t+1} \right] = \exp\left(- C(x) \sigma^2_{t+1} - D(x) \sigma^2_t - E(x)\right).
\end{equation}
%
for all $x \in \R$. The function $C(x)$, $D(x)$, and $E(x)$ are parameterized as follows such that the return has a conditional Gaussian distribution.

\begin{defn}{Parameterize the Return Dynamics}
    \label{defn:physical_return_dynamics}
    \begin{align}
        C(x) &\coloneqq \psi x + \frac{1 - \phi^2}{2} x^2,\\
        D(x) &\coloneqq \beta x, \\
        E(x) &\coloneqq \gamma x.
    \end{align}
with $\phi \in [-1+\epsilon, 0]$ for some $\epsilon>0$.
\end{defn}



Under this specification, we have the following representation of the conditional mean and variance for $r_{t+1}$.

\begin{remark}[Return Moment Conditions] 
	\label{remark:return_moment_conditions}
	\begin{align}
		\label{eqn:rtn_cond_mean}
		\E\left[r_{t+1} \mvert \sigma^2_t, \sigma^2_{t+1}\right] = \psi \sigma^2_{t+1} + \beta \sigma^2_t + \gamma \\
		%   
		\label{eqn:rtn_cond_vol}
		\Var\left[r_{t+1} \mvert \sigma^2_t, \sigma^2_{t+1}\right] = (1 - \phi^2) \sigma^2_{t+1}.
		%   
	\end{align}
\end{remark}


Both the parameter $\psi$ and $\phi$ are linked to the leverage effect as shown below.

\section{Link Functions}

So far, we have introduced the following parameters: $(m_{0},m_{1},\theta
,\pi )$ in SDF, $(\rho ,c,\delta )$ in the volatility dynamic, and $(\psi
,\beta ,\gamma ,\phi )$ in the return dynamic. Next, we explore restrictions
among these parameters that are consistent with this model. In other words,
not all of these parameters can change freely under the structural model.

We use these restrictions to construct link functions between a set of
reduced-form parameter and a set of structural parameters. These link
functions play an important role on separating the regularly behaved
reduced-form parameters from the structural parameters. They also are used
to conduct identification robust inference for the structural parameters
based on a minimum distance criterion.

All of these restrictions are also presented and imposed in the GMM
estimation in Han et al. (2018). However, they estimate all parameters
together with all of these restriction imposed, because they assume all
parameters being estimated are strongly identified.

\subsection{Pricing Equation Restrictions}

We first explore the restriction implied by the pricing equation $\mathbb{E[}%
M_{t,t+1}\exp (r_{t+1})|\mathcal{F}_{t}]=1.$ We first provide a simple
result stating that the constants $m_{0}$ and $m_{1}$ are normalization
constants implied by all the other parameters. Thus, $m_{0}$ and $m_{1}$ are
not free parameters to be estimated. Instead, they should take the value
given, once other parameters are specified. These restrictions on $m_{0}$
and $m_{1}$ are obtained by applying the restriction $\mathbb{E[}%
M_{t,t+1}\exp (r_{t+1})|\mathcal{F}_{t}]=1$ to the risk free asset. Applying
the same argument to any other asset, we also obtain another set of two
restrictions, which can be written in terms of coefficient $\beta $ and $%
\gamma $ under the linear form of $D(x)$ and $E(x).$

\begin{lemma}
\label{Lemma m0 and m1}Given the parameterization in the model. The pricing
equation $\mathbb{E[}M_{t,t+1}\exp (r_{t+1})|\mathcal{F}_{t}]=1$ implies that%
\begin{eqnarray*}
m_{0} &=&E(\theta )+B\left( \pi +C\left( \theta \right) \right) , \\
m_{1} &=&D\left( \theta \right) +A\left( \pi +C\left( \theta \right) \right)
,
\end{eqnarray*}%
and%
\begin{eqnarray*}
\gamma  &=&B\left( \pi +C\left( \theta -1\right) \right) -B\left( \pi
+C\left( \theta \right) \right) , \\
\beta  &=&A\left( \pi +C\left( \theta -1\right) \right) -A\left( \pi
+C\left( \theta \right) \right) .
\end{eqnarray*}
\end{lemma}

The two equalities on $\beta $ and $\gamma $ link them to the market risk
price $\theta $ and volatility risk $\pi $ through the functions $A(\cdot
),B(\cdot ),C(\cdot ),$ which also involve parameters $(\rho ,c,\delta ,\psi
,\phi ).$ We treat these two equalities as link functions in the minimum
distance criterion specified below.

\subsection{Leverage Effect Restrictions}

We first show that both parameter $\psi $ and $\phi $ are linked to the leverage effect. Given the variance of $r_{t+1}$ conditional on $(\sigma _{t+1}^{2},\sigma _{t}^{2}),$ specified in (**), we have
%
\begin{equation*}
    \phi ^{2}=\sigma _{t+1}^{2}-\Var[r_{t+1}|\sigma _{t+1}^{2},\sigma _{t}^{2}].
\end{equation*}
%
This shows that $\phi $ is linked to the leverage effect because it measures the return volatility reduction after conditioning on the volatility path.  On the other hand, given the mean of $r_{t+1}$ conditional on $(\sigma _{t+1}^{2},\sigma _{t}^{2}),$ specified in (**), we have\footnote{To see this result, note that he mean of $r_{t+1}-\psi \sigma _{t+1}^{2}$ given $(\sigma _{t+1}^{2},\sigma _{t}^{2})$ does not depend on $\sigma
_{t+1}^{2}.$}%
%
\begin{equation}
    \E[r_{t+1}|\sigma _{t+1}^{2},\sigma _{t}^{2}]-E[r_{t+1}|\sigma _{t}^{2}]=\psi \left \{ \sigma _{t+1}^{2}-E[\sigma _{t+1}^{2}|\sigma _{t}^{2}]\right \} .
\end{equation}%
The parameter $\psi $ is also linked to the leverage effect because it incorporates the instantaneous relationship between change in the innovation in $\sigma _{t+1}^{2}$ and change in the return forecast. Besides the leverage effect, $\psi $ also include the change in $r_{t+1}$ through two other channels: (i) the market risk price due to the correlation between $M_{t,t+1}$ and $\exp (r_{t+1})$ and (ii) the Jensen's effect term that captures the change in the mean of $\exp (r_{t+1})$ with the volatility of $ r_{t+1}$ (with a 1/2 factor). Following \textcite{han2018leverage}, these two measures of the leverage effect are restricted to 
%
\begin{equation}
    \psi -(1-\phi ^{2})\theta +\frac{1}{2}(1-\phi ^{2})=k\phi 
    \label{leverage restriction}
\end{equation}
%
for a constant $k,$ where the left hand side is the leverage effect in $\psi$ after the other two effects are removed. \Textcite{han2018leverage} show that an appropriate choice of $k$ is the value under which $corr[r_{t+1},\sigma _{t+1}^{2}|\sigma _{t}^{2}]=\phi $ if this correlation is indeed time invariant. Guided by this condition, they show that $k=1/(2c)^{1/2}$ should be used for the volatility dynamic specified in () and ().

\subsection{Structural and Reduced-Form Parameters}

Because $\phi $ is the leverage effect parameter, we group it together with market risk price $\theta $ and the volatility risk price $\pi $ and call $ \lambda =(\theta ,\pi ,\phi )^{\prime }$ structural parameters. These structural parameters are estimated by restrictions from this structural model. In contrast, the other parameters in the conditional mean and variance of the return and volatility, see ()-() and ()-(), are simply estimated using these moments, without any model restriction. As such, we call them the reduced-form parameter. Because $1-\phi ^{2}$ shows up in the conditional variance of $r_{t+1},$ see (**), we define $\zeta =1-\phi ^{2}$ as a reduced-form parameter and link it to the structural parameter $\phi $ through this relationship. To sum up, the reduced-form parameters are $\omega =(\rho ,c,\delta ,\psi ,\beta ,\gamma ,\zeta )^{\prime }.$

Using $\zeta $ as a reduced-form parameter has the additional benefit that
the sample variance, denoted by $\widehat{\zeta },$ is a simple consistent
estimator with normal distribution. Estimation of $\phi $ is more
complicated because $-1<\phi \leq 0$ by definition and this estimation
involves boundary constraints that results in a non-standard distribution.
The inference procedure below does not require estimation of $\phi $ and is
uniform over $\phi $ even if its true value is on or close to the boundary $%
0.$

The link functions between the structural parameter $\lambda $ and the
reduced-form parameter $\omega $ are collected together in%
\begin{equation}
   g(\lambda, \omega) = 
%
    \begin{pmatrix}
        B\left( \pi +C\left( \theta -1\right) \right) -B\left( \pi +C\left( \theta \right) \right) -\gamma  \\ 
        A\left( \pi +C\left( \theta -1\right) \right) -A\left( \pi +C\left( \theta \right) \right) -\beta  \\ 
        \psi -(1-\phi ^{2})\theta +\frac{1}{2}(1-\phi ^{2})-1/(2c)^{1/2}\phi  \\ \zeta -\left( 1-\phi ^{2}\right) 
    \end{pmatrix}.
\end{equation}%
For the inference problem studied below, we know $g(\lambda _{0},\omega
_{0})=0$ when evaluated at the true value of $\lambda $ and $\omega .$

\subsection{Identification}

One of the important contributions of Han etal is to establish the
relationship between the identification of the volatility risk price and the
leverage effect. In particular, they show that when the leverage effect
parameter $\phi =0,$ the volatility price $\pi $ is not identified. To see
this result, note that the only source of identification information on $\pi 
$ are the first two link functions in $g(\lambda _{0},\omega _{0})=0,$ which
comes from Lemma \ref{Lemma m0 and m1}. In these two equalities, $\pi $ is
not identified if $C(\theta )=C(\theta -1).$ Using the definition $C(\theta
)=\psi \theta +(1-\phi ^{2})\theta ^{2}/2$ and the restriction (\ref%
{leverage restriction}), we have 
\begin{equation*}
C(\theta )-C(\theta -1)=\psi +(1-\phi ^{2})\left( \theta -\frac{1}{2}\right)
=k\phi .
\end{equation*}%
Therefore, a non-zero leverage effect i.e., $\phi \neq 0,$ is required for
the identification of the volatility risk price $\pi $.

The identification result above is based on the theoretical model only,
without considering data uncertainty in practical applications. With a
finite-sample size and different types of noise in the data, such as
measurement errors and omitted variables, a much more substantial leverage
effect is required to obtain a standard identification situation where the
noise in the data is negligible compared to the information to identify $\pi
.$ However, if only a small leverage effect is documented, e.g., Ait-Sahalia
et al (2014), or the magnitude of the leverage effect is completely unknown,
an identification robust procedure is needed to conduct inference in this
problem. We provide such a procedure now.

\section{Robust Inference for Risk Prices}

\subsection{Asymptotic Distribution of the Reduced-Form Parameter}

Write $\omega :=(\omega _{1},\omega _{2},\omega _{3})^{\prime },$ where $%
\omega _{1}=(\rho ,c,\delta )\in O_{1},$ $\omega _{2}=(\gamma ,\beta ,\psi
)\in O_{2},$ and $\omega _{3}=\zeta \in O_{3}.$ The parameter space for $%
\omega $ is $O=O_{1}\times O_{2}\times O_{3}\subset R^{d_{\omega }}.$ The
true value of $\omega $ is assumed to be in the interior of the parameter
space.

Below we describe the estimator $\widehat{\omega }:=(\widehat{\omega }_{1},%
\widehat{\omega }_{2},\widehat{\omega }_{3})^{\prime }$ and provide its
asymptotic distribution. We estimate these parameters separately because $%
\omega _{1}$ only shows up in the conditional mean and variance of $\sigma
_{t+1}^{2},$ $\omega _{2}$ only shows up in the conditional mean of $%
r_{t+1}, $ and $\omega _{3}$ only shows up in the conditional variance of $%
r_{t+1}.$

We first estimate $\omega _{1}=(\rho ,c)$ based on the conditional mean and
variance of $\sigma _{t+1}^{2}$, which can be equivalently written as 
\begin{eqnarray}
E[\sigma _{t+1}^{2}|\sigma _{t}^{2}] &=&A\text{ and }E[\sigma
_{t+1}^{4}|\sigma _{t}^{2}]=B,\text{ where }  \notag \\
A &=&\rho \sigma _{t}^{2}+c\delta \text{ and }B=A^{2}+\left( 2c\rho \sigma
_{t}^{2}+c^{2}\delta \right) .
\end{eqnarray}%
Because the conditional mean of $\sigma _{t+1}^{2}$ and $\sigma _{t+1}^{4}$
are linear and quadratic functions, respectively, of the conditioning
variable $\sigma _{t}^{2},$ without loss of efficiency, they can be
transformed to the unconditional moments%
\begin{equation}
E[h_{t}(\omega _{10})]=0,\text{ where }h_{t}(\omega _{1})=[(1,\sigma
_{t}^{2})\otimes (\sigma _{t+1}^{2}-A),(1,\sigma _{t}^{2},\sigma
_{t}^{4})\otimes (\sigma _{t+1}^{4}-B)]^{\prime },
\end{equation}%
where $\omega _{10}$ represents the true value of $\omega _{1}.$ The
two-step GMM estimator of $\omega _{1}$ is%
\begin{equation}
\widehat{\omega }_{1}=\underset{\omega _{1}\in O_{1}}{\arg \min }\left(
T^{-1}\sum_{t=1}^{T}h_{t}(\omega _{1})\right) ^{\prime }\widehat{V}%
_{1}\left( T^{-1}\sum_{t=1}^{T}h_{t}(\omega _{1})\right) ,
\label{omega 1 est}
\end{equation}%
where $\widehat{V}_{1}$ is a consistent estimator of $V_{1}=\sum_{m=-\infty
}^{\infty }\mathbb{C}ov[h_{t}(\omega _{10}),h_{t+m}(\omega _{10})].$

We estimate $\omega _{2}$ by the generalized least squares (GLS) estimator
because the conditional mean of $r_{t+1}$ is a linear function of the
conditioning variable $\sigma _{t}^{2}$ and $\sigma _{t+1}^{2}$ and the
conditional variance is proportional to $\sigma _{t+1}^{2}.$ The GLS\
estimator of $\omega _{2}$ is%
\begin{eqnarray}
\widehat{\omega }_{2} &=&\left( \sum_{t=1}^{T}x_{t}x_{t}^{\prime }\right)
^{-1}\sum_{t=1}^{T}x_{t}y_{t},\text{ where }  \notag \\
x_{t} &=&\sigma _{t+1}^{-1}(1,\sigma _{t}^{2},\sigma _{t+1}^{2})^{\prime }%
\text{ and }y_{t}=\sigma _{t+1}^{-1}r_{t+1}.  \label{omega 2 est}
\end{eqnarray}%
We estimate $\omega _{3}$ by the sample variance estimator%
\begin{equation}
\widehat{\omega }_{3}=T^{-1}\sum_{t=1}^{T}\left( y_{t}-\widehat{y}%
_{t}\right) ^{2},\text{ where }\widehat{y}_{t}=x_{t}^{\prime }\widehat{%
\omega }_{2}.  \label{omega 3 est}
\end{equation}

Let $P$ denote the distribution of the data $\mathcal{W}=\{W_{t}=(r_{t+1},%
\sigma _{t+1}^{2}):t\geq 1\}$ and $\mathcal{P}$ denote the parameter space
of $P$. Note that the true values of the structural parameter and the
reduced-form parameters are all determined by $P.$ We allow $P$ to change
with $T.$ For notational simplicity, the dependence on $P$ and $T$ is
suppressed.

Let 
\begin{equation}
f_{t}(\omega )=\left( 
\begin{array}{c}
h_{t}(\omega _{1}) \\ 
x_{t}(y_{t}-x_{t}^{\prime }\omega _{2}) \\ 
(y_{t}-x_{t}^{\prime }\omega _{2})^{2}%
\end{array}%
\right) \in R^{d_{f}}\text{ and }V=\sum_{m=-\infty }^{\infty }\mathbb{C}%
\mathbf{ov}[f_{t}(\omega _{0}),f_{t+m}(\omega _{0})].
\end{equation}%
The estimator $\widehat{\omega }$ defined above is based on the first moment
of $f_{t}(\omega ).$ Thus, the limiting distribution of $\widehat{\omega }$
relates to the limiting distribution of $T^{-1/2}\sum_{t=1}^{T}f_{t}(\omega
_{0})-\mathbb{E[}f_{t}(\omega _{0})$ following from the central limit
theorem. Furthermore, because $\omega _{1}$ is the GMM\ estimator based on
some nonlinear moment conditions, we need uniform convergence of the the
sample moments and their derivatives to show the consistency and asymptotic
normality of $\widehat{\omega }_{1}.$ These uniform convergence follows from
the uniform law of large numbers. Because $\widehat{\omega }_{2}$ is a
simple OLS estimator by regressing $y_{t}$ and $x_{t},$ we need
no-multicollinearity among the regressors. We make all of the these
assumptions below.. All of them are easily verifiable with weakly dependent
time series data.

Let $\widehat{V}$ denote a heteroskedasticity and autocorrelation consistent
(HAC) estimator of $V$. The estimator $\widehat{V}_{1}$ is a submatrix of $%
\widehat{V}$ associate with $V_{1}.$ Let $H_{t}(\omega _{1})=\partial
h_{t}(\omega _{1})/\partial \omega _{1}^{\prime }.$

\smallskip

\noindent \textbf{Assumption R}. The following conditions hold uniformly
over $P\in \mathcal{P}$, for some fixed $0<C<\infty .$

\noindent (i) $T^{-1}\sum_{t=1}^{T}(h_{t}(\omega _{1})-\mathbb{E[}%
h_{t}(\omega _{1}))\rightarrow _{p}0$ and $T^{-1}\sum_{t=1}^{T}(H_{t}(\omega
_{1})-\mathbb{E[}H_{t}(\omega _{1})\mathbb{])}\rightarrow _{p}0,$ $\mathbb{E[%
}H_{t}(\omega _{1})\mathbb{]}$ is continuous in $\omega _{1},$ all uniformly
over the parameter space of $\omega _{1}.$

%\QTP{Body Math}

\noindent (ii) $T^{-1}\sum_{t=1}^{T}(x_{t}x_{t}^{\prime }-\mathbb{E[}%
x_{t}x_{t}^{\prime }\mathbb{])\rightarrow }_{p}0.$

%\QTP{Body Math}
\noindent (iii) $V^{-1/2}\{T^{-1/2}(\sum_{t=1}^{T}f_{t}(\omega _{0})-\mathbb{%
E[}f_{t}(\omega _{0})\mathbb{]\} \rightarrow }_{d}N(0,I)$ and $\widehat{V}%
-V\rightarrow _{p}0.$

\noindent (iv) $C^{-1}\leq \lambda _{\min }(A)\leq \lambda _{\max }(A)\leq C$
for $A=V,\mathbb{E[}H_{t}\left( \omega _{1,0}\right) ^{\prime }H_{t}\left(
\omega _{1,0}\right) ]),\mathbb{E[}x_{t}x_{t}^{\prime }],\mathbb{E[}%
z_{t}z_{t}^{\prime }],$ where $z_{t}=(1,\sigma _{t}^{2},\sigma
_{t}^{4})^{\prime }.$

\smallskip

Let $H(\omega _{1})=\mathbb{E[}H_{t}(\omega _{1})]$ and $\overline{H}(\omega
_{1})=T^{-1}\sum_{t=1}^{T}H_{t}(\omega _{1}).$ Define%
\begin{eqnarray}
\mathcal{B} &=&\diag\{[H(\omega _{10})V_{1}^{-1}H(\omega _{10})]^{-1}H(\omega
_{10})V_{1}^{-1},\mathbb{E[}x_{t}x_{t}^{\prime }]^{-1},1\},  \notag \\
\widehat{\mathcal{B}} &=&\diag\{[\overline{H}(\widehat{\omega }_{1})^{\prime }%
\widehat{V}_{1}^{-1}\overline{H}(\widehat{\omega }_{1})]^{-1}\overline{H}(%
\widehat{\omega }_{1})^{\prime }\widehat{V}_{1}^{-1},[T^{-1}%
\sum_{t=1}^{T}x_{t}x_{t}^{\prime }]^{-1},1\}.  \label{Fhat}
\end{eqnarray}%
The following Lemma provides the asymptotic distribution of the reduced-form
parameter and a consistent estimator of its asymptotic covariance. Note that
we put the asymptotic covariance on the left side of the convergence to
allow the distribution of the data to change with sample size $T.$

\begin{lemma}
\label{Lemma Reduce}Suppose Assumption R holds. The following results hold
uniformly over $P\in \mathcal{P}$.

\noindent \emph{(i)} $\xi _{T}:=\Omega ^{-1/2}T^{-1/2}(\widehat{\omega }%
-\omega _{0})\rightarrow _{d}\xi \sim N(0,I),$ where $\Omega =\mathcal{B}V%
\mathcal{B}^{\prime }.$

\noindent \emph{(ii)} $\widehat{\Omega }-\Omega \rightarrow _{p}0,$ where $%
\widehat{\Omega }=\widehat{\mathcal{B}}\widehat{V}\widehat{\mathcal{B}}%
^{\prime }.$
\end{lemma}

\subsection{Weak Identification}

The true value of the structural parameter $\lambda $ and the reduced-form
parameter $\omega $ satisfies the link function $g(\lambda _{0},\omega
_{0})=0.$In a standard problem without any identification issue, we can
estimate $\lambda _{0}$ by the minimum distance estimator $\widehat{\lambda }%
=(\widehat{\theta },\widehat{\pi },\widehat{\phi })$ that minimizes $%
Q_{T}(\lambda )=g(\lambda ,\widehat{\omega })^{\prime }W_{T}g(\lambda ,%
\widehat{\omega })$ for some weighting matrix $W_{T}$ and construct tests
and confidence sets for $\lambda _{0}$ based on the asymptotic normal
distribution of $T^{1/2}(\widehat{\lambda }-\lambda _{0})$. However, this
standard method does not work in the present problem when $\pi _{0}$ is only
weak identified. In this case, $g(\lambda ,\widehat{\omega })$ is almost
flat in $\pi $ and the minimum distance estimator of $\widehat{\pi }$ is not
even consistent. To make the problem even more complicated, the
inconsistency of $\widehat{\pi }$ has a spillover effect on $\widehat{\theta 
}$ and $\widehat{\phi },$ making the distribution of $\widehat{\theta }$ and 
$\widehat{\phi }$ non-normal even in large sample.

Before presenting the robust test, we first introduce some useful quantities
and provide some heuristic discussions of the identification problem and its
consequence. Let $G(\lambda ,\omega )$ denote the partial derivative of $%
g(\lambda ,\omega )$ wrt $\omega .$ Let $g_{0}(\lambda )=g(\lambda ,\omega
_{0})$ and $G_{0}(\lambda )=G(\lambda ,\omega _{0})$ be the link function
and its derivative evaluated at $\omega _{0}$ and $\widehat{g}(\lambda
)=g(\lambda ,\widehat{\omega })$ and $\widehat{G}(\lambda )=G(\lambda ,%
\widehat{\omega })$ be the same quantities evaluate at the estimator $%
\widehat{\omega }.$ The delta method gives 
\begin{equation}
\eta _{T}(\lambda ):=T^{1/2}\left[ \widehat{g}(\lambda )-g_{0}(\lambda )%
\right] =G_{0}(\lambda )\Omega ^{1/2}\cdot \xi _{T}+o_{p}(1),
\label{emp pro}
\end{equation}%
where $\xi _{T}\rightarrow _{d}N(0,I)$ following Lemma \ref{Lemma Reduce}.
Thus, $\eta _{T}(\cdot )$ weakly converges to a Gaussian process $\eta
(\cdot )$ with covariance function $\Sigma (\lambda _{1},\lambda
_{2})=G_{0}(\lambda _{1})\Omega G_{0}(\lambda _{2})^{\prime }.$

Following (\ref{emp pro}), we can write $T^{1/2}\widehat{g}(\lambda )=\eta
_{T}(\lambda )+T^{1/2}g_{0}(\lambda ),$ where $\eta _{T}(\lambda )$ is the
noise from the reduced-form parameter estimation and $T^{1/2}g_{0}(\lambda )$
is the signal from the link function. Under weak identification, $%
g_{0}(\lambda )$ is almost flat in $\lambda ,$ modelled by the signal $%
T^{1/2}g_{0}(\lambda )$ being finite even for $\lambda \neq \lambda _{0}$
and $T\rightarrow \infty .$ Thus, the signal and the noise are of the same
order of magnitude, yielding an inconsistent minimum distance estimator $%
\widehat{\lambda }.$ This is in contrast with the strong identification
scenario, where $T^{1/2}g_{0}(\lambda )\rightarrow \infty $ for $\lambda
\neq \lambda _{0}$ as $T\rightarrow \infty $ and $g_{0}(\lambda _{0})=0.$ In
this case, the signal is so strong that the minimum distance estimator is
consistent.

The identification strength of $\lambda _{0}$ is determined by the function $%
T^{1/2}g_{0}(\lambda ).$ However, this function is unknown and cannot be
consistently estimated (due to $T^{1/2}$). Thus, we take the conditional
inference procedure as in Andrews and Mikusheva (2016)\ and view $%
T^{1/2}g_{0}(\lambda )$ as an infinite dimensional nuisance parameter for
the inference for $\lambda _{0}$. The goal is to control robust confidence
set (CS) for $\lambda _{0}$ that has correct size asymptotically regardless
of this unknown nuisance parameter.

\subsection{Conditional QLR\ Test}

We construct a confidence set for $\lambda $ by inverting the test $%
H_{0}:\lambda =\lambda _{0}$ vs $H_{1}:\lambda \neq \lambda _{0}$. The test
statistic is a QLR\ statistic that takes the form%
\begin{equation}
QLR(\lambda _{0})=T\widehat{g}(\lambda _{0})^{\prime }\widehat{\Sigma }%
(\lambda _{0},\lambda _{0})^{-1}\widehat{g}(\lambda _{0})-\underset{\lambda
\in \Lambda }{\min }T\widehat{g}(\lambda )^{\prime }\widehat{\Sigma }%
(\lambda ,\lambda )^{-1}\widehat{g}(\lambda ),  \label{QLR stat}
\end{equation}%
where $\widehat{\Sigma }(\lambda _{1},\lambda _{2},)=\widehat{G}(\lambda
_{1})\widehat{\Omega }\widehat{G}(\lambda _{2})^{\prime }$ and $\widehat{%
\Omega }$ is the consistent estimator of $\Omega $ defined above$.$

Andrews and Mikusheva (2016) provide the conditional QLR\ test in a
nonlinear GMM problem, where $\widehat{g}(\lambda )$ is replaced by a sample
moment. The same method can be applied to the present nonlinear minimum
distance problem. Following AM, we first project $\widehat{g}(\lambda )$
onto $\widehat{g}(\lambda _{0})$ and construct a residual process%
\begin{equation}
\widehat{r}(\lambda )=\widehat{g}(\lambda )-\widehat{\Sigma }(\lambda
,\lambda _{0})\widehat{\Sigma }(\lambda _{0},\lambda _{0})^{-1}\widehat{g}%
(\lambda _{0}).  \label{red process}
\end{equation}%
The limiting distribution of $\widehat{r}(\lambda )$ and $\widehat{g}%
(\lambda _{0})$ are Gaussian and independent. Thus, conditional on $\widehat{%
r}(\lambda ),$ the asymptotic distribution of $\widehat{g}(\lambda )$ no
longer depends on the nuisance parameter $T^{1/2}g_{0}(\lambda ),$ making
the procedure robust to all identification strength.

Specifically, we obtain the $1-\alpha $ conditional quantile of the QLR
statistic, denoted by $c_{1-\alpha }(r,\lambda _{0}),$ as follows. For $%
b=1,..,B,$ we take independent draws $\eta _{b}^{\ast }\sim N(0,\widehat{%
\Sigma }(\lambda _{0},\lambda _{0}))$ and produce a simulated process 
\begin{equation}
g_{b}^{\ast }(\lambda )=\widehat{r}(\lambda )+\widehat{\Sigma }(\lambda
,\lambda _{0})\widehat{\Sigma }(\lambda _{0},\lambda _{0})^{-1}\eta
_{b}^{\ast }
\end{equation}%
and a simulated statistic%
\begin{equation}
QLR_{b}^{\ast }(\lambda _{0})=T\widehat{g}(\lambda _{0})^{\prime }\widehat{%
\Sigma }(\lambda _{0},\lambda _{0})^{-1}\widehat{g}(\lambda _{0})-\underset{%
\lambda \in \Pi }{\min }Tg_{b}^{\ast }(\lambda )^{\prime }\widehat{\Sigma }%
(\lambda ,\lambda )^{-1}g_{b}^{\ast }(\lambda ).
\end{equation}%
Let $b_{0}=\lceil (1-\alpha )B\rceil ,$ the smallest integer no smaller than 
$(1-\alpha )B$. Then the critical value $c_{1-\alpha }(r,\lambda _{0})$ is
the $b_{0}^{th}$ smallest value among $\{QLR_{b}^{\ast },b=1,...,B\}.\Omega $

\smallskip

To sum up, we execute the following steps for a robust CS for $\lambda .$

\noindent (i) Estimate the reduced-form parameter $\widehat{\omega }=(%
\widehat{\omega }_{1},\widehat{\omega }_{2},\widehat{\omega }_{3})^{\prime }$
following the estimators defined in (\ref{omega 1 est}) and (\ref{omega 2
est}). Obtain a consistent estimator of its asymptotic covariance $\widehat{%
\Omega }=\widehat{\mathcal{B}}\widehat{V}\widehat{\mathcal{B}}^{\prime },$
where $\widehat{\mathcal{B}}$ is define in (\ref{Fhat}) and $\widehat{V}$ is
a HAC estimator of $V.$

\noindent For $\lambda _{0}\in \Lambda ,$ execute steps (ii)-(iv) below.

\noindent (ii) Construct the QLR statistic $QLR(\lambda _{0})$ in (\ref{QLR
stat}) using $g(\lambda ,\omega ),$ $G(\lambda ,\omega ),$ $\widehat{\omega }%
,$ and $\widehat{\Omega }.$

\noindent (iii) Compute the residual process $\widehat{r}(\lambda )$ in (\ref%
{red process}).

\noindent (iv) Given $\widehat{r}(\lambda ),$ compute the critical value $%
c_{1-\alpha }(r,\lambda _{0})$ described above.

\noindent (v) Repeat steps (ii)-(iv) for different values of $\lambda _{0}$.
Construct a confidence set by collecting the null values that are not
rejected, i.e., nominal level $1-\alpha $ confidence set for $\lambda _{0}$
is%
\begin{equation}
CS_{T}=\{ \lambda _{0}:QLR_{T}(\lambda _{0})\leq c_{1-\alpha }(r,\lambda
_{0})\}.
\end{equation}

\smallskip

To obtain confidence intervals for each element of $\lambda _{0},$ one
simple solution is to project the confidence set constructed above to each
axis. The resulting confidence interval also has correct coverage. An
alternative solution is to first concentrate out the nuisance parameters
before apply the conditional inference approach above, see Section 5\ of AM.
However, this concentration approach only works when the nuisance parameter
is strongly identified. In the present set-up, this approach does not work
for $\theta $ and $\phi $ because the nuisance parameter $\pi $ is weakly
identified.

\smallskip

\noindent \textbf{Assumption S}. The following conditions hold over $P\in 
\mathcal{P},$ for any $\lambda $ in its parameter space, and any $\omega $
in some fixed neighborhood around its true value, for some fixed $0<C<\infty
.$

\noindent (i) $g(\lambda ,\omega )$ is partially differentiable in $\omega ,$
with partial derivative $G(\lambda ,\omega )$ that satisfies $||G(\lambda
_{1},\omega )-G(\lambda _{2},\omega )||\leq C||\lambda _{1}-\lambda _{2}||$
and $||G(\lambda ,\omega _{1})-G(\lambda ,\omega _{2})||\leq C||\omega
_{1}-\omega _{2}||.$

\noindent (ii) $C^{-1}\leq \lambda _{\min }(G(\lambda ,\omega )^{\prime
}G(\lambda ,\omega ))\leq \lambda _{\max }(G(\lambda ,\omega )^{\prime
}G(\lambda ,\omega ))\leq C$.

\smallskip

\begin{lemma}
\label{Lemma CS}Suppose Assumption R and S hold. Then, 
\begin{equation*}
\underset{T\rightarrow \infty }{\lim \inf }\underset{P\in \mathcal{P}}{\inf }%
\Pr \left( \lambda _{0}\in CS_{T}\right) \geq 1-\alpha .
\end{equation*}
\end{lemma}

This Lemma states that the confidence set constructed by the conditional
QLR\ test has correct uniform asymptotic size. Uniformity is important for
this confidence set to cover the true parameter with a probability close to $%
1-\alpha $ in finite-sample. Most importantly, this uniform result is
established over a parameter $\mathcal{P}$ that is large enough to allow the
weak identification of the structural parameter $\lambda .\qquad QED$

\bigskip


\section{Conclusion}

we have reduce the overlap with the abstract and intro. I will think about it.

%In structural stochastic volatility models such as the one developed here, changes in the volatility affect returns through two channels. First, the investor's willingness to tolerate high volatility in order to get high expected returns as measured by the price of market risk. Standard economic models imply there is a static trade-off between risk and expected return. Consequently, when the risk changes the expected return must change to restore equilibrium. However, investors may also be directly averse to changes in future volatility, and the empirical evidence shows that they are. \Textcite{han2018leverage} shows that we can disentangle these two channels through their differing relationships to the leverage effect. However, estimating the leverage effect is quite tricky, \parencite{aitsahalia2013leverage}. This difficulty poses a subtle identification problem that invalidates standard inference. When the data's signal-to-noise ratio is small, as it is here, standard tests and confidence intervals provide misleading results. We adopt the discrete-time exponentially-affine of \textcite{han2018leverage} and adapt weak identification methods to this framework to ensure that the resulting confidence intervals are uniformly valid.
%
%In particular, we develop a minimum distance criterion that links the market risk price, the volatility risk price, and the leverage effect to some well-behaved reduced-form parameters that govern the return and volatility's joint distribution. We do this by adapting the conditional quasi-likelihood ratio test (CLR) \textcite{andrews2016conditional} develop in a GMM framework to a minimum distance framework. The resulting CLR test is uniformly valid. We invert this test to derive a robust confidence set. We then apply this methodology to data on the S\&P 500 and show that the market risk price lies between \num{0.20} and \num{0.60} yearly percentage points, and the volatility risk price lies between \num{-0.20} and \num{0.00} yearly percentage points. These estimates are both substantially smaller in magnitude than the values chosen by \textcite{han2018leverage}.
%

\clearpage

\begin{appendices}

\section{Proofs}\label{sec:proofs}

% \subsection{\texorpdfstring{\cref{Lemma Reduce}{Lemma 1}}}

\begin{proof}

    Under the assumption that (i) $ \mathbb{E(}z_{t}z_{t}^{\prime })$ has the smallest eigenvalue bounded away from 0 and (ii) $c>\varepsilon $ and $\delta >\varepsilon $ for some $ \varepsilon >0,$ we not only have $\omega _{10}$ as an uniquely minimizer of $||\mathbb{E}[h_{t}(\omega _{1})]||$ but also have a uniform positive lower bound for $\norm{\E[h_{t}(\omega _{1})]}$ for $\norm{\omega _{1}-\omega _{10}} \geq \varepsilon$. Thus, consistency of $\widehat{\omega }_{1}$ follows from standard arguments for the consistency of a GMM estimator under an uniform convergence of the criterion under Assumption R(i) and R(ii).

Let $\overline{h}(\omega _{1})=T^{-1}\sum_{t=1}^{T}h_{t}(\omega _{1})$ and $ \overline{H}(\omega )=T^{-1}\sum_{t=1}^{T}H_{t}(\omega _{1}).$ By construction, the estimator satisfies the first order condition
%
\begin{align}
    0 &= 
    \begin{pmatrix} 
        \overline{H}(\widehat{\omega }_{1})^{\prime }\widehat{V}_{1}^{-1}\overline{h} (\widehat{\omega }_{1}) \\ 
%
        T^{-1}\sum_{T=1}^{T}x_{t}(y_{t}-x_{t}^{\prime }\widehat{\omega }_{2}) \\ 
%
        \widehat{\omega }_{3}-T^{-1}\sum_{t=1}^{T}\left( y_{t}-\widehat{y} _{t}\right) ^{2} 
    \end{pmatrix} \nonumber \\ 
%
    &= 
%
    \begin{pmatrix}
%
        \overline{H}(\widehat{\omega }_{1})^{\prime }\widehat{V}_{1}^{-1}\overline{h} (\omega _{10})+\overline{H}(\widehat{\omega }_{1})^{\prime }\widehat{V} _{1}^{-1}\overline{H}(\widetilde{\omega }_{1})(\widehat{\omega }_{1}-\omega
_{10}) \\ 
%
        T^{-1}\sum_{t=1}^{T}x_{t}(y_{t}-x_{t}^{\prime }\omega _{20})-T^{-1}\sum_{t=1}^{T}x_{t}x_{t}^{\prime }\left( \widehat{\omega } _{2}-\omega _{20}\right) \\ 
%
        \left( \widehat{\omega }_{3}-\omega _{3}\right) +\omega _{3}-T^{-1}\sum_{t=1}^{T}\left( y_{t}-x_{t}\widehat{\omega }_{2}\right) ^{2}
    \end{pmatrix},
%
  \label{L-R-1}
\end{align}
%
where the second equality follows from a mean value expansion of $\overline{h }(\widehat{\omega }_{1})$ around $\omega _{10},$ with $\widetilde{\omega } _{1}$ between $\omega _{10}$ and $\widehat{\omega }_{1}$. 
Let
%
\begin{equation}
%
    \widetilde{\mathcal{B}} = \diag\left\lbrace[\overline{H}(\widehat{\omega }_{1})^{\prime } \widehat{V}_{1}^{-1}\overline{H}(\widetilde{\omega }_{1})]^{-1}\overline{H}( \widehat{\omega }_{1})^{\prime }\widehat{V}_{1}^{-1},[T^{-1} \sum_{t=1}^{T}x_{t}x_{t}^{\prime }]^{-1},1\right\rbrace.  
%
\end{equation}
%
Then (\ref{L-R-1}) implies that 
%
\begin{align}
    T^{1/2}\left( \widehat{\omega }-\omega \right) 
%    
    &= \widetilde{\mathcal{B}} \cdot T^{-1/2}\sum_{t=1}^{T} 
%
    \begin{pmatrix}
        -h_{t}(\omega _{10}) \\ 
%
        x_{t}(y_{t}-x_{t}^{\prime }\omega _{20}) \\ 
%
        \left( y_{t}-x_{t}\widehat{\omega }_{2}\right) ^{2}-\omega _{3}%
    \end{pmatrix}  \nonumber \\
%
    &=
%
    \widetilde{\mathcal{B}}\cdot T^{-1/2}\sum_{t=1}^{T} 
%
    \begin{pmatrix}
        -h_{t}(\omega _{10}) \\ 
%
        x_{t}(y_{t}-x_{t}^{\prime }\omega _{20}) \\ 
%
        \left( y_{t}-x_{t}^{\prime }\omega _{20}\right) ^{2}-\E\left[\left( y_{t}-x_{t}^{\prime }\omega _{20}\right)^{2}\right]%
%
    \end{pmatrix}%
%
    +
%
    \begin{pmatrix}
        0 \\ 
        0 \\ 
        \varepsilon_{T}
    \end{pmatrix},
%
    \label{L-R-2}
%
\end{align}%
%
where the second equality uses $\omega _{3}=\E[\left( y_{t}-x_{t}^{\prime }\omega _{20}\right) ^{2}]$ by definition and 
%
\begin{align}
    \varepsilon _{T} 
%
    &= T^{-1/2}\sum_{t=1}^{T}\left[ \left( y_{t}-x_{t}^{\prime } \widehat{\omega }_{2}\right) ^{2}-\left( y_{t}-x_{t}^{\prime }\omega _{20}\right) ^{2}\right]  \nonumber \\
%
    &= 2T^{-1}\sum_{t=1}^{T}\left( y_{t}-x_{t}^{\prime }\omega _{20}\right) x_{t}^{\prime }\left[ T^{1/2}\left( \widehat{\omega }_{2}-\omega _{20}\right) \right] +o_{p}(1)  \nonumber \\
%
    &= o_{p}(1)  
%
    \label{L-R-3}
\end{align}
%
because $T^{-1}\sum_{t=1}^{T}\left( y_{t}-x_{t}^{\prime }\omega _{20}\right) x_{t}^{\prime }\rightarrow _{p}0$ and $T^{1/2}(\widehat{\omega }_{2}-\omega _{20})=O_{p}(1)$ following Assumption R. 
In addition, 
%
\begin{equation}
    \widetilde{\mathcal{B}}\rightarrow _{p}\mathcal{B}  
    \label{L-R-4}
\end{equation}%
%
following from the consistency of $\widehat{\omega }_{1}$ and Assumption R.
Finally, the desirable result follows from (\ref{L-R-2})-(\ref{L-R-4}) and Assumption R. 
The consistency of $\widehat{\Omega }$ follows from the consistency of $\widehat{\mathcal{B}}$ and $\widehat{V}$. 

\end{proof}

\begin{proof}

\noindent Proof of Lemma \ref{Lemma CS}. 
We obtain this result by applying \textcite[Theorem 1]{andrews2016conditional}. 
We now verify Assumptions 1-3 in \textcite{andrews2016conditional}. 
To show weak convergence $\eta _{T}(\cdot )$ to $\eta (\cdot )$ uniformly over $\mathcal{P}$, note that by a second-order Taylor expansion,
%
\begin{align}
    \eta _{T}(\lambda) 
%
    &\coloneqq T^{1/2}\left[ \widehat{g}(\lambda )-g_{0}(\lambda ) \right] = G_{0}(\lambda )\Omega ^{1/2}\xi _{T}+\delta _{T},\text{ where} \nonumber \\
%
    \xi _{T} 
    &= \Omega ^{-1/2}T^{1/2}\left( \widehat{\omega }-\omega _{0}\right), \text{ and } 
%
    \delta _{T} =\left( G(\lambda ,  \widetilde{\omega })-G(\lambda, \omega _{0})\right) T^{1/2}(\widehat{\omega }-\omega _{0})
%
\end{align}%
%
and $\widetilde{\omega }$ is between $\widehat{\omega }$ and $\omega_{0}$.
%
Because $\norm{G(\lambda ,\widetilde{\omega })-G(\lambda ,\omega _{0})} \leq C \norm{\widetilde{\omega }-\omega_{0}}$, $\delta _{T}=o_{p}(1)$ uniformly over $ \mathcal{P}$ following Lemma \ref{Lemma Reduce}. 
To show $G_{0}(\lambda)\Omega ^{1/2}\xi _{T}$ weakly converges to $\eta (\cdot ),$ it is sufficient to show (i) the pointwise convergence%
%
\begin{equation}
% 
    \begin{pmatrix}
        G_{0}(\lambda _{1})\Omega ^{1/2}\xi _{T} \\ 
        G_{0}(\lambda _{2})\Omega ^{1/2}\xi _{T}%
    \end{pmatrix}%
%
    \rightarrow _{d}( 
%
    \begin{pmatrix}
        \eta (\lambda _{1}) \\ 
        \eta (\lambda _{2})%
    \end{pmatrix},
%
\end{equation}%
%
which follows from Lemma \ref{Lemma Reduce}, and (ii) the stochastic equicontinuity condition, i.e., for every $\varepsilon >0$ and $\xi >0,$ there exists a $\delta >0$ such that
%
\begin{equation}
    \underset{T\rightarrow \infty }{\lim\sup }\Pr \left( \underset{P\in \mathcal{P}}{\sup }\underset{\norm{\lambda _{1}-\lambda _{2}}\leq \delta }{\sup }\norm*{G_{0}(\lambda _{1})\Omega ^{1/2}\xi _{T}-G_{0}(\lambda
    _{2})\Omega ^{1/2}\xi _{T}} >\varepsilon \right) < \xi.
\end{equation}%
%
For some $C < \infty$, we have $\norm{G_{0}(\lambda _{1})-G(\lambda _{2})} \leq C \norm{\lambda _{1}-\lambda _{2}}$ under a uniform bound for the derivative in Assumption S, and we have $||\Omega ^{1/2}||\leq C$ under Assumption R
because $F$ and $V$ both have bounded largest eigenvalue. 
Thus,
%
\begin{align}
    &\phantom{=} \underset{T\rightarrow \infty }{\lim \sup }\Pr\left( \underset{P\in \mathcal{P}}{\sup }\underset{\norm{\lambda _{1}-\lambda _{2}}\leq \delta }{\sup }\norm*{ G_{0}(\lambda _{1})\Omega ^{1/2}\xi _{T}-G_{0}(\lambda _{2})\Omega ^{1/2}\xi _{T}} > \varepsilon \right)  \nonumber \\
%
    &\leq \underset{T\rightarrow \infty }{\lim \sup }\Pr \left( C^{2}\underset{ P\in \mathcal{P}}{\sup }\left \Vert \xi _{T}\right \Vert >\frac{\varepsilon }{\delta }\right). 
%
    \label{EC}
\end{align}
%
Because $\xi _{T}=O_{p}(1)$ uniformly over $P\in \mathcal{P},$ there exists $%
\delta $ such that $\varepsilon /\delta $ is large enough to make the right
hand side of the inequality in (\ref{EC}) smaller than $\xi .$

Assumptions 2 and 3 of \textcite[Theorem 1]{andrews2016conditional} follow from Assumption R. 
\end{proof}

Proof of Lemma \ref{Lemma m0 and m1}. 

\begin{proof}
For the risk free asset, $ r_{t+1}=0.$ 
Therefore, we have
%
\begin{align*}
    1 &= E\left[ \exp \left( m_{0}+m_{1}\sigma _{t}^{2}-\pi \sigma _{t+1}^{2}-\theta r_{t+1}\right) \mvert \F_{t}\right]  \\
%
    &= \exp (m_{0}+m_{1}\sigma _{t})E\left[ \exp \left( -\pi \sigma _{t+1}^{2}\right) E\left[ \exp \left( -\theta r_{t+1}\right) \mvert \F _{t},\sigma _{t+1}^{2}\right] \mvert \F_{t}\right]  \\
%
    &= \exp (m_{0}-E\left( \theta \right) +m_{1}\sigma _{t}-D\left( \theta \right) \sigma _{t}^{2})E\left[ \exp \left( -\pi \sigma _{t+1}^{2}-C\left( \theta \right) \sigma _{t+1}^{2}\right) \mvert \F_{t}\right]  \\
%
    &= \exp (m_{0}-E\left( \theta \right) +m_{1}\sigma _{t}-D\left( \theta \right) \sigma _{t}^{2}-A\left( \pi +C\left( \theta \right) \right) \sigma _{t}^{2}-B\left( \pi +C\left( \theta \right) \right) ),
%
\end{align*}
%
where the first equality follows from the pricing equation, the second equality follows from the law of iterated expectations, the third equation uses the Laplace transform for $r_{t+1}$ in \cref{eqn:return_laplace_transform}, and the last equality follows from the Laplace transform for $\sigma _{t+1}^{2}$ in \cref{eqn:vol_laplace_transform}. 
% TODO
We restrict the constant term and the coefficient for $\sigma_{t}^{2}$ to equal 0, which gives the claimed result for $m_{0}$ and $m_{1}.$

We can apply the same argument above to any asset $r_{t+1}$. This gives the same result, except $\theta$ is replaced by $\theta -1$ throughout. 
This implies that the two equalities for $m_{0}$ and $m_{1}$ also hold with $\theta $ replaced by $\theta -1$. 
Therefore, 
%
\begin{align*}
    E(\theta -1)+B\left( C\left( \theta -1\right) +\pi \right)  
%
    &= E(\theta)+B\left( C\left( \theta \right) +\pi \right) , \\
%
    D\left( \theta -1\right) +A\left( C\left( \theta -1\right) +\pi \right) 
%
    &= D\left( \theta \right) +A\left( C\left( \theta \right) +\pi \right).
\end{align*}
%
The claimed results for $\gamma $ and $\beta $ follow from $\gamma = \E(\theta)-E(\theta -1)$ and $\beta =D(\theta )-D(\theta -1)$ under the linear specification of $E(x)=\gamma x$ and $D(x)=\beta x$.

\end{proof}

\end{appendices}

\end{document}


